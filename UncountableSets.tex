\begin{section}{Uncountable Sets}

Recall from Definition~\ref{def:countable} that a set $A$ is \textbf{uncountable} iff $A$ is not countable.  Since all finite sets are countable, the only way a set could be uncountable is if it is infinite.  It follows that a set $A$ is uncountable iff there is never a bijection between $\mathbb{N}$ and $A$.  It's not clear that uncountable sets even exist!  It turns out that uncountable sets do exist and in this section, we will discover a few of them.

Our first task is to prove that the open interval $(0,1)$ is uncountable.  By Exercise~\ref{exer:someInfiniteSets}(h), we know that $(0,1)$ is an infinite set, so it is at least plausible that $(0,1)$ is uncountable.  The following problem outlines the proof of Theorem~\ref{thm:(0,1)uncountable}.  Our approach is often referred to as \textbf{Cantor's Diagonalization Argument}.

Before we get started, recall that every number in $(0,1)$ can be written in decimal form. However, there may be more than one way to write a given number in decimal form.  For example, $0.2$ equals $0.1\overline{99}$.  A number $x=0.a_1a_2a_3\ldots$ is said to be in \textbf{standard form} iff there is no $k$ such that for all $i>k$, $a_i=9$. That is, a decimal expansion is in standard form iff the expansion doesn't end with a repeating sequence of 9's. For example, $0.2$ is in standard form while $0.1\overline{99}$ is not, even though both represent the same number. It turns out that every real number can be expressed uniquely in standard form.

\begin{problem}
For sake of a contradiction, assume the interval $(0,1)$ is countable.  Then there exists a bijection $f:\mathbb{N}\to (0,1)$. For each $n\in\mathbb{N}$, its image under $f$ is some number in $(0,1)$.  Let $f(n):=0.a_{1n}a_{2n}a_{3n}\ldots$, where $a_{1n}$ is the first digit in the decimal form for the image of $n$, $a_{2n}$ is the second digit, and so on. If $f(n)$ terminates after $k$ digits, then our convention will be to continue the decimal form with 0's. Now, define $b=0.b_1b_2b_3\ldots$, where
\[
b_i=\begin{cases}
2, & \text{if }a_{ii}\neq 2\\
3, & \text{if }a_{ii}=2.
\end{cases}
\]
\begin{enumerate}[label=\textrm{(\alph*)}]
\item Prove that the decimal expansion that defines $b$ above is in standard form.
\item Prove that for all $n\in\mathbb{N}$, $f(n)\neq b$.
\item Prove that $f$ is not onto.
\item Explain why we have a contradiction.
\item Explain why it follows that the open interval $(0,1)$ cannot be countable.
\end{enumerate}
\end{problem}

The steps above prove the following theorem.

\begin{theorem}\label{thm:(0,1)uncountable}
The open interval $(0,1)$ is uncountable.
\end{theorem}

Loosely speaking, what Theorem~\ref{thm:(0,1)uncountable} says is that the open interval $(0,1)$ is ``bigger" in terms of the number of elements it contains than the natural numbers and even the rational numbers.  This shows that there are infinite sets of different sizes!

One consequence of Theorem~\ref{thm:(0,1)uncountable} is that we know there is at least one uncountable set.  The next three results are useful for finding other uncountable sets.

\begin{theorem}\label{thm:containsUncountableSubset}
If $A$ and $B$ are sets such that $A\subseteq B$ and $A$ is uncountable, then $B$ is uncountable.\footnote{\emph{Hint:} Try a proof by contradiction. Take a look at Theorem~\ref{thm:subsetsCountableSets}.}
\end{theorem}

\begin{corollary}
If $A$ and $B$ are sets such that $A$ is uncountable and $B$ is countable, then $A\setminus B$ is uncountable.
\end{corollary}

\begin{theorem}
If $f:A\to B$ is a one-to-one function and $A$ is uncountable, then $B$ is uncountable.
\end{theorem}

\begin{theorem}
The set $\mathbb{R}$ of real numbers is uncountable.  Moreover, $\card((0,1))=\card(\mathbb{R})$.\footnote{\emph{Hint:} To show that $\mathbb{R}$ is uncountable, appeal to Theorem~\ref{thm:containsUncountableSubset}. To show that $\card((0,1))=\card(\mathbb{R})$, consider the function $f:(0,1)\to \mathbb{R}$ defined via $f(x)=\tan(\pi x-\frac{\pi}{2})$. It is worth pointing out that proving $\card((0,1))=\card(\mathbb{R})$ automatically proves that $\mathbb{R}$ is uncountable.}
\end{theorem}

\begin{theorem}
If $a,b\in\mathbb{R}$ with $a<b$, then $(a,b)$, $[a,b]$, $(a,b]$, and $[a,b)$ are all uncountable.
\end{theorem}

\begin{theorem}
The set of irrational numbers is uncountable.
\end{theorem}

\begin{theorem}
The set $\mathbb{C}$ of complex numbers is uncountable.
\end{theorem}

\begin{problem}
Determine whether each of the following statements is true or false. If a statement is true, prove it.  If a statement is false, provide a counterexample.
\begin{enumerate}[label=\textrm{(\alph*)}]
\item If $A$ and $B$ are sets such that $A$ is uncountable, then $A\cup B$ is uncountable.
\item If $A$ and $B$ are sets such that $A$ is uncountable, then $A\cap B$ is uncountable.
\item If $A$ and $B$ are sets such that $A$ is uncountable, then $A\times B$ is uncountable.
\item If $A$ and $B$ are sets such that $A$ is uncountable, then $A\setminus B$ is uncountable.
\end{enumerate}
\end{problem}

\begin{problem}\label{prob:sequence01} 
Let $S$ be the set of infinite sequences of 0's and 1's. Determine whether $S$ is countable or uncountable and prove that your answer is correct. 
\end{problem}

It turns out that the two uncountable sets may or may not have the same cardinality.  Perhaps surprisingly, there are sets that are even ``bigger" than the set of real numbers. Given any set, we can always increase the cardinality by considering its power set.

\begin{theorem}\label{thm:cardPowerSet}
If $A$ is a set, then $\card(A)<\card(\mathcal{P}(A))$.\footnote{\emph{Hint:} Mimic Cantor's Diagonalization Argument for showing that the interval $(0,1)$ is uncountable.}
\end{theorem}

Recall that cardinality provides a way for talking about ``how big" a set is. The fact that the natural numbers and the real numbers have different cardinality (one countable, the other uncountable), tells us that there are at least two different ``sizes of infinity".  Theorem~\ref{thm:cardPowerSet} tells us that there are infinitely many ``sizes of infinity."

\begin{theorem}
Consider the set $S$ from Problem~\ref{prob:sequence01}. Then $\card(\mathcal{P}(\mathbb{N}))=\card(S)$.
\end{theorem}

\end{section}