\documentclass[12pt,oneside]{book}

\usepackage[scale=2]{ccicons}
\usepackage{booktabs}
\usepackage{enumitem}
\usepackage{newclude}
\usepackage{multicol}
\usepackage{tabu}
\usepackage[table]{xcolor}
\usepackage{tikz}
\usetikzlibrary{arrows,automata,positioning,fit,shapes}
\usepackage{rotating}
\usepackage[notextcomp]{kpfonts} 
\usepackage{graphicx}
\usepackage{eurosym}
\usepackage{amsfonts}
\usepackage{amsmath}
\usepackage{amssymb}
\usepackage{stmaryrd}
\usepackage{wasysym}
\usepackage{amsthm}
\usepackage[margin=1in]{geometry}
\usepackage[hang,flushmargin]{footmisc}
\usepackage{color}
\definecolor{darkblue}{rgb}{0, 0, .6}
\definecolor{grey}{rgb}{.7, .7, .7}
\usepackage[breaklinks]{hyperref}
\hypersetup{
	colorlinks=true,
	linkcolor=darkblue,
	anchorcolor=darkblue,
	citecolor=darkblue,
	pagecolor=darkblue,
	urlcolor=darkblue,
	pdftitle={},
	pdfauthor={},
    bookmarksnumbered
}

\usepackage{fancyhdr}
\pagestyle{fancy}
\lhead{\leftmark}
\chead{}
\rhead{}
\lfoot{}
\cfoot{}
\rfoot{}

\theoremstyle{definition}
\newtheorem{theorem}{Theorem}[chapter]
\newtheorem{acknowledgement}[theorem]{Acknowledgement}
\newtheorem{algorithm}[theorem]{Algorithm}
\newtheorem{axiom}[theorem]{Axiom}
\newtheorem{case}[theorem]{Case}
\newtheorem{claim}[theorem]{Claim}
\newtheorem{conclusion}[theorem]{Conclusion}
\newtheorem{condition}[theorem]{Condition}
\newtheorem{conjecture}[theorem]{Conjecture}
\newtheorem{corollary}[theorem]{Corollary}
\newtheorem{criterion}[theorem]{Criterion}
\newtheorem{definition}[theorem]{Definition}
\newtheorem{example}[theorem]{Example}
\newtheorem{exercise}[theorem]{Exercise}
\newtheorem{journal}[theorem]{Journal}
\newtheorem{lemma}[theorem]{Lemma}
\newtheorem{notation}[theorem]{Notation}
\newtheorem{problem}[theorem]{Problem}
\newtheorem{proposition}[theorem]{Proposition}
\newtheorem{remark}[theorem]{Remark}
\newtheorem{solution}[theorem]{Solution}
\newtheorem{summary}[theorem]{Summary}
\newtheorem{skeleton}[theorem]{Skeleton Proof}
\newtheorem{activity}[theorem]{Activity}
\newtheorem{intuitivedef}[theorem]{Intuitive Definition}
\newtheorem{question}[theorem]{Question}

\newsavebox{\savepar}
\newenvironment{textbox}{\noindent\begin{lrbox}{\savepar}\begin{minipage}[c]{.98\textwidth}}{\end{minipage}\end{lrbox}\fcolorbox{black}{white}{\usebox{\savepar}}}

\newcommand{\dom}{\operatorname{Dom}}
\newcommand{\codom}{\operatorname{Codom}}
\newcommand{\range}{\operatorname{Rng}}
\newcommand{\Spin}{\operatorname{Spin}}
\newcommand{\lcm}{\operatorname{lcm}}

\begin{document}

\title{An Introduction to Proof via Inquiry-Based Learning}
\author{Dana C.~Ernst, PhD\\
Northern Arizona University}
\date{Spring 2017}

\maketitle
\thispagestyle{empty}

\noindent\copyright{ \the\year\ Dana C.~Ernst.  Some Rights Reserved.\\

\bigskip

\noindent This work is licensed under the Creative Commons Attribution-Share Alike 4.0 United States License.  You may copy, distribute, display, and perform this copyrighted work, but only if you give credit to Dana C.~Ernst, and all derivative works based upon it must be published under the Creative Commons Attribution-Share Alike 4.0 International License. Please attribute this work to Dana C.~Ernst, Mathematics Faculty at Northern Arizona University, \url{dana.ernst@nau.edu}. To view a copy of this license, visit
\begin{center}
\url{https://creativecommons.org/licenses/by-sa/4.0/}
\end{center}
or send a letter to Creative Commons, 171 Second Street, Suite 300, San Francisco, California, 94105, USA.}

\medskip

\begin{center}
\ccbysa
\end{center}

\noindent Below is a partial list of people that I need to thank for supplying content, advice, and feedback.
\begin{itemize}
\item \href{http://www.stanyoshinobu.com}{Stan Yoshinobu} (Cal Poly) and \href{http://www4.csudh.edu/library/info/civic-directory/f-j/matthew-g-jones}{Matthew Jones} (CSU, Dominguez Hills). A few of the sections were originally adaptations of notes written by Stan and Matt. The first version of these notes relied heavily on their work.
\item \href{http://users.dickinson.edu/~richesod/}{Dave Richeson} (Dickinson College). Dave is responsible for much of the content in Chapter~\ref{chap:three famous theorems}: Three Famous Theorems, Appendix~\ref{appendix:fancy_math_terms}: Fancy Mathematical Terms, and Appendix~\ref{appendix:definitions}: Definitions in Mathematics.
\item \href{http://jasongrout.org}{Jason Grout} (Bloomberg, L.P.).  I'm extremely grateful to Jason for helping me with a variety of technical aspects of writing these notes. 
\item \href{http://www.paulellis.org}{Paul Ellis} (Manhattanville College). Paul corrected several typos and provided lots of useful feedback.
%\item Thanks to Tenner and Petersen
%\item Thanks to R.L. Jayne (complete induction)
\item \href{http://home.snc.edu/andershendrickson/}{Anders Hendrickson} (St.~Norbert College). Anders is the original author of the content in Appendix~\ref{appendix:elements_of_style}: Elements of Style for Proofs. The current version in Appendix~\ref{appendix:elements_of_style} is a result of modifications made by myself with some suggestions from Dave Richeson.
\end{itemize}

\tableofcontents
\thispagestyle{empty}

\chapter{Introduction}\label{chap:intro}
%\thispagestyle{empty}

\begin{section}{What is This Course All About?}%new title?

The foundations of mathematics refers to logic and set theory; the axioms of number and space.  Also, it refers to an introduction to the techniques of proof, and at a larger level the process of \emph{doing Mathematics}.  Proof is central to doing mathematics.

Up to this point, it is likely that your experience of mathematics has been about using formulas and algorithms. That is only one part of mathematics. Mathematicians do much more than just use formulas.  Mathematicians experiment, make conjectures, write definitions, and prove theorems.  In this class, then, we will learn about doing all of these things.

What will this class require?  Daily practice.  Just like learning to play an instrument or sport, you will have to learn new skills and ideas.  Sometimes you'll feel good, sometimes frustrated.  You'll probably go through a range of feelings from being exhilarated, to being stuck.  Figuring it out, victories, defeats, and all that is part of real life is what you can expect.  Most importantly it will be rewarding.  Learning mathematics requires dedication.  It will require that you be patient despite periods of confusion.  It will require that you persevere in order to understand.  As the instructor, I am here to guide you, but I cannot do the learning for you, just as a music teacher cannot move your fingers and your heart for you.  Only you can do that.  I can give suggestions, structure the course to assist you, and try to help you figure out how to think through things.  Do your best, be prepared to put in a lot of time, and do all the work.  Ask questions in class, ask questions in office hours, and ask your classmates questions.  When you work hard and you come to understand, you feel good about yourself.  In the meantime, you have to believe that your work will pay off in intellectual development.

How will this class be organized?  You have probably heard that mathematics is not a spectator sport.  Our focus in this class is on learning to DO mathematics, not learning to sit patiently while others do it.  Therefore, class time will be devoted to working on problems, and especially on students presenting conjectures and proofs to the class, asking questions of presenters in order to understand their work and their thinking, and sharing and clarifying our thinking and understanding of each other's ideas.  

The class is fueled by your ability to prove theorems and share your ideas.  As we progress, you will find that you have ideas for proofs, but you are unsure of them.  In that case, you can either bring your idea to the class, or you can bring it to office hours.  By coming to office hours, you have a chance to refine your ideas and get individual feedback before bringing them to the class.  The more you use office hours, the more you will learn.  If the whole class is stuck, we can work on some ego-booster problems to get your ideas flowing.

Finally, this is a very exciting time in your mathematical career.  It's where you learn what mathematics is really about!

\epigraph{The mathematician does not study pure mathematics because it is useful; he studies it because he delights in it, and he delights in it because it is beautiful.}{\emph{Henri Poincar\'e}}

\end{section}

\begin{section}{An Inquiry-Based Approach}

This is not a lecture-oriented class or one in which mimicking prefabricated examples will lead you to success. You will be expected to work actively to construct your own understanding of the topics at hand with the readily available help of me and your classmates. Many of the concepts you learn and problems you work on will be new to you and ask you to stretch your thinking. You will experience \emph{frustration} and \emph{failure} before you experience \emph{understanding}. This is part of the normal learning process. If you are doing things well, you should be confused at different points in the semester. The material is too rich for a human being to completely understand it immediately. Your viability as a professional in the modern workforce depends on your ability to embrace this learning process and make it work for you.

\epigraph{Don't fear failure.  Not failure, but low aim, is the crime. In great attempts it is glorious even to fail.}{\emph{Bruce Lee}}

In order to promote a more active participation in your learning, we will incorporate ideas from an educational philosophy called inquiry-based learning (IBL).  Loosely speaking, IBL is a student-centered method of teaching mathematics that engages students in sense-making activities.  Students are given tasks requiring them to solve problems, conjecture, experiment, explore, create, and communicate.  Rather than showing facts or a clear, smooth path to a solution, the instructor guides and mentors students via well-crafted problems through an adventure in mathematical discovery.  According to \href{https://www.colorado.edu/eer/sites/default/files/attached-files/laursenrasmussencommentaryauthorversion0219.pdf}{Laursen and Rasmussen (2019)}, the Four Pillars of IBL are:
\begin{itemize}
\item Students engage deeply with coherent and meaningful mathematical tasks.
\item Students collaboratively process mathematical ideas.
\item Instructors inquire into student thinking.
\item Instructors foster equity in their design and facilitation choices.
\end{itemize}

Much of the course will be devoted to students presenting their proposed solutions or proofs on the board and a significant portion of your grade will be determined by how much mathematics you produce.  I use the word \emph{produce} because I believe that the best way to learn mathematics is by doing mathematics.  Someone cannot master a musical instrument or a martial art by simply watching, and in a similar fashion, you cannot master mathematics by simply watching; you must do mathematics!

In any act of creation, there must be room for experimentation, and thus allowance for mistakes, even failure. A key goal of our community is that we support each other---sharpening each other's thinking but also bolstering each other's confidence---so that we can make failure a productive experience. Mistakes are inevitable, and they should not be an obstacle to further progress. It's normal to struggle and be confused as you work through new material. Accepting that means you can keep working even while feeling stuck, until you overcome and reach even greater accomplishments.

\epigraph{You will become clever through your mistakes.}{\emph{German Proverb}}

Furthermore, it is important to understand that solving genuine problems is difficult and takes time.  You shouldn't expect to complete each problem in 10 minutes or less.  Sometimes, you might have to stare at the problem for an hour before even understanding how to get started.

In this course, everyone will be required to
\begin{itemize}
\item read and interact with course notes and textbook on your own;
\item write up quality solutions/proofs to assigned problems;
\item present solutions/proofs on the board to the rest of the class;
\item participate in discussions centered around a student's presented solution/proof;
\item call upon your own prodigious mental faculties to respond in flexible, thoughtful, and creative ways to problems that may seem unfamiliar on first glance.
\end{itemize}
As the semester progresses, it should become clear to you what the expectations are.

\epigraph{Tell me and I forget, teach me and I may remember, involve me and I learn.}{\emph{Benjamin Franklin}}

\end{section}

\begin{section}{Rights of the Learner}
As a student in this class, you have the right:
\begin{enumerate}
\item to be confused,
\item to make a mistake and to revise your thinking,
\item to speak, listen, and be heard, and
\item to enjoy doing mathematics.
\end{enumerate}

\epigraph{You may encounter many defeats, but you must not be defeated.}{\emph{Maya Angelou}}
	
\end{section}

\begin{section}{Your Toolbox, Questions, and Observations}

Throughout the semester, we will develop a list of \emph{tools} that will help you understand and do mathematics. Your job is to keep a list of these tools, and it is suggested that you keep a running list someplace.

Next, it is of utmost importance that you work to understand every proof. (Every!)  Questions are often your best tool for determining whether you understand a proof.  Therefore, here are some sample questions that apply to any proof that you should be prepared to ask of yourself or the presenter:
\begin{itemize}
\item What method(s) of proof are you using?
\item What form will the conclusion take?
\item How did you know to set up that [equation, set, whatever]?
\item How did you figure out what the problem was asking?
\item Was this the first thing you tried?
\item Can you explain how you went from this line to the next one?
\item What were you thinking when you introduced this?
\item Could we have \ldots instead?
\item Would it be possible to \ldots?
\item What if \ldots?
\end{itemize}

Another way to help you process and understand proofs is to try and make observations and connections between different ideas, proof statements and methods, and to compare approaches used by different people. Observations might sound like some of the following:
\begin{itemize}
\item When I tried this proof, I thought I needed to \ldots But I didn't need that, because \ldots
\item I think that \ldots is important to this proof, because \ldots
\item When I read the statement of this theorem, it seemed similar to this earlier theorem. Now I see that it [is/isn't] because \ldots
\end{itemize}

\end{section}

\begin{section}{Rules of the Game}
You should \emph{not} look to resources outside the context of this course for help. That is, you should not be consulting the Internet, other texts, other faculty, or students outside of our course. On the other hand, you may use each other, the course notes, me, and your own intuition.  In this class, earnest failure outweighs counterfeit success; you need not feel pressure to hunt for solutions outside your own creative and intellectual reserves.  For more details, check out the Syllabus.

\end{section}

\begin{section}{Structure of the Notes}

As you read the notes, you will be required to digest the material in a meaningful way.  It is your responsibility to read and understand new definitions and their related concepts.  However, you will be supported in this sometimes difficult endeavor. In addition, you will be asked to complete exercises aimed at solidifying your understanding of the material.  Most importantly, you will be asked to make conjectures, produce counterexamples, and prove theorems.

Most items in the notes are labelled with a number.  The items labelled as \textbf{Definition} and \textbf{Example} are meant to be read and digested.  However, the items labelled as \textbf{Exercise}, \textbf{Question}, \textbf{Theorem}, \textbf{Corollary}, and \textbf{Problem} require action on your part.  In particular, items labelled as \textbf{Exercise} are typically computational in nature and are aimed at improving your understanding of a particular concept.  There are very few items in the notes labelled as \textbf{Question}, but in each case it should be obvious what is required of you.  Items with the \textbf{Theorem} and \textbf{Corollary} designation are mathematical facts and the intention is for you to produce a valid proof of the given statement.  The main difference between a \textbf{Theorem} and \textbf{Corollary} is that corollaries are typically statements that follow quickly from a previous theorem.  In general, you should expect corollaries to have very short proofs.  However, that doesn't mean that you can't produce a more lengthy yet valid proof of a corollary.  The items labelled as \textbf{Problem} are sort of a mixed bag.  In many circumstances, I ask you to provide a counterexample for a statement if it is false or to provide a proof if the statement is true.  Usually, I have left it to you to determine the truth value.  If the statement for a problem is true, one could relabel it as a theorem.

It is important to point out that there are very few examples in the notes.  This is intentional.  One of the goals of the items labelled as \textbf{Exercise} is for you to produce the examples.

Lastly, there are many situations where you will want to refer to an earlier definition or theorem/corollary/problem.  In this case, you should reference the statement by number.  For example, you might write something like, ``By Theorem 2.14, we see that\ldots."

\end{section}

\begin{section}{Some Minimal Guidance}
Especially in the opening sections, it won't be clear what facts from your prior experience in mathematics we are ``allowed" to use.  Unfortunately, addressing this issue is difficult and is something we will sort out along the way.  However, in general, here are some minimal and vague guidelines to keep in mind.  

First, there are times when we will need to do some basic algebraic manipulations.  You should feel free to do this whenever the need arises.  But you should show sufficient work along the way.  You do not need to write down justifications for basic algebraic manipulations (e.g., adding 1 to both sides of an equation, adding and subtracting the same amount on the same side of an equation, adding like terms, factoring, basic simplification, etc.).  

On the other hand, you do need to make explicit justification of the logical steps in a proof.  When necessary, you should cite a previous definition, theorem, etc. by number.

Unlike the experience many of you had writing proofs in geometry, our proofs will be written in complete sentences.  You should break sections of a proof into paragraphs and use proper grammar.  There are some pedantic conventions for doing this that I will point out along the way.  Initially, this will be an issue that most students will struggle with, but after a few weeks everyone will get the hang of it.

Ideally, you should rewrite the statements of theorems before you start the proof.  Moreover, for your sake and mine, you should label the statement with the appropriate number.  I will expect you to indicate where the proof begins by writing ``\emph{Proof.}" at the beginning.  Also, we will conclude our proofs with the standard ``proof box" (i.e., $\square$ or $\blacksquare$), which is typically right-justified.

Lastly, every time you write a proof, you need to make sure that you are making your assumptions crystal clear.  Sometimes there will be some implicit assumptions that we can omit, but at least in the beginning, you should get in the habit of stating your assumptions up front.  Typically, these statements will start off ``Assume\ldots" or ``Let\ldots".  

This should get you started.  We will discuss more as the semester progresses.  Now, go have fun and start exploring mathematics!

\epigraph{If you want to sharpen a sword, you have to remove a little metal.}{\emph{Unknown}}

\end{section}
\include*{IntroToMath}
\include*{TasteNumberTheory}
\include*{IntroToLogic}
\include*{NegatingAndContradiction}
\include*{IntroQuantification}
\include*{MoreQuantification}
\include*{IntroSetTheoryTopology}
\include*{Sets}
\include*{PowerSetsParadoxes}
\include*{IndexingSets}
\include*{Topology}
\include*{ThreeFamousTheorems}
\include*{FundamentalTheoremArithmetic}
\include*{IrrationalityRoot2}
\include*{InfinitudeOfPrimes}
\include*{Induction}
\include*{IntroInduction}
\include*{MoreInduction}
\include*{CompleteInduction}
\include*{Relations}
\include*{IntroRelations}
\include*{EquivalenceRelations}
\include*{Partitions}
\include*{Posets}
\include*{Functions}
\include*{IntroFunctions}
\include*{CompositionsInverses}
%\include*{OrderRelations}
%Appendices
\appendix
\chapter{Elements of Style for Proofs}
\label{appendix:elements_of_style}

Mathematics is about discovering proofs and writing them clearly and compellingly. The following guidelines apply whenever you write a proof.  Keep these guidelines handy so that you may refer to them as you write your proofs.

\begin{enumerate}

\item \textbf{The burden of communication lies on you, not on your reader.}
It is your job to explain your thoughts; it is not your reader's job to guess them from a few hints. You are trying to convince a skeptical reader who does not believe you, so you need to argue with airtight logic in crystal clear language; otherwise the reader will continue to doubt. If you did not write something on the paper, then (a) you did not communicate it,(b) the reader did not learn it, and (c) the grader has to assume you did not know it in the first place.
          
\item \textbf{Tell the reader what you are proving or citing.}
The reader does not necessarily know or remember what ``Theorem 2.13'' is. Even a professor grading a stack of papers might lose track from time to time. Therefore, the statement you are proving should be on the same page as the beginning of your proof. 

In most proofs you will want to refer to an earlier definition, problem, theorem, or corollary.  In this case, you should reference the statement by number, but it is also helpful to the reader to summarize the statement you are citing.  For example, you might write something like, ``By Theorem~2.3, the sum of two consecutive integers is odd, and so\ldots."

\item \textbf{Use English words.}
Although there will usually be equations or mathematical statements in your proofs, use English sentences to connect them and display their logical relationships. If you look at proofs in textbooks and research papers, you will see that they consist mostly of English words.

\item \textbf{Use complete sentences.}
If you wrote a history essay in sentence fragments, the reader would not understand what you meant; likewise in mathematics you must use complete sentences, with verbs, to convey your logical train of thought.
        
Some complete sentences can be written purely in mathematical symbols, such as equations (e.g., $a^3=b^{-1}$), inequalities (e.g., $x<5$), and other relations (like $5\big|10$ or $7\in\mathbb{Z}$). These statements usually express a relationship between two mathematical \emph{objects}, like numbers or sets.  However, it is considered bad style to begin a sentence with symbols.  A common phrase to use to avoid starting a sentence with mathematical symbols is ``We see that...''.

\item \textbf{Show the logical connections among your sentences.}
Use phrases like ``Therefore'', ``Thus", ``Hence", ``Then", ``since", ``because'', ``if\ldots, then\ldots'', or ``if and only if'' to connect your sentences.
  
\item \textbf{Know the difference between statements and objects.}
A mathematical object is a \emph{thing}, a noun, such as a set, an element, a number, an ordered pair, a vector space, etc. Objects either exist or do not exist. Statements, on the other hand, are mathematical \emph{sentences}:  they are either true or false.
        
When you see or write a cluster of math symbols, be sure you know whether it is an object (e.g., ``$x^2+3$'') or a statement (e.g., ``$x^2+3<7$''). One way to tell is that every mathematical statement includes a verb, such as $=$, $\leq$, $\in$, ``divides'', etc.
        
\item \textbf{The symbol ``$=$'' means ``equals".}
Do not write $A=B$ unless you mean that $A$ actually equals $B$. This guideline seems obvious, but there is a great temptation to be sloppy.  In calculus, for example, some people might write $f(x)=x^{2}=2x$ (which is false), when they really mean that ``if $f(x)=x^{2}$, then $f'(x)=2x$.''

\item \textbf{Do not interchange ${=}$ and ${\implies}$.}
The equals sign connects two \emph{objects}, as in ``$x^2=b$''; the symbol ``$\implies$'' is an abbreviation for ``implies'' and connects two \emph{statements}, as in ``$a+b=a \implies b=0$.''  You should avoid using $\implies$ in formal write-ups of proofs.

\item \textbf{Avoid logical symbols in your proofs.}  
Similar to $\implies$, you should avoid using the logical symbols $\forall, \exists, \vee, \wedge$, and $\logeq$ in your formal write-ups.  These symbols are useful for abbreviating in your scratch work. 

\item \textbf{Say exactly what you mean.}
Just as $=$ is sometimes abused, so too people sometimes write $A\in B$ when they mean $A\subseteq B$, or write $a_{ij}\in A$ when they mean that $a_{ij}$ is an entry in matrix $A$. Mathematics is a very precise language, and there is a way to say exactly what you mean; find it and use it.

\item \textbf{Do not utilize anything unproven.}
Every statement in your proof should be something you \emph{know} to be true. The reader expects your proof to be a series of statements, each proven by the statements that came before it. If you ever need to write something you do not yet know is true, you \emph{must} preface it with words like ``assume,'' ``suppose,'' or ``if'' if you are temporarily assuming it, or with words like ``we need to show that'' or ``we claim that'' if it is your goal. Otherwise, the reader will think they have missed part of your proof.

\item \textbf{Write strings of equalities (or inequalities) in the proper order.}
When your reader sees something like
\[
A=B\leq C=D,
\]
they expect to understand easily why $A=B$, why $B\leq C$, and why $C=D$, and they expect the point of the entire line to be the more complicated fact that $A\leq D$. For example, if you were computing the distance $d$ of the point $(12,5)$ from the origin, you could write
\[
d = \sqrt{12^2+5^2} = 13.
\]
In this string of equalities, the first equals sign is true by the Pythagorean theorem, the second is just arithmetic, and the conclusion is that the first item equals the last item: $d=13$.
        
A common error is to write strings of equations in the wrong order. For example, if you were to write ``$\sqrt{12^2+5^2}=13=d$'', your reader would understand the first equals sign, would be baffled as to how we know $d=13$, and would be utterly perplexed as to why you wanted or needed to go through $13$ to prove that $\sqrt{12^2+5^2}=d$.

\item \textbf{Avoid circularity.}  Be sure that no step in your proof makes use of the conclusion!
        
\item \textbf{Do not write the proof backwards.}
Beginning students often attempt to write ``proofs'' like the following, which attempts to prove that $\tan^2(x)  = \sec^2(x) - 1$:
\begin{align*}
\tan^2(x) & = \sec^2(x) - 1 \\
\left(\frac{\sin(x)}{\cos(x)}\right)^2 & = \frac{1}{\cos^2(x)} - 1 \\
\frac{\sin^2(x)}{\cos^2(x)} & =  \frac{1-\cos^2(x)}{\cos^2(x)} \\
\sin^2(x) & = 1-\cos^2(x) \\
\sin^2(x) + \cos^2(x) & = 1 \\
1 & = 1
\end{align*}        
Notice what has happened here:  the student \emph{started} with the conclusion, and deduced the true statement ``$1=1$.'' In other words, they have proved ``If $\tan^2(x) = \sec^2(x) - 1$, then $1=1$,'' which is true but highly uninteresting.
        
Now this is not a bad way of \emph{finding} a proof. Working backwards from your goal often is a good strategy \emph{on your scratch paper}, but when it is time to \emph{write} your proof, you have to start with the hypotheses and work to the conclusion.

Here is an example of a suitable proof for the desired result, where each expression follows from the one immediately proceeding it:
\begin{align*}
\sec^2(x) - 1 & = \frac{1}{\cos^2(x)} - 1\\
& = \frac{1-\cos^2(x)}{\cos^2(x)} \\
& = \frac{\sin^2(x)}{\cos^2(x)} \\
& = \left(\frac{\sin(x)}{\cos(x)}\right)^2 \\
& = \left(\tan(x)\right)^2 \\
& = \tan^2(x).
\end{align*}

\item \textbf{Be concise.}
Many beginning proof writers err by writing their proofs too short, so that the reader cannot understand their logic. It is nevertheless quite possible to be too wordy, and if you find yourself writing a full-page essay, it is possible that you do not really have a proof, but just some intuition. When you find a way to turn that intuition into a formal proof, it will be much shorter.

\item \textbf{Introduce every symbol you use.}
If you use the letter ``$k$,'' the reader should know exactly what $k$ is. Good phrases for introducing symbols include ``Let $n\in \mathbb{N}$,'' ``Let $k$ be the least integer such that\ldots,'' ``For every real number $a$\ldots,'' and ``Suppose $A\subseteq\mathbb{R}\ldots$".
          
\item \textbf{Use appropriate quantifiers (once).}
When you introduce a variable $x\in S$, it must be clear to your reader whether you mean ``for all $x\in S$'' or just ``for some $x\in S$.'' If you just say something like ``$y=x^2$ where $x\in S$,'' the word ``where'' does not indicate whether you mean ``for all'' or ``some''.
        
Phrases indicating the quantifier ``for all'' include ``Let $x\in S$''; ``for all $x\in S$''; ``for every $x\in S$''; ``for each $x\in S$''; etc. Phrases indicating the quantifier ``some'' or ``there exists'' include ``for some $x\in S$''; ``there exists an $x\in S$''; ``for a suitable choice of $x\in S$''; etc.

Once you have said ``Let $x\in S$,'' the letter $x$ has its meaning defined. You do not need to say ``for all $x\in S$'' again, and you definitely should \emph{not} say ``let $x\in S$'' again.

\item \textbf{Use a symbol to mean only one thing.}
Once you use the letter $x$ once, its meaning is fixed for the duration of your proof. You cannot use $x$ to mean anything else. There is an exception to this guideline.  Sometimes a proof will include multiple subproofs that are distinct from each other.  In this case, you can reuse a variable or symbol as long as it is clear to the reader that you have concluded with the previous subproof and have moved onto a new subproof.   

\item \textbf{Do not ``prove by example.''}\label{pfbyexample}
Most problems ask you to prove that something is true ``for all''---You \emph{cannot} prove this by giving a single example, or even a hundred. Your proof will need to be a logical argument that holds for \emph{every example there possibly could be}.

On the other hand, if the claim that you are trying to prove involves the existence of a mathematical object with a particular property, then providing a specific example is sufficient.
        
\item \textbf{Write ``Let $x=\dots$,'' not ``Let $\dots=x$.''} 
When you have an existing expression, say $a^{2}$, and you want to give it a new, simpler name like $b$, you should write ``Let $b=a^{2}$,'' which means, ``Let the new symbol $b$ mean $a^{2}$.'' This convention makes it clear to the reader that $b$ is the brand-new symbol and $a^{2}$ is the old expression he/she already understands.
        
If you were to write it backwards, saying ``Let $a^{2}=b$,'' then your startled reader would ask, ``What if $a^{2}\neq b$?''
  
\item \textbf{Make your counterexamples concrete and specific.}
Proofs need to be entirely general, but counterexamples should be concrete. When you provide an example or counterexample, make it as specific as possible. For a set, for example, you must specify its elements, and for a function you must specify the corresponding relation (possibly an algebraic rule) and its domain and codomain. Do not say things like ``$f$ could be one-to-one but not onto''; instead, provide an actual function $f$ that is one-to-one but not onto.
    
\item \textbf{Do not include examples in proofs.}
Including an example very rarely adds anything to your proof. If your logic is sound, then it does not need an example to back it up. If your logic is bad, a dozen examples will not help it (see Guideline~\ref{pfbyexample}). There are only two valid reasons to include an example in a proof: if it is a \emph{counterexample} disproving something, or if you are performing complicated manipulations in a general setting and the example is just to help the reader understand what you are saying.

 \item \textbf{Use scratch paper.}
Finding your proof will be a long, potentially messy process,  full of false starts and dead ends. Do all that on scratch paper until you find a real proof, and only then break out your clean paper to write your final proof carefully.
        
Only sentences that actually contribute to your proof should be part of the proof. Do not just perform a ``brain dump,'' throwing everything you know onto the paper before showing the logical steps that prove the conclusion. \emph{That is what scratch paper is for.}

\end{enumerate}
\chapter{Fancy Mathematical Terms}
\label{appendix:fancy_math_terms}

Here are some important mathematical terms that you will encounter in this course and throughout your mathematical career.

\begin{enumerate}
\item
\textbf{Definition}---a precise and unambiguous description of the meaning of a mathematical term.  It characterizes the meaning of a word by giving all the properties and only those properties that must be true.
\item
\textbf{Theorem}---a mathematical statement that is proved using rigorous mathematical reasoning.  In a mathematical paper, the term theorem is often reserved for the most important results.
\item
\textbf{Lemma}---a minor result whose sole purpose is to help in proving a theorem.  It is a stepping stone on the path to proving a theorem. Very occasionally lemmas can take on a life of their own (Zorn's Lemma, Urysohn's Lemma, Burnside's Lemma, Sperner's Lemma).
\item
\textbf{Corollary}---a result in which the (usually short) proof relies heavily on a given theorem (we often say that ``this is a corollary of Theorem A'').
\item
\textbf{Proposition}---a proved and often interesting result, but generally less important than a theorem.
\item
\textbf{Conjecture}---a statement that is unproved, but is believed to be true (Collatz Conjecture, Goldbach Conjecture, Twin prime Conjecture).
\item
\textbf{Claim}---an assertion that is then proved.  It is often used like an informal lemma.
\item
\textbf{Axiom/Postulate}---a statement that is assumed to be true without proof. These are the basic building blocks from which all theorems are proved (Euclid's five postulates, axioms of ZFC, Peano axioms).
\item
\textbf{Identity}---a mathematical expression giving the equality of two (often variable) quantities (trigonometric identities, Euler's identity).
\item
\textbf{Paradox}---a statement that can be shown, using a given set of axioms and definitions, to be both true and false. Paradoxes are often used to show the inconsistencies in a flawed axiomatic theory (e.g., Russell's Paradox).  The term paradox is also used informally to describe a surprising or counterintuitive result that follows from a given set of rules (Banach-Tarski Paradox, Alabama Paradox, Gabriel's Horn).
\end{enumerate}
\chapter{Definitions in Mathematics}
\label{appendix:definitions}
\thispagestyle{empty}

It is difficult to overstate the importance of definitions in mathematics. Definitions play a different role in mathematics than they do in everyday life. 

Suppose you give your friend a piece of paper containing the definition of the rarely-used word \textbf{rodomontade}. According to the Oxford English Dictionary\footnote{http://www.oed.com/view/Entry/166837} (OED) it is:
\begin{quote}
A vainglorious brag or boast; an extravagantly boastful, arrogant, or bombastic speech or piece of writing; an arrogant act.
\end{quote}
Give your friend some time to study the definition. Then take away the paper. Ten minutes later ask her to define rodomontade. Most likely she will be able to give a reasonably accurate definition. Maybe she'd say something like, ``It is a speech or act or piece of writing created by a pompous or egotistical person who wants to show off how great they are.'' It is unlikely that she will have quoted the OED word-for-word. In everyday English that is fine---you would probably agree that your friend knows the meaning of the rodomontade. This is because most definitions are \emph{descriptive}. They describe the common usage of a word. 

Let us take a mathematical example. The OED\footnote{http://www.oed.com/view/Entry/40280}  gives this definition of \emph{continuous}.
\begin{quote}
Characterized by continuity; extending in space without interruption of substance; having no interstices or breaks; having its parts in immediate connection; connected, unbroken.
\end{quote}
Likewise, we often hear calculus students speak of a continuous function as one whose graph can be drawn ``without picking up the pencil.'' This definition is descriptive. (As we learned in calculus the picking-up-the-pencil description is not a perfect description of continuous functions.) This is not a mathematical definition. 

Mathematical definitions are \emph{prescriptive}. The definition must prescribe the exact and correct meaning  of a word. Contrast the OED's descriptive definition of continuous with the the definition of continuous found in a real analysis textbook.
\begin{quote}
A function $f:A\to \mathbb{R}$ is \textbf{continuous at a point} $c\in A$ if,  for all $\varepsilon>0$, there exists $\delta>0$ such that whenever $|x-c|<\delta$ (and $x\in A$) it follows that $|f(x)-f(c)|<\varepsilon$. If $f$ is continuous at every point in the domain $A$, then we say that $f$ is \textbf{continuous on} $A$.\footnote{This definition is taken from page 109 of Stephen Abbott's \emph{Understanding Analysis}, but the definition would be essentially the same in any modern real analysis textbook.} 
\end{quote}
In mathematics there is very little freedom in definitions. Mathematics is a deductive theory; it is impossible to state and prove theorems without clear definitions of the mathematical terms. The definition of a term must completely, accurately, and unambiguously describe the term. Each word is chosen very carefully and the order of the words is  critical. In the definition of continuity changing ``there exists'' to ``for all,'' changing the orders of quantifiers, changing $<$ to $\leq$ or $>$, or changing $\mathbb{R}$ to $\mathbb{Z}$ would completely change the meaning of the definition. 

What does this mean for you, the student? Our recommendation is that at this stage you memorize the definitions word-for-word. It is the safest way to guarantee that you have it correct. As you gain confidence and familiarity with the subject you may be ready to modify the wording. You may want to change ``for all'' to ``given any'' or you may want to change $|x-c|<\delta$ to $-\delta<x-c<\delta$ or to ``the distance between $x$ and $c$ is less than $\delta$.'' 

Of course, memorization is not enough; you must have a conceptual understanding of the term, you must see how the formal definition matches up with your conceptual understanding, and you must know how to work with the definition. It is perhaps with the first of these that descriptive definitions are useful. They are useful for building intuition and for painting the ``big picture.'' Only after days (weeks, months, years?) of experience does one get an intuitive feel for the $\varepsilon,\delta$-definition of continuity; most mathematicians have the ``picking-up-the-pencil'' definitions in their head. This is fine as long as we know that it is imperfect, and that when we prove theorems about continuous functions in mathematics we use the mathematical definition. 

We end this discussion with an amusing real-life example in which a descriptive definition was not sufficient. In 2003 the German version of the game show \emph{Who wants to be a millionaire?} contained the following question: ``Every rectangle is: (a) a rhombus, (b) a trapezoid, (c) a square, (d) a parallelogram.'' 

The confused contestant decided to skip the question and left with \euro 4000. Afterward the show received letters from irate viewers. Why were the contestant and the viewers upset with this problem? Clearly a rectangle is a parallelogram, so (d) is the answer. But what about (b)? Is a rectangle a trapezoid? We would describe a trapezoid as a quadrilateral with a pair of parallel sides. But this leaves open the question: can a trapezoid have \emph{two} pairs of parallel sides or must there only be \emph{one} pair? The viewers said two pairs is allowed, the producers of the television show said it is not. This is a case in which a clear, precise, mathematical definition is required.

\end{document}