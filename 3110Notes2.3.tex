\documentclass[11pt]{article}

\usepackage{amsfonts}
\usepackage{amsmath}
\usepackage{amssymb}
\usepackage{stmaryrd}
\usepackage{amsthm}
\usepackage{fancyhdr}
\usepackage[margin=1in]{geometry}
\usepackage[hang,flushmargin,symbol*]{footmisc}
\usepackage{color}
\definecolor{darkblue}{rgb}{0, 0, .6}
\definecolor{grey}{rgb}{.7, .7, .7}
\usepackage[breaklinks]{hyperref}
\hypersetup{
	colorlinks=true,
	linkcolor=darkblue,
	anchorcolor=darkblue,
	citecolor=darkblue,
	pagecolor=darkblue,
	urlcolor=darkblue,
	pdftitle={},
	pdfauthor={}
}

\pagestyle{fancy}

\lhead{\scriptsize Course Notes for Logic, Proof, \& Axiomatic Systems (Spring 2011)} 
\rhead{\scriptsize Instructor: \href{http://oz.plymouth.edu/~dcernst}{D.C. Ernst}} 
\lfoot{\scriptsize This work is an adaptation of notes written by Stan Yoshinobu of Cal Poly and Matthew Jones of California State University, Dominguez Hills.} 
\cfoot{} 
\renewcommand{\headrulewidth}{0.4pt} 
\renewcommand{\footrulewidth}{0.4pt} 

\theoremstyle{definition}
\newtheorem{theorem}{Theorem}[section]
\newtheorem{acknowledgement}[theorem]{Acknowledgement}
\newtheorem{algorithm}[theorem]{Algorithm}
\newtheorem{axiom}[theorem]{Axiom}
\newtheorem{case}[theorem]{Case}
\newtheorem{claim}[theorem]{Claim}
\newtheorem{conclusion}[theorem]{Conclusion}
\newtheorem{condition}[theorem]{Condition}
\newtheorem{conjecture}[theorem]{Conjecture}
\newtheorem{corollary}[theorem]{Corollary}
\newtheorem{criterion}[theorem]{Criterion}
\newtheorem{definition}[theorem]{Definition}
\newtheorem{example}[theorem]{Example}
\newtheorem{exercise}[theorem]{Exercise}
\newtheorem{journal}[theorem]{Journal}
\newtheorem{lemma}[theorem]{Lemma}
\newtheorem{notation}[theorem]{Notation}
\newtheorem{problem}[theorem]{Problem}
\newtheorem{proposition}[theorem]{Proposition}
\newtheorem{remark}[theorem]{Remark}
\newtheorem{solution}[theorem]{Solution}
\newtheorem{summary}[theorem]{Summary}
\newtheorem{question}[theorem]{Question}

\begin{document}

\addtocounter{section}{1}

\begin{section}{Set Theory and Topology}

\addtocounter{subsection}{2}
\addtocounter{theorem}{43}

\begin{subsection}{Indexing Sets}

Suppose we wish to consider the following collection of open intervals:
\[
(0,1), (0,1/2), (0,1/4), \ldots, (0,1/2^{n-1}), \ldots
\]
This collection has a natural way for us to ``index" the sets:
\[
I_1=(0,1), I_2=(0,1/2), \ldots, I_n=(0,1/2^{n-1}), \ldots
\]
In this case the sets are \textbf{indexed} by the set $\mathbb{N}$.  The subscripts on the capital letters are taken from the \textbf{index set}.  If we wanted to talk about an arbitrary set from this indexed collection, we could use the notation $I_n$.

Let's consider another example:
\[
\{a\}, \{a,b\}, \{a,b,c\}, \ldots, \{a,b,c,\ldots,z\}
\]
An obvious way to index these sets is as follows:
\[
A_1=\{a\}, A_2=\{a,b\}, A_3=\{a,b,c\}, \ldots, A_{26}=\{a,b,c,\ldots,z\}
\]
In this case, the collection of sets is indexed by $\{1,2,\ldots, 26\}$.

\begin{remark}
Using indexing sets in mathematics is an extremely useful notational tool, but it is important to keep straight the difference between the sets that are being indexed, the elements in each set being indexed, the indexing set, and the elements of the indexing set.
\end{remark}

Any set (finite or infinite) can be used as an indexing set.  Often capital Greek letters are used to denote arbitrary indexing sets and small Greek letters to represent elements of these sets.  For example, we might use $\Delta$ (capital delta) to refer to an indexing set and write $\alpha \in \Delta$ for an individual index.  Typically, if the indexing set is some subset of $\mathbb{Z}$ (like $\mathbb{N}$), then we would use letters like $k,m,n,l$ for an individual index.  Likewise, if the indexing set is $\mathbb{R}$, then we might use $s,t,x,y$ as indices.  

\begin{example}
Here are some examples of common notation that you will encounter.
\begin{enumerate}
\item If $\Delta$ is a set and we have a collection of sets indexed by $\Delta$, then we may write
\[
\{S_{\alpha}\}_{\alpha\in \Delta}
\]
to refer to this collection.  We read this as ``the set of $S$-alphas over alpha in Delta."
\item If a collection of sets is indexed by the natural numbers, then we may write
\[
\{U_n\}_{n\in\mathbb{N}}
\]
or
\[
\{U_n\}_{n=1}^{\infty}.
\]
\item Borrowing from this idea, we can write the collection $\{A_1,\ldots,A_{26}\}$ from the beginning of the section as
\[
\{A_n\}_{n=1}^{26}.
\]

\end{enumerate}
\end{example}

\begin{definition}
Suppose we have a collection $\{A_{\alpha}\}_{\alpha\in\Delta}$.

\begin{enumerate}
\item The \textbf{union of the entire collection} is defined via
\[
\bigcup_{\alpha\in\Delta} A_{\alpha}=\{x:x\in A_{\alpha} \mbox{ for some }\alpha\in \Delta\}.
\]

\item The \textbf{intersection of the entire collection} is defined via
\[
\bigcap_{\alpha\in\Delta} A_{\alpha}=\{x:x\in A_{\alpha} \mbox{ for all }\alpha\in \Delta\}.
\]

\end{enumerate}
\end{definition}

\begin{remark} 
In the special case that $\Delta=\mathbb{N}$, we write
\[
\bigcup_{n=1}^{\infty}A_n= \{ x : x \in A_n \mbox{ for some } n \in \mathbb{N}\}
\] 
and
\[
\bigcap_{n=1}^{\infty}A_n= \{ x : x \in A_n \mbox{ for all } n \in \mathbb{N}\}.
\] 
\end{remark}

\begin{exercise}
Let $\{I_n\}_{n\in\mathbb{N}\cup\{0\}}$ be the collection of open intervals from the beginning of the section.  Find each of the following.
\begin{enumerate}
\item $\displaystyle \bigcup_{n\in\mathbb{N}\cup\{0\}}I_n$
\item $\displaystyle \bigcap_{n\in\mathbb{N}\cup\{0\}}I_n$
\end{enumerate}
\end{exercise}

\begin{exercise}
Repeat the previous exercise, but assume that the sets are closed intervals.
\end{exercise}

\begin{exercise}
Let $\{A_n\}_{n=1}^{26}$ be the collection from earlier in the section.  Find each of the following.
\begin{enumerate}
\item $\displaystyle \bigcup_{n=1}^{26}A_n$
\item $\displaystyle \bigcap_{n=1}^{26}A_n$
\end{enumerate}
\end{exercise}

\begin{exercise}
Let $S_n = \{x \in \mathbb{R} \ : \ n-1<x<n \}$.  Find each of the following.
\begin{enumerate}
\item $\displaystyle \bigcup_{n=1}^{\infty}S_n$

\item $\displaystyle \bigcap_{n=1}^{\infty}S_n$
\end{enumerate}
\end{exercise}

\begin{exercise}
Let $T_n = \{x \in \mathbb{R} \ : \ -\frac{1}{n}<x< \frac{1}{n} \}$.  Find each of the following.
\begin{enumerate}
\item $\displaystyle \bigcup_{n=1}^{\infty}T_n$

\item $\displaystyle \bigcap_{n=1}^{\infty}T_n$
\end{enumerate}

\end{exercise}

\begin{exercise}
For each $r\in\mathbb{Q}$ (the rational numbers), let $N_r$ be the set containing all real numbers \emph{except} $r$.  Find each of the following.
\begin{enumerate}
\item $\displaystyle \bigcup_{r\in\mathbb{Q}}N_r$

\item $\displaystyle \bigcap_{r\in\mathbb{Q}}N_r$
\end{enumerate}

\end{exercise}

\begin{definition}
We say that a collection of sets $\{A_{\alpha}\}_{\alpha\in\Delta}$ is \textbf{pairwise disjoint} if $A_{\alpha} \cap A_{\beta}=\emptyset$ whenever $\alpha\neq \beta$.
\end{definition}

\begin{exercise}
Draw a Venn diagram of a collection of 3 sets that are pairwise disjoint.
\end{exercise}

\begin{exercise}
Provide an example of a collection of three sets, say $\{A_1, A_2, A_3\}$, such that the collection is \emph{not} pairwise disjoint, but 
\[
\bigcap_{n=1}^3 A_n=\emptyset.
\]
\end{exercise}

\begin{theorem}[*, Generalized Distribution of Union and Intersection]
Suppose we have a collection $\{A_{\alpha}\}_{\alpha\in\Delta}$.  Let $B$ be any set.  Then
\begin{enumerate}
\item $\displaystyle B \cup \left(\bigcap_{\alpha\in\Delta}A_{\alpha}\right)=\bigcap_{\alpha\in\Delta}(B\cup A_{\alpha})$,
\item $\displaystyle B \cap \left(\bigcup_{\alpha\in\Delta}A_{\alpha}\right)=\bigcup_{\alpha\in\Delta}(B\cap A_{\alpha})$.
\end{enumerate}
(Prove the one not done in class.)
%(You only need to prove one of these; the other is similar.)
\end{theorem}

\begin{theorem}[*, Generalized DeMorgan's Law]
Suppose we have a collection $\{A_{\alpha}\}_{\alpha\in\Delta}$.  Then
\begin{enumerate}
\item $\displaystyle \left(\bigcup_{\alpha\in\Delta} A_{\alpha}\right)^C=\bigcap_{\alpha\in\Delta}A_{\alpha}^{C}$,
\item $\displaystyle \left(\bigcap_{\alpha\in\Delta} A_{\alpha}\right)^C=\bigcup_{\alpha\in\Delta}A_{\alpha}^{C}$.
\end{enumerate}
(You only need to prove one of these; the other is similar.)
\end{theorem}

\end{subsection}

\end{section}

\end{document}