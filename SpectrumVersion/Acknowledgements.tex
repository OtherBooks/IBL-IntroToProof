\chapter*{Acknowledgements}
\addcontentsline{toc}{chapter}{\protect\numberline{}Acknowledgements}

The first draft of this book was written in 2009. At that time, several of the sections were adaptations of course materials written by Matthew Jones (CSU Dominguez Hills) and Stan Yoshinobu (University of Toronto). The current version of the book is the result of many iterations that involved the addition of new material, retooling of existing sections, and feedback from instructors that have used the book. The current version of the book is a far cry from what it looked like in 2009.

This book has been an open-source project since day one. Instructors and students can download the PDF for free and modify the source as they see fit. Several instructors and students have provided extremely useful feedback, which has improved the book at each iteration. Moreover, due to the open-source nature of the book, I have been able to incorporate content written by others. Below is a partial list of people (alphabetical by last name) that have contributed content, advice, or feedback.

\begin{itemize}
\item \href{https://math.depaul.edu/cdrupies/}{Chris Drupieski}, \href{http://math.depaul.edu/tpeter21/}{T.~Kyle Petersen}, and \href{http://math.depaul.edu/bridget/}{Bridget Tenner} (DePaul University). Modifications that these three made to the book inspired me to streamline some of the exposition, especially in the early chapters.
\item \href{http://www.paulellis.org}{Paul Ellis} (Manhattanville College). Paul has provided lots of useful feedback and several suggestions for improvements. Paul suggested problems for Chapter~\ref{chap:Induction} and provided an initial draft of Section~\ref{sec:Images and Preimages}: Images and Preimages of Functions.
\item \href{http://jasongrout.org}{Jason Grout} (Bloomberg, L.P.).  I am extremely grateful to Jason for feedback on early versions of this manuscript, as well as for helping me with a variety of technical aspects of writing an open-source textbook.
\item \href{https://www.linkedin.com/in/andershendrickson/}{Anders Hendrickson} (Milliman). Anders is the original author of the content in Appendix~\ref{appendix:elements_of_style}: Elements of Style for Proofs. The current version in Appendix~\ref{appendix:elements_of_style} is a result of modifications made by myself with some suggestions from David Richeson.
\item \href{http://www.hsc.edu/rebecca-jayne}{Rebecca Jayne} (Hampden--Sydney College). The current version of Section~\ref{sec:CompleteInduction}: Complete Induction is a derivative of content originally contributed by Rebecca.
\item \href{http://www4.csudh.edu/library/info/civic-directory/f-j/matthew-g-jones}{Matthew Jones} (CSU Dominguez Hills) and \href{http://www.stanyoshinobu.com}{Stan Yoshinobu} (University of Toronto). A few of the sections were originally adaptations of notes written by Matt and Stan. Early versions of this textbook relied heavily on their work. Moreover, Matt and Stan were two of the key players that contributed to shaping my approach to teaching.
\item \href{http://users.dickinson.edu/~richesod/}{David Richeson} (Dickinson College). David is responsible for much of the content in Appendix~\ref{appendix:fancy_math_terms}: Fancy Mathematical Terms, Appendix~\ref{appendix:paradoxes}: Paradoxes, and Appendix~\ref{appendix:definitions}: Definitions in Mathematics. In addition, the current version of Chapter~\ref{chap:ThreeFamousTheorems}: Three Famous Theorems is heavily based on content contributed by David.
\item \href{http://www2.kenyon.edu/Depts/Math/schumacherc/public_html/}{Carol Schumacher} (Kenyon College). When I was transitioning to an IBL approach to teaching, Carol was one of my mentors and played a significant role in my development as a teacher.  Moreover, this work is undoubtably influenced my Carol's excellent book \emph{Chapter Zero: Fundamental Notions of Advanced Mathematics}, which I used when teaching my very first IBL course.
\item \href{http://webpages.csus.edu/wiscons/}{Josh Wiscons} (CSU Sacramento). The current version of Section~\ref{sec:ModularArithmetic}: Modular Arithmetic is a derivative of content contributed by Josh.
\end{itemize}
