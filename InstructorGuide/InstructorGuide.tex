\documentclass[11pt]{article}%{amsart}

\usepackage[margin=1in]{geometry}
\usepackage{amsmath}
\usepackage{amssymb}
\usepackage{amsthm}
\usepackage{url}
\usepackage[breaklinks]{hyperref}
\usepackage{color}
\hypersetup{
	colorlinks=true,
	linkcolor=darkblue,
	anchorcolor=darkblue,
	citecolor=darkblue,
	pagecolor=darkblue,
	urlcolor=darkblue,
	pdftitle={},
	pdfauthor={},
    bookmarksnumbered
}
\definecolor{darkblue}{rgb}{0, 0, .6}
\usepackage{tikz}
\usepackage[framemethod=TikZ]{mdframed}

\mdfdefinestyle{skeleton}{roundcorner=5pt,leftmargin=5pt,rightmargin=5pt,backgroundcolor = black!5!white}

\setlength{\parindent}{0pt}
\setlength{\fboxsep}{10pt}

\newcommand{\blankline}{\pagebreak[2]\vspace{.5\baselineskip}}

\DeclareMathOperator{\rel}{rel}
\DeclareMathOperator{\Rel}{Rel}

%----------------%
\begin{document}
%----------------%

\title{An Introduction to Proof via Inquiry-Based Learning\\
Instructor Guide}
\author{Dana C.~Ernst}
\date{\today}

\maketitle

\begin{mdframed}[style=skeleton]
This Instructor Guide is a work in progress!  If you have questions about something not addressed in this guide, you can either submit an issue on \href{https://github.com/dcernst/IBL-IntroToProof/issues}{GitHub} or send me an email at \url{dana.ernst@nau.edu}.  I want instructors that utilize \emph{An Introduction to Proof via Inquiry-Based Learning} to be as successful as possible, so please reach out with questions or concerns.
\end{mdframed}

%----------------%

\section*{Overview}

\emph{An Introduction to Proof via Inquiry-Based Learning} is intended to be used for a one-semester or quarter introduction to proof course (sometimes referred to as a transition to proof course). The purpose of the book is to introduce the reader to the process of constructing and writing formal and rigorous mathematical proofs. The intended audience is mathematics majors and minors. However, the book is also appropriate for anyone curious about mathematics and writing proofs. Most users of the book will have taken at least one semester of calculus, although other than some familiarity with a few standard functions in Chapter~8: Functions, content knowledge of calculus is not required. 

\blankline

In order to promote a more active participation in student learning, \emph{An Introduction to Proof via Inquiry-Based Learning} adheres to an educational philosophy called inquiry-based learning (IBL). IBL is a student-centered method of teaching that engages students in sense-making activities and challenges them to create or discover mathematics.  The book expects readers to actively engage with the topics at hand and to construct their own understanding.  The reader will be given tasks requiring them to solve problems, conjecture, experiment, explore, create, and communicate.  Rather than showing facts or a clear, smooth path to a solution, the book guides and mentors the reader through an adventure in mathematical discovery. 

\blankline

However, the book makes no assumptions about the specifics of how the instructor will choose to implement an IBL approach. Ultimately, the instructor should do what is best for their students. Generally speaking, students are told which problems and theorems to grapple with for the next class sessions, and then the majority of class time is devoted to students working in groups on unresolved solutions/proofs or having students present their proposed solutions/proofs to the rest of the class. Students should---as much as possible---be responsible for guiding the acquisition of knowledge and validating the ideas presented. That is, studens should not be looking to the instructor as the sole authority. In an IBL course, instructor and students have joint responsibility for the depth and progress of the course. While effective IBL courses come in a variety of forms, they all possess a few essential ingredients. According to \href{https://www.colorado.edu/eer/sites/default/files/attached-files/laursenrasmussencommentaryauthorversion0219.pdf}{Laursen and Rasmussen (2019)}, the Four Pillars of IBL are:
\begin{itemize}
\item Students engage deeply with coherent and meaningful mathematical tasks.
\item Students collaboratively process mathematical ideas.
\item Instructors inquire into student thinking.
\item Instructors foster equity in their design and facilitation choices.
\end{itemize}
The book can only address the first pillar while it is the responsibility of the instructor and class to develop a culture that provides an adequate environment for the remaining pillars to take root. Again, I would like to emphasize that the book is mostly agnostic about the approach that an instructor would take when teaching out of the book.  Heck, I don't see any reason why an instructor couldn't use the book for a lecture-based class. Although that's certainly not what I had in mind when writing the book.

\blankline

The book includes more content than one can expect to cover in a single semester or quarter. This allows the instructor/reader to pick and choose the sections that suit their needs and desires. Each chapter takes a focused approach to the included topics, but also includes many gentle exercises aimed at developing intuition.

\blankline

The following sections form the core of the book and are likely the sections that an instructor would focus on in a one-semester introduction to proof course.
\begin{itemize}
\item Chapter~2: Mathematics and Logic. All sections.
\item Chapter~3: Set Theory. Sections 3.1, 3.3, 3.4, and 3.5.
\item Chapter~4: Induction. All sections.
\item Chapter~7: Relations and Partitions. Sections 7.1, 7.2, and 7.3.
\item Chapter~8: Functions. Sections 8.1, 8.2, 8.3, and 8.4.
\item Chapter~9: Cardinality. All sections.
\end{itemize}
Time permitting, instructors can pick and choose topics from the remaining sections.  I typically cover the core sections listed above together with Chapter~6: Three Famous Theorems during a single semester. 

\blankline

There are many useful resources available that instructors can utilize for designing an effective IBL/active learning experience for their students.  The \href{http://www.inquirybasedlearning.org}{Academy of Inquiry Based Learning} is a good place to get started.  I also suggest consulting the MAA's \href{https://www.maa.org/programs-and-communities/curriculum%20resources/instructional-practices-guide}{Instructional Practices Guide}, which is a guide to evidence-based instructional practices in undergraduate mathematics.  One effective approach to getting started with IBL is mimicking another instructor's approach and then refining for your purposes over time.  Feel free to borrow as many ideas as you would like from how I set up the course I teach using \emph{An Introduction to Proof via Inquiry-Based Learning}.  You can find the syllabus, homework assignments, etc from two recent iterations of my course at the following links:
\begin{itemize}
\item \href{http://danaernst.com/teaching/mat320f21/}{Fall 2021}
\item \href{http://danaernst.com/teaching/mat320s20/}{Spring 2020}
\end{itemize}
My Fall 2021 course utilized a version of the book that is nearly identical to the current version, but I happened to cover far less material than usual.  I covered more ground in Spring 2020, but the version of the book I used that semester may look a bit different than the current version.  I would consider the amount of material I covered in Spring 2020 to be fairly typical.  The differences in the amount of material that I covered had nothing to do with the version of the book!  Feel free to reach out with questions about how to set up your course or how to best make use of the book.

%----------------%

\section*{Evidence in Favor of Active Learning}

If you are already using or considering using \emph{An Introduction to Proof via Inquiry-Based Learning}, you likely don't need to be convinced of the merits of active learning.  Nonetheless, below is very brief summary of some the data that supports the use of active learning.

\blankline

Evidence in favor of some form of active engagement of students is strong across STEM disciplines. \href{https://pubmed.ncbi.nlm.nih.gov/24821756/}{Freeman et al.~(2014)} conducted a meta-analysis of 225 studies of various forms of active learning, and found that students were 1.5 times more likely to fail in traditional courses as compared to active learning courses, and students in active learning courses outperformed students in traditional courses by 0.47 standard deviations on examinations and concept inventories. The following snippet from Freeman et al.~(2014) captures the importance of utilizing active learning across STEM education:
\begin{quote}
\emph{``The results raise questions about the continued use of traditional lecturing as a control in research studies, and support active learning as the preferred, empirically validated teaching practice in regular classrooms."}
\end{quote}

For IBL specifically, a research group from the University of Colorado Boulder led by Sandra Laursen conducted a comprehensive study of student outcomes in IBL undergraduate mathematics courses while linking these outcomes to students' and instructors' experiences of IBL (see \href{https://www.colorado.edu/eer/research-areas/student-centered-stem-education/inquiry-based-learning-college-mathematics}{Laursen et al.~2011; Laursen 2013; Kogan and Laursen 2014; Laursen et al.~2014}). This quasi-experimental, longitudinal study examined over 100 courses at four different campuses over a period that spanned two years.

\blankline

On average over 60\% of IBL class time was spent on student-centered activities including student-led presentations, discussion, and small-group work. In contrast, in non-IBL courses, 87\% of class time was devoted to students' listening to an instructor talk. In addition, the IBL sections were rated more highly for a supportive classroom environment and students conveyed that engaging in meaningful mathematical tasks while collaborating was essential to their learning. Below is a brief summary of some of the outcomes of Laursen et al.'s work.
\begin{itemize}
\item After an IBL or comparative course, IBL students reported higher learning gains than their non-IBL peers, across cognitive, affective, and collaborative domains of learning.
\item In later courses, students who had taken an IBL course earned grades as good or better than those of students who took non-IBL sections, despite having ``covered" less material.
\item Non-IBL courses show a marked gender gap: across the board, women reported lower learning gains and less supportive attitudes than did men (effect size 0.4--0.5). Women's confidence and sense of mastery of mathematics, and their interest in continued study of math were lower. This difference appears to be primarily affective, not due to real differences in women's mathematical preparation or achievement.
\item This gender gap was erased in IBL classes: women's learning gains were equal to men's, and their confidence and intent to persist similar. IBL approaches leveled the playing field for women, fixing a course that is problematic for women yet with no harm to men.
\end{itemize}

You can watch a short YouTube video of Sandra Laursen summarizing most of the recent research about inquiry-based learning \href{https://www.youtube.com/watch?v=m_HK6b3RGOc&feature=youtu.be}{here}. The \href{https://www.cbmsweb.org}{Conference Board of the Mathematical Sciences (CBMS)} wrote the following in their \href{https://www.cbmsweb.org/2016/07/active-learning-in-post-secondary-mathematics-education/}{position statement on active learning in 2016}:

\begin{quote}
\emph{``\ldots we call on institutions of higher education, mathematics departments and the mathematics faculty, public policy-makers, and funding agencies to invest time and resources to ensure that effective active learning is incorporated into post-secondary mathematics classrooms."}
\end{quote}

In addition, the Manifesto of the \href{https://www.maa.org/programs-and-communities/curriculum%20resources/instructional-practices-guide}{MAA Instructional Practices Guide} states: 

\begin{quote}
\emph{``We must gather the courage to advocate beyond our own classroom for student-centered instructional strategies that promote equitable access to mathematics for all students. We stand at a crossroads, and we must choose the path of transformation in order to fulfill our professional responsibility to our students."}
\end{quote}

%----------------%

\section*{Structure of the Textbook}

As students read the book, they should be digesting the material in a meaningful way.  In addition to reading and understanding new definitions and their related concepts, students will be asked to complete problems aimed at solidifying their understanding of the material.  In particular, the reader is asked to make conjectures, produce counterexamples, and prove theorems. All of these tasks will almost always be challenging.

\blankline

The items labeled as \textbf{Definition} and \textbf{Example} are meant to be read and digested.  However, the items labeled as \textbf{Problem}, \textbf{Theorem}, and \textbf{Corollary} require action on the reader's part.  Items labeled as \textbf{Problem} are sort of a mixed bag. Some Problems are computational in nature and aimed at improving understanding of a particular concept while others ask the reader to provide a counterexample for a statement if it is false or to provide a proof if the statement is true. Items with the \textbf{Theorem} and \textbf{Corollary} designation are mathematical facts and the intention is for the reader to produce a valid proof of the given statement. All of this is spelled out in the Introduction of the book, but I suggest taking the time to communicate this to your students.

\blankline

Oftentimes, the problems and theorems are guiding the reader towards a substantial, more  general result. Other times, they are designed to get the reader to apply ideas in a new way. Please take the time to tell your students that every task in the book is doable but sometimes very challenging.  They may not be successful on their first, or even second or third try.  This is okay!  Remind them of this often. 

\blankline

Discussion of new topics is typically kept at a minimum and there are very few examples in the book. This is intentional.  One of the objectives of the items labeled as \textbf{Problem} is for the reader to produce the examples needed to internalize unfamiliar concepts.  The overarching goal of the book is to help the reader develop a deep and meaningful understanding of the processes of producing mathematics by putting them in direct contact with mathematical phenomena.

%----------------%

\section*{General Comments}

blah

%----------------%

\section*{Chapter 1: Introduction}

I ask my students to read this short chapter and then take the time in class to highlight the content in \textbf{Section~1.4}.  Despite this, I always have at least one student that asks me what they are supposed to do the first time I assign a theorem.  At a minimum, I would ask students to read \textbf{Sections~1.2--1.4}.

%----------------%

\section*{Chapter 2: Mathematics and Logic}

%----------------%

Most instructors will likely choose to cover Chapter 2 as part of their course.  While this chapter sets the stage for what is to come later in the book, I caution you from getting too bogged down here.  I recommend covering this chapter as swiftly as your students can manage, so that you have time later for more challenging topics.  In my opinion, you can crank through this chapter without causing any damage. If necessary, you can always return to the foundation in this chapter as needed. None of the mathematical theorems (in contrast to the logical theorems, e.g., \textbf{Theorem~2.26}) are needed in future chapters.  The sole purpose of the mathematical theorems in Chapter 2 is to provide an onramp to proving theorems in a familiar context.

\blankline

\textbf{Section 2.1} is a bit unusual as it asks students to jump right into proving theorems without any initial guidance.  This is intentional, and in my experience is a very effective way to start the course.  First, it gives you a sense of where your students are at in terms of their general understanding, or lack thereof, mathematical proofs.  I think you will be surprised at what students are capable of without being explicitly told how to write a proof.  Second, and more importantly, this section develops an intellectual need for more rigor and more structure.  Take the time to communicate to your students what the purpose of this section is.  Emphasize to them that they should just jump in and try their hand at writing proofs without too much concern for whether they are ``doing it the right way." Lots of interesting issues will bubble to the surface and you will have to pick and choose which topics you want to carefully address now versus later in the semester.  \textbf{Theorem~2.2} is the first theorem in the book and you should remind students that the intended task for theorems is to prove them.

\blankline

The definitions for even and odd integers are given in \textbf{Definition~2.1}, but many students will attempt to use their favorite characterizations  that they are familiar with.  For example, they might say that an integer is even if it is divisible by 2.  While this is certainly true, they won't encounter the definition of ``divides" until \textbf{Definition~2.5}. It's a good idea to point out that this characterization of even follows immediately from \textbf{Definitions~2.1} and \textbf{2.5}. Students are also often surprised that negative integers can be odd or even.  

\blankline

You will have to decide how firm of a stance you want to take on phrases like ``even integer" versus ``even number."  While the former is certainly the correct choice of language, my view is that the adjectives ``even" and ``odd" only refer to integers, so there shouldn't be much confusion when one says, ``even number."  You should decide in advance how you want to approach this issue.  At the very least, be sure to have a conversation about this with your students.  

\blankline

To save some time, notice that I wrote, ``For the remainder of this section, you may assume that every integer is either even or odd but never both" in the paragraph below \textbf{Definition~2.1}.  This is something you should point out to students and be sure to tell them that officially this is something one would need to prove. We will be more careful in later chapters.

\blankline

\textbf{Problem~2.6} and the paragraph that follows address the difference between ``$n$ divides $m$" versus ``$m$ divided by $n$."  The former is a sentence while the latter is a noun (i.e., the number one obtains after dividing $m$ by $n$).  You should be prepared for this confusion to rear its head multiple times during your course.

\blankline

\textbf{Corollary~2.9} is the first encounter with a corollary in the book. It's a good idea to remind students what a corollary is (even though there is a reminder in the text right above \textbf{Corollary~2.9}).  One can prove \textbf{Corollary~2.9} by appealing to \textbf{Theorem~2.8} or directly.  I always find a way to make sure both approaches get presented (by either students or by me) and then we take a moment to discuss the fact that we can appeal to previous results or return to the definition and prove things from scratch.

\blankline

The same two approaches could be employed in the proof of \textbf{Theorem~2.10}.  In fact, perhaps I should have called Theorem~2.10 a corollary.

\blankline

\textbf{Definition~2.18(d)} introduces the definition of the truth value of a conditional statement.  It's important to emphasize to students that this is a definition.  Students are often disturbed by the fact that $A \implies B$ is true when $A$ is false no matter what the truth value of $B$ is.  The purpose of \textbf{Problem~2.23} is to help students reconcile any cognitive dissonance that students might have with this definition.

\blankline

While \textbf{Section~2.1} asks students to jump in head first without a formal understanding of what it was we were doing, \textbf{Section~2.2} hits the reset button and takes a more careful approach while introducing the basics of propositional logic.  One detail you could get lots in the weeds on is whether a given English-language sentence is a proposition or not.  My suggestion is to not lose focus here.  There are lots of subtle issues that I have chosen not to formally address in the book. 

\blankline

My suggestion is minimize the importance of the symbolic logic (e.g., strings utilizing symbols such as $\forall$, $\exists$, $\wedge$, $\vee$, $\implies$, etc.) presented in \textbf{Sections~2.2--2.5}.  I think is important for students to be exposed to symbolic logic, but I essentially abandon it after \textbf{Section~2.5}.  Discourage your students from using the formal symbolism in their mathematical proofs.

\blankline

The intended proof technique for most of the theorems in \textbf{Section~2.2} is create a truth table.  You'll likely need to provide some guidance to your students about how many intermediate columns they should include in their truth tables.  \textbf{Corollary~2.41} can be proved by either creating a truth table or by appealing to \textbf{Theorem~2.40}, \textbf{Problem~2.27}, and \textbf{Theorem~2.25} and then relying on the transitivity of logical equivalence. Note that I do not explicitly state transitivity of logical equivalence anywhere in the textbook.

\blankline

While you could skip \textbf{Theorem~2.56} and \textbf{Theorem~2.57}, these theorems provide a first opportunity to practice writing a proof by contradiction and a biconditional proof, respectively.

\blankline

One thing worth pointing out is that there is an implicit universal quantifier in front of all statements of the form ``if $A(x)$, then $B(x)$", where $A(x)$ and $B(x)$ are predicates and $x$ is some variable. I attempt to address this issue just after \textbf{Problem~2.67}. This is standard convention in mathematics and is definitely worth pointing out to your students. The statement ``For all $x$, if $A(x)$, then $B(x)$" is a proposition. This statement may sometimes be written in two sentences, where the first sentence bounds the variable. For example, ``Let $n$ be a natural number. If $A(n)$, then $B(n)$." Another way to write the same statement would be ``If $n$ is a natural number such that $A(n)$, then $B(n)$." This is how most of the theorems in Sections 2.1 and 2.3 are written. More commonly, if the universe of discourse is understood (or can be inferred), one might simply write ``If $A(n)$, then $B(n)$."  Officially, ``If $A(n)$, then $B(n)$" is a predicate, but if we take this as short hand for ``For all $n$, if $A(n)$, then $B(n)$", we have a proposition.  

\blankline

It's not crucial that you cover \textbf{Problems~2.70-2.73}, but I do find it helpful for students to practice the conversion from plain English to symbolic logic and vice versa.  The idea in \textbf{Problem~2.73} is probably a good one to discuss.

\blankline

Students struggle with proving \textbf{Theorem~2.76}, which handles the negation of quantifiers.  This is a good example where you must think very deeply, but there isn't much to write down.  This is a good example of a theorem where I would probably ask students to prove one of the two parts (leaving it to them to choose which part).

\blankline

\textbf{Problem~2.86} is time consuming.  One way you could speed things up is to tell them which parts are true versus false.  You could also skip some of the parts.  While \textbf{Theorems~2.89} and \textbf{2.90} are not crucial for later, they could be skipped, but I recommend doing at least one of them.  \textbf{Theorem~2.91} is also not crucial for future chapters, but provides an opportunity for proving uniqueness.  It's a long while before this issue crops up again (the next instance is \textbf{Theorem~4.36}), so you could defer the discussion about uniqueness until it is absolutely necessary. 

\blankline

One detail that gets swept under the rug in \textbf{Sections~2.4} and \textbf{2.5} is that we discuss the truth value of conditional propositions in \textbf{Definition~2.18} and then take for granted that we can apply the same structure to conditional statements involving predicates. For better or worse, this seems to be the standard approach in introduction to proof books.  My personal take is that I don't want to get bogged down in these sorts of pedantic details, but if you feel otherwise, you should address the issue carefully with your students.


\section*{Chapter 3: Set Theory}

Unless you know your students are familiar with the content in Chapter~3, I think covering this chapter is essential.  In my experience, once students get the hang of ``element chasing", this chapter goes smoothly.  You could skip \textbf{Section~3.2} about paradoxes or simply assign it as outside reading.  However, in my experience, students tend to be captivated by paradoxes.  Moreover, at least at my institution, there isn't another place in the curriculum that a student would be exposed to the idea of a paradox in mathematics.  If you have intentions of ever discussing the Axiom of Choice (\textbf{Axiom~3.43}), you should consider discussing \textbf{Section~3.2}.

\blankline

Taking the time to carefully outline a skeleton proof for proving $A\subseteq B$ in \textbf{Problem~3.9} will help tremendously as you progress through Chapter~3. 

\blankline

\textbf{Theorem~3.20} provides an excellent opportunity to showcase two different proof techniques.  One method is to prove the two set containments by ``element chasing".  The second method involves a chain of equality of sets.  I recommend showcasing both approaches.  For example, here is a one sentence proof of \textbf{Theorem~3.20}.

\blankline

\emph{Proof of Theorem~3.20.} Let $A$ and $B$ be sets in some universe of discourse $U$. We see that
\begin{align*}
A\setminus B & = \{x\in U\mid x\in A \text{ and } x\notin B\}\\
& = \{x\in U\mid x\in A \text{ and } x\in B^c\}\\
& = \{x\in U\mid x\in A \cap B^c\}\\
& = A\cap B^c. 
\end{align*}
\vspace{-2em}
 
\qed

\blankline

For \textbf{Theorems~3.21} and \textbf{3.22}, I typically ask students to prove one of Parts~(a) or~(b).

\blankline

\textbf{Problem~3.23} is not necessary for later, but I find it to be a good exercise for students.

\blankline

\textbf{Problem~3.31} is time consuming, but extremely worthwhile.  One shortcut you could take is telling the students which parts are true versus false.

\blankline

In my experience, \textbf{Section~3.3} is probably the most challenging section in Chapter~3.  For this reason, I wouldn't skip any of the easy exercises (e.g., \textbf{Problems~3.33--3.3.37}).  

\blankline

For \textbf{Theorems~3.41} and \textbf{3.42}, I typically ask students to prove one of Parts~(a) or~(b).  For both theorems, I encourage students to prove both set containments as opposed to attempting to chain together equality of sets.

\blankline

If you have planes to cover \textbf{Chapters~7} or \textbf{8}, you will need to cover at least parts of \textbf{Section~3.5}.  One can safely skip \textbf{Theorem~3.56--Problem~3.62}, but I enjoy covering this material if time allows. \textbf{Problem~3.61} is time consuming and provides a nice challenge for students.  \textbf{Problem~3.62} is also a fun challenge, but certainly isn't necessary.
%----------------%

\section*{Chapter 4: Induction}

Most instructors will likely cover Chapter~4.  If your students already have some familiarity with induction, you can probably find places to cut corners.  However, in my experience, I find it best to assume students know nothing about induction and give it a careful treatment.  

\blankline

After students tackle \textbf{Theorem~4.2} (Principle of Mathematical Induction), I spend time discussing \textbf{Skeleton Proof~4.3} and then often present the first few induction proofs in order to set the standard for what the proofs should look like. One potentially misleading aspect of \textbf{Skeleton Proof~4.3}, is that students might think that are supposed to actually write ``$P(k)$" and ``$P(k+1)$" in their proofs.  I discourage this.  Instead, I instruct students to write down whatever the actual predicate is for $P(k)$.  For example, for the inductive step of the proof of \textbf{Theorem~4.4}, I would want students to write, ``Let $k\in\mathbb{N}$ and assume $\sum_{i=1}^k i=\frac{k(k+1)}{2}$ is true." In fact, I think we can safely omit the phrase ``is true" and simply write, ``Let $k\in\mathbb{N}$ and assume $\sum_{i=1}^k i=\frac{k(k+1)}{2}$."  If I present a few induction proofs, we can iron out these details quickly.

\blankline

\textbf{Theorem~4.7} and \textbf{Problem~4.8} provide an opportunity for students to wrestle with induction proofs that are simply manipulation of strings of symbols.  I highly recommend not skipping these two. \textbf{Problem~4.8} provided inspiration for the cover of the print version of the book.

\blankline

While I make sure we witness a precise proof of \textbf{Theorem~4.2}, sometimes I might wave my hands a bit and skimp on a rigorous proof of \textbf{Theorem~4.9}, which states that we do not need to start induction at $k=1$.  The proof of \textbf{Theorem~4.9} is simply a clever adjustment to the proof of \textbf{Theorem~4.2}.  

\blankline

I view \textbf{Theorem~4.11} as essential, but you definitely do not need to assign all of \textbf{Theorem~4.12--4.23} as none of these are needed for later.  I typically ask students to proof any two of \textbf{Theorem~4.12--4.23}.  \textbf{Problem~4.24} is not essential but provides an excellent opportunity for students to tinker and practice the skills they have developed.

\blankline

The proof of \textbf{Theorem~4.25} (Principle of Complete Mathematical Induction) often causes some confusion for students.  This provides a good opportunity to discuss the notion of ``weaker" versus ``stronger" hypotheses.  We know that The Principle of Mathematical Induction is true and this theorem has weaker hypotheses than its complete analogue, and hence the Principle of Complete Mathematical Induction must also be true.  Emphasize to students that despite the name, complete induction is not any stronger or more powerful than ordinary induction.  One can prove that the two notions are equivalent, but I've chosen not to formally ask students to prove this fact.

\blankline

I suggest you pick and choose a few of \textbf{Theorem~4.27--Problem~4.34} to have students to work on.  Note that \textbf{Problem~4.34} involves the Fibonacci numbers, which are introduced in \textbf{Problem~4.29}.

\blankline

The goal of \textbf{Section~4.5} is to prove \textbf{Theorem~4.38} (Well-Ordering Principle) and two generalizations that appear in \textbf{Theorems~4.39} and \textbf{4.40}. My intended proof of the Well-Ordering Principle is to utilize \textbf{Theorem~4.25} (Principle of Complete Mathematical Induction).  This is a nice application of induction.  If you plan to cover \textbf{Sections~5.1} or \textbf{6.1}, you will need the Well-Ordering Principle.  Specifically, the Well-Ordering Principle is needed in the proofs of \textbf{Theorems~6.6}, \textbf{6.7}, and \textbf{6.13}, and its generalization \textbf{Theorem~4.40} is needed in the proof of \textbf{Theorem~5.46}. Note that I have included the proof of \textbf{Theorem~6.7} (Division Algorithm) in Chapter~6. 

%----------------%

\section*{Chapter 5: The Real Numbers}

Chapter~5 takes a deep dive into the structure of the real numbers by building up from a collection of axioms. Most instructors will likely skip this chapter. None of the content in this chapter is necessary later in the book. However, if you are interested in students experiencing a careful and rigorous development of a mathematical topic, consider including \textbf{Section~2.1} in your course.  

\blankline

\textbf{Section~2.1} is essentially the content I cover at the beginning of the undergraduate real analysis course that I teach.  However, the material in this section is certainly appropriate for students in an introduction to proof course. Technically, you could cover the content in \textbf{Section~2.1} any time after Chapter~4.  The only reason Chapter~4 is necessary is because the generalization of the Well-Ordering Principle given in \textbf{Theorem~4.40} is needed in the proof of \textbf{Theorem~5.46}.   Otherwise, \textbf{Section~2.1} only requires the proof techniques developed in Chapter~2 and general understanding of sets that are covered in Chapter~3.

\blankline

\textbf{Section~2.2} develops the basics of the standard topology of the real numbers.  This section is a lot of fun!  However, it's not content that is traditionally covered in an introduction to proof course.  I have covered this material a few times in the past and in my experience students tend to enjoy it.  But it's not easy!  If you want to cover \textbf{Section~2.2}, you should also cover \textbf{Section~2.1}.  There is one theorem (namely, \textbf{Theorem~5.59}) in \textbf{Section~2.2} whose recommended proof technique is induction (covered in Chapter~4).  Otherwise, only the content in Chapters 2 and 3 are needed in \textbf{Section~2.2}.

\blankline

Be warned that covering this chapter, even just \textbf{Section~2.1}, takes some time.  Moreover, there aren't too many places you can cut corners.  You could ask students to only prove a subset of the results in \textbf{Theorem~5.8} (although all of the results in this theorem are needed for later).

%----------------%

\section*{Chapter 6: Three Famous Theorems}

As the title of the chapter implies, the goal of Chapter~6 is to tackle three main theorems with some intermediate results along the way.  In \textbf{Section~6.1}, we develop all of the concepts necessary to state and then prove the Fundamental Theorem of Arithmetic (\textbf{Theorem 6.17}). To prove the Fundamental Theorem of Arithmetic, my suggested proof makes use of the Division Algorithm (\textbf{Theorem 6.7}), which in turn utilizes the Well-Ordering Principle (\textbf{Theorem 4.38}) from Chapter~4. Note that I have included the proof of the Division Algorithm in the book. In \textbf{Section~2.2}, we prove that $\sqrt{2}$ is irrational, which settles a claim made in \textbf{Section~5.1}. In \textbf{Section~6.3}, the final section, we prove that there are infinitely many primes.  

\blankline

None of these sections depend on one another and you could pick and choose which ones they would like to cover.  Moreover, you could cover them in any order.  \textbf{Sections~2.2} and \textbf{2.3} are fairly straightforward for students, but in my experience, \textbf{Section~2.1} is much more challenging for students than you might expect.  If you choose to cover  \textbf{Section~2.1}, which I highly recommend, plan on it taking longer than the number of pages might suggest.  If I'm on schedule, I usually cover all three sections.  If I'm a little behind schedule, I might omit \textbf{Section~6.1} but cover the other two.

\blankline

I've already given a substantial hint for \textbf{Theorem~6.6}, but students still struggle with this one.  Feel free to give them additional guidance.  \textbf{Theorem 6.13} (Special Case of B\'ezout's Lemma) is another one that students find difficult.  My hint for this one is quite involved.  You might need to assist your students with digesting my hint.  \textbf{Theorem 6.15} (Euclid's Lemma) is the final crux in proving \textbf{Theorem~6.17} (Fundamental Theorem of Arithmetic).  Again, students struggle with this one despite my hint.  For \textbf{Theorem~6.17} (Fundamental Theorem of Arithmetic) you might need to remind students how to prove uniqueness (see \textbf{Skeleton Proof~2.90} and the paragraph that precedes it).

\blankline

My suggested proof for \textbf{Theorem~6.19} ($\sqrt{2}$ is irrational) generalizes nicely and leads to the proof of \textbf{Theorem~6.20}.

\blankline

Note that I've included the proof of \textbf{Theorem~6.23}. It's worth pointing out to students that the proof might be more involved than they would expect, given how trivial the theorem seems.

\blankline

\textbf{Problem~6.26} is a fun challenge for students, but definitely optional. 

%----------------%

\section*{Chapter 7: Relations and Partitions}

All the work we have done up to this point has laid the foundation for what will come in the remainder of the book. In my view, \textbf{Sections~7.1--7.3} mark a transition in the conceptual level of the material presented in this book.  

\blankline

\textbf{Section~7.1} introduces relations together with all the accompanying notation and terminology. This section contains more examples than any other in the book.  In addition, there are lots of problems aimed at developing a deeper intuitive understanding of relations.  You could skip some of the easier problems, but be warned that some of them are referenced later (e.g., \textbf{Problems~7.22} and \textbf{7.34}).  It is worth explicitly pointing out the differences in notation for $\rel(a)$ (lowercase ``r'') versus $\Rel(R)$ (uppercase ``R''), as well as the differences in the ``input" for each.

\blankline

It's up to you how much you want to focus on using digraphs/graphs as visual representations of relations, but I find this to be a useful tool for students. In particular, I think the discussions that come out of \textbf{Problem~7.30} are extremely valuable and provide insight into whether a student has an intuitive understanding of reflexive, symmetric, and transitive.  Moreover, this particular problem also provides insight into whether a student still has potential misconceptions concerning conditional statements.  For example, students will often say that if a relation is symmetric, then every pair of vertices in the digraph for the relation must have both potential arrows joining the vertices. A similar issue arises for transitivity. Be sure to address any potential misconceptions that could arise here. You can address these same issues when discussing \textbf{Skeleton Proofs~7.31--7.33}.

\blankline

You will need to plan in advance how much time you want to commit to \textbf{Problem~7.34}, which asks students to determine whether each of a long list of relations is reflexive, symmetric, or transitive, and then to either provide a specific counterexample or a proof.  If you cover every detail of every problem, it will be very time consuming.  I typically facilitate a full-class discussion where we march through each relation and quickly summarize why it is or is not reflexive, symmetric, or transitive.  Instead of writing down formal proofs, we simply discuss the essence of what needs to be proved.  When assigning this problem, I explicitly tell my students to just write quick sketches of proofs and the gist of counterexamples.  However, you may prefer for your students to write full-blown formal proofs/counterexamples. 

\blankline

Be careful with \textbf{Problem~7.41} as the answer ``yes" since the empty relation on the empty set is vacuously an equivalence relation.  Students may not arrive at this conclusion themselves, but are typically okay with it after the record is set straight.  However, students do tend to resist the analogous concept for partitions, so spending some extra time here making sure everyone is on board can pay off later. 

 

The notions of reflexive, symmetric, and transitive are introduced in \textbf{Section~7.1}, but we formally introduce equivalence relations in \textbf{Section~7.2}. The main result of this section is \textbf{Theorem~7.43}, which leads to the definition of equivalence class.  

\blankline

\textbf{Theorems~7.42} and \textbf{7.43} provide excellent opportunities for students to review the skills and proof techniques they developed up to this point.  Proofs of both of these theorems are places where I think it is worth spending time addressing all of the finer details.  In both instances, be sure to take the time to reflect on where the properties of reflexive, symmetric, and transitive are being used.

\blankline

You could skip \textbf{Problems~7.49} and \textbf{7.50}, but I find them to be a good test of student understanding.  I have put these two problems on exams (sometimes simply as true/false questions) without ever having covered them explicitly.

\blankline

\textbf{Section~7.3} introduces partitions.  The key results in this section are \textbf{Theorems~7.59} and \textbf{7.73}, which explicitly describe the relationship between equivalence relations and partitions.  I often remind students that in light of \textbf{Definition~7.51}, their proof of \textbf{Theorem~7.59} needs to address three items. Following the proof of \textbf{Theorem~7.59}, I suggest taking a moment to reflect back on \textbf{Problem~7.47}. Hopefully students will recognize that most of the heavy lifting for the proof of \textbf{Theorem~7.73} was already done when they proved \textbf{Theorems~7.68}, \textbf{7.70}, and \textbf{7.71}.

\blankline

Similar to \textbf{Problem~7.41}, be careful with \textbf{Problem~7.58}. Indeed, the empty set (not $\{\emptyset\}$) is a partition of the empty set at the three conditions of \textbf{Definition~7.51} vacuously hold in this case.  Students are typically uncomfortable with this. Thinking of the empty set as a collection of subsets of the empty set may be a challenging idea for students.  The issues that arise in \textbf{Problems~7.41} and \textbf{7.58} do not play a crucial role later, but they do provide a good opportunity for careful thinking and reflecting deeply on definitions.  You could safely decide to ignore these subtle issues if you so choose.  On the other hand, taking time to reflect on what impact requiring the set $A$ to be nonempty in \textbf{Definitions~7.17} (relation on a set $A$), \textbf{7.35} (equivalence relation on a set $A$), and \textbf{7.51} (partition of a set $A$) would have could be a useful exercise.  I have made the typical choice when it comes to defining relations and partitions on a set by allowing the set $A$ to be empty, but there are some sources that make the opposite choice.

\blankline

Be warned that the word ``nonempty" appeared in the statements of \textbf{Theorems~7.59}, \textbf{7.68}, and \textbf{7.74} in the original print version of the textbook.  However, all three theorems hold true even if $A$ is the empty set.  In light of the removal of the word ``nonempty" from the statement of these theorems, \textbf{Problems~7.60}, \textbf{7.69}, and \textbf{7.72} have been retooled and differ from the print version of the book.

\blankline

I typically omit \textbf{Section~7.4}, which takes a deep dive on modular arithmetic.  However, some instructors will make this section a priority.  Be warned that if you decide to cover this section, it will take a significant amount of time.  If you want to cover some but not all of this section, consider stopping at \textbf{Theorem~7.88} as the rest of the section is often covered in an abstract algebra (or possibly number theory) course.

%----------------%

\section*{Chapter 8: Functions}

%----------------%

\section*{Chapter 9: Cardinality}

\end{document}