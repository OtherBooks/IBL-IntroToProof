\chapter{Fancy Mathematical Terms}
\label{appendix:fancy_math_terms}

Here are some important mathematical terms that you will encounter throughout mathematics.

\begin{enumerate}
\item \textbf{Definition}---a precise and unambiguous description of the meaning of a mathematical term.  It characterizes the meaning of a word by giving all the properties and only those properties that must be true.
\item \textbf{Theorem}---a mathematical statement that is proved using rigorous mathematical reasoning.  In a mathematical paper, the term theorem is often reserved for the most important results.
\item \textbf{Proposition}---a proved and often interesting result, but generally less important than a theorem.
\item \textbf{Lemma}---a minor result whose sole purpose is to help in proving a theorem.  It is a stepping stone on the path to proving a theorem. Occasionally lemmas can take on a life of their own (Zorn's Lemma, Urysohn's Lemma, Burnside's Lemma, Sperner's Lemma).
\item \textbf{Corollary}---a result in which the (usually short) proof relies heavily on a given theorem (we often say that ``this is a corollary of Theorem A'').
\item \textbf{Conjecture}---a statement that is unproved, but is believed to be true (Collatz Conjecture, Goldbach Conjecture, Twin prime Conjecture).
\item \textbf{Claim}---an assertion that is then proved.  It is often used like an informal lemma.
\item \textbf{Counterexample}---a specific example showing that a statement is false.
\item \textbf{Axiom/Postulate}---a statement that is assumed to be true without proof. These are the basic building blocks from which all theorems are proved (Euclid's five postulates, axioms of ZFC, Peano axioms).
\item \textbf{Identity}---a mathematical expression giving the equality of two (often variable) quantities (trigonometric identities, Euler's identity).
\item \textbf{Paradox}---a statement that can be shown, using a given set of axioms and definitions, to be both true and false. Paradoxes are often used to show the inconsistencies in a flawed axiomatic theory (e.g., Russell's Paradox).  The term paradox is also used informally to describe a surprising or counterintuitive result that follows from a given set of rules (Banach-Tarski Paradox, Alabama Paradox, Gabriel's Horn).
\end{enumerate}