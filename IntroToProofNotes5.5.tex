\documentclass[11pt]{article}

\usepackage{amsfonts}
\usepackage{amsmath}
\usepackage{wasysym}
\usepackage{amssymb}
\usepackage{stmaryrd}
\usepackage{amsthm}
\usepackage{fancyhdr}
\usepackage[margin=1in]{geometry}
\usepackage[hang,flushmargin,symbol*]{footmisc}
\usepackage{color}
\definecolor{darkblue}{rgb}{0, 0, .6}
\definecolor{grey}{rgb}{.7, .7, .7}
\usepackage[breaklinks]{hyperref}
\hypersetup{
	colorlinks=true,
	linkcolor=darkblue,
	anchorcolor=darkblue,
	citecolor=darkblue,
	pagecolor=darkblue,
	urlcolor=darkblue,
	pdftitle={},
	pdfauthor={}
}

\newcommand{\dom}{\operatorname{Dom}}
\newcommand{\codom}{\operatorname{Codom}}
\newcommand{\range}{\operatorname{Rng}}

\pagestyle{fancy}

\lhead{\scriptsize Course Notes for Introduction to Proof (Version 1.1)} 
\rhead{\scriptsize Instructor: \href{http://danaernst.com}{D.C. Ernst}}
\lfoot{\scriptsize This work is an adaptation of notes written by Stan Yoshinobu of Cal Poly and Matthew Jones of California State University, Dominguez Hills.}
\cfoot{}

\renewcommand{\headrulewidth}{0.4pt} 
\renewcommand{\footrulewidth}{0.4pt} 

\theoremstyle{definition}
\newtheorem{theorem}{Theorem}[section]
\newtheorem{acknowledgement}[theorem]{Acknowledgement}
\newtheorem{algorithm}[theorem]{Algorithm}
\newtheorem{axiom}[theorem]{Axiom}
\newtheorem{case}[theorem]{Case}
\newtheorem{claim}[theorem]{Claim}
\newtheorem{conclusion}[theorem]{Conclusion}
\newtheorem{condition}[theorem]{Condition}
\newtheorem{conjecture}[theorem]{Conjecture}
\newtheorem{corollary}[theorem]{Corollary}
\newtheorem{criterion}[theorem]{Criterion}
\newtheorem{definition}[theorem]{Definition}
\newtheorem{example}[theorem]{Example}
\newtheorem{exercise}[theorem]{Exercise}
\newtheorem{journal}[theorem]{Journal}
\newtheorem{lemma}[theorem]{Lemma}
\newtheorem{notation}[theorem]{Notation}
\newtheorem{problem}[theorem]{Problem}
\newtheorem{proposition}[theorem]{Proposition}
\newtheorem{remark}[theorem]{Remark}
\newtheorem{solution}[theorem]{Solution}
\newtheorem{summary}[theorem]{Summary}
\newtheorem{question}[theorem]{Question}

\begin{document}

\addtocounter{section}{2}

\begin{section}{Relations and Functions}

\addtocounter{subsection}{4}
\addtocounter{theorem}{91}

\begin{subsection}{Compositions and Inverses}

\begin{definition}
If $f:X\to Y$ and $g:Y\to Z$ are functions, then a new function $g\circ f:X\to Z$ can be defined by $(g\circ f)(x)=g(f(x))$ for all $x\in\dom(f)$.
\end{definition}

\begin{remark}
It is important to notice that the function on the right is the one that ``goes first.''
\end{remark}

\begin{exercise}
In each case, give examples of finite sets $X$, $Y$, and $Z$, and functions $f:X\to Y$ and $g:Y\to Z$ that satisfy the given conditions.  Drawing bubble diagrams is sufficient.
\begin{enumerate}
\item $f$ is onto, but $g\circ f$ is not onto.
\item $g$ is onto, but $g\circ f$ is not onto.
\item $f$ is one-to-one, but $g\circ f$ is not one-to-one.
\item $g$ is one-to-one, but $g\circ f$ is not.
\end{enumerate}
\end{exercise}

\begin{theorem}[*]
If $f:X\to Y$ and $g:Y\to Z$ are both functions that are onto, then $g\circ f$ is also onto.
\end{theorem}

\begin{theorem}[*]
If $f:X\to Y$ and $g:Y\to Z$ are both functions that are one-to-one, then $g\circ f$ is also one-to-one.
\end{theorem}

\begin{corollary}
If $f:X\to Y$ and $g:Y\to Z$ are both one-to-one correspondences, then $g\circ f$ is also a one-to-one correspondence.
\end{corollary}

\begin{problem}
Assume that $f:X\to Y$ and $g:Y\to Z$ are both functions.  For each of the following statements, if the statement is true, then prove it.  If the statement is false, provide a counterexample.
\begin{enumerate}
\item If $g\circ f$ is one-to-one, then $f$ is one-to-one.
\item If $g\circ f$ is one-to-one, then $g$ is one-to-one.
\item If $g\circ f$ is onto, then $f$ is onto.
\item If $g\circ f$ is onto, then $g$ is onto.
\end{enumerate}
\end{problem}

\begin{definition}
Let $f:X\to Y$ be a function.  The relation $f^{-1}$, called \textbf{$f$ inverse}, is defined via
\[
f^{-1}=\{(f(x),x):x\in X\}.
\]
\end{definition}

\begin{remark}
Notice that we called $f^{-1}$ a relation and not a function.  In some circumstances $f^{-1}$ will be a function and sometimes it won't be.
\end{remark}

\begin{exercise}
Provide an example of a function $f:X\to Y$ such that $f^{-1}$ is \emph{not} a function.  A bubble diagram is sufficient.
\end{exercise}

\begin{exercise}
Provide an example of a function $f:X\to Y$ such that $f^{-1}$ is a function. A bubble diagram is sufficient.
\end{exercise}

\begin{theorem}[*]
Let $f:X\to Y$ be a function.  Then $f^{-1}$ is a function iff $f$ is \underline{\ \ \ \ \ \ \ \ \ \ \ \ \ \ \ \ \ \ \ \ \ \ \ \ }.
\end{theorem}

\begin{theorem}[*]\label{thm:comp of inverses}
Let $f:X\to Y$ be a function and suppose that $f^{-1}$ is a function.  Then
\begin{enumerate}
\item $(f\circ f^{-1})(x)=x$ for all $x\in Y$, and
\item $(f^{-1}\circ f)(x)=x$ for all $x\in X$.
\end{enumerate}
(You only need to prove one of these statements; the other is similar.)
\end{theorem}

\begin{theorem}[*]\label{thm:unique inverse}
Let $f:X\to Y$ and $g:Y\to X$ be functions such that $f$ is a one-to-one correspondence.  If $(f\circ g)(x)=x$ for all $x\in Y$ and $(g\circ f)(x)=x$ for all $x\in X$, then $g=f^{-1}$.
\end{theorem}

\begin{remark}
The upshot of the previous two theorems is that if $f^{-1}$ is a function, then it is the only one satisfying the two-sided ``undoing'' property exhibited in Theorem \ref{thm:comp of inverses}.
\end{remark}

The next theorem can be considered to be a converse of Theorem \ref{thm:unique inverse}.

\begin{theorem}[*]
Let $f:X\to Y$ and $g:Y\to X$ be functions satisfying $(f\circ g)(x)=x$ for all $x\in Y$ and $(g\circ f)(x)=x$ for all $x\in X$.  Then $f$ is a one-to-one correspondence.
\end{theorem}

\begin{theorem}[*]
Let $f:X\to Y$ and $g:Y\to Z$ be functions.  If $f$ and $g$ are both one-to-one correspondences, then $(g\circ f)^{-1}=f^{-1}\circ g^{-1}$.
\end{theorem}

\end{subsection}

\end{section}

\end{document}