\begin{section}{Countable Sets}

Recall that if $A=\emptyset$, then we say that $A$ has cardinality 0.  Also, if $\card(A)=\card([n])$ for $n\in\mathbb{N}$, then we say that $A$ has cardinality $n$.  We have a special way of describing sets that are in bijection with the natural numbers.

\begin{definition}
If $A$ is a set such that $\card(A)=\card(\mathbb{N})$, then we say that $A$ is \textbf{denumerable} and has \textbf{cardinality} $\mathbf{\aleph_0}$ (read ``aleph naught").
\end{definition}

Notice if a set $A$ has cardinality $1,2,\ldots$, or $\aleph_0$, we can label the elements in $A$ as ``first", ``second", and so on.  That is, we can ``count" the elements in these situations. Certainly, if a set has cardinality 0, counting isn't an issue.  This idea leads to the following definition.

\begin{definition}\label{def:countable}
A set $A$ is called \textbf{countable} iff $A$ is finite or denumerable. A set is called \textbf{uncountable} iff it is not countable.
\end{definition}

\begin{problem}
Quickly justify that each of the following set is countable. Feel free to appeal to previous problems.
\begin{enumerate}[label=\textrm{(\alph*)}]
\item The set $A:=\{a,b,c\}$
\item The set of odd natural numbers.
\item The set of even natural numbers.
\item The set $R:=\{\frac{1}{2^n}\mid n\in \mathbb{N}\}$.
\item The set of perfect squares.
\item The integers.
\item The set $\mathbb{N}\times \{x\}$, where $x\notin \mathbb{N}$.
\item The set $\mathbb{N}\times \mathbb{N}$.
\end{enumerate}
\end{problem}

\begin{theorem}
Let $A$ and $B$ be sets such that $A$ is countable. If $f:A\to B$ is a bijection, then $B$ is countable.
\end{theorem}

\begin{theorem}\label{thm:subsetsCountableSets}
Every subset of a countable set is countable.\footnote{\emph{Hint:} Let $A$ be a countable set.  Consider the cases when $A$ is finite versus infinite. The contrapositive of Corollary~\ref{cor:infiniteSetInfiniteSubset} should be useful for the case when $A$ is finite.}
\end{theorem}

\begin{theorem}
A set is countable iff it has the same cardinality of some subset of the natural numbers.
\end{theorem}

\begin{theorem}
If $f:\mathbb{N}\to A$ is an onto function, then $A$ is countable.
\end{theorem}

Loosely speaking, the next theorem tells us that we can arrange all of the rational numbers then count them ``first", ``second", ``third", etc. Given the fact that between any two distinct rational numbers on the number line, there are an infinite number of other rational numbers (justified by taking repeated midpoints), this may seem counterintuitive.  

\begin{theorem}
The set of rational numbers $\mathbb{Q}$ is countable.\footnote{\emph{Hint:} Make a table that column headings $0, 1, -1, 2,-2,\ldots$ and row headings $1,2,3,4,5,\ldots$.  If a column has heading $m$ and a row has heading $n$, then the corresponding entry in the table is given by the fraction $m/n$.  Find a way to zig-zag through the table making sure to hit every entry in the table (not including column and row headings) exactly once.  This justifies that there is a bijection between $\mathbb{N}$ and the entries in the table.  Do you see why?  Now, we aren't done yet because every rational number appears an infinite number of times in the table. Appeal to Theorem~\ref{thm:subsetsCountableSets}.}
\end{theorem}

\begin{theorem}
If $A$ and $B$ are countable sets, then $A\cup B$ is countable.
\end{theorem}

We would like to prove a stronger result than the previous theorem. To do so, we need a lemma.

\begin{lemma}
Let $\{A_n\}_{n=1}^{\infty}$ be a (countable) collection of sets. Define $B_1:=A_1$ and for each natural number $n>1$, define
\[
B_n:=A_n\setminus \bigcup_{i=1}^{n-1}A_i.
\]
Then we we have the following:
\begin{enumerate}[label=\textrm{(\alph*)}]
\item The collection $\{B_n\}_{n=1}^{\infty}$ is pairwise disjoint.
\item $\displaystyle \bigcup_{n=1}^{\infty}A_n =\bigcup_{n=1}^{\infty}B_n$.
\end{enumerate}
\end{lemma}

\begin{theorem}
Every countable union of countable sets is countable.\footnote{\emph{Hint:} A countable union is a union of countably many sets.  Recall that a countable set may be finite or infinite.  Consider three cases: (1) finite union of countable sets (use induction with base case $n=2$), (2) countably infinite union of finite sets, (3) countably infinite union of countably infinite sets.}
\end{theorem}

\begin{theorem}
If $A$ and $B$ are countable sets, then $A\times B$ is countable.
\end{theorem}

\begin{theorem}
The set of all finite sequences of 0's and 1's (e.g., $0110010$ is a finite sequence of 0's and 1') is countable. 
\end{theorem}



\end{section}