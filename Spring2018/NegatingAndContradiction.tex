\begin{section}{Negating Implications and Proof by Contradiction}

So far we have discussed how to negate propositions of the form $A$, $A\wedge B$, and $A\vee B$ for propositions $A$ and $B$.  However, we have yet to discuss how to negate propositions of the form $A\implies B$.  To begin, try proving the following result with a truth table.

\begin{theorem}\label{thm:ImplicationAsDisjuction}
The implication $A\implies B$ is equivalent to the disjunction $\neg A \vee B$.
\end{theorem}

The next result follows quickly from Theorem~\ref{thm:ImplicationAsDisjuction} together with DeMorgan's Law.

\begin{corollary}\label{cor:NegateImplication}
The proposition $\neg(A \implies B)$ is equivalent to $A \wedge \neg B$.
\end{corollary}

\begin{exercise}\label{exer:Darth Vader}
Let $A$ and $B$ be the propositions ``Darth Vader is a hippie'' and ``Sarah Palin is a liberal,'' respectively.
\begin{enumerate}[label=\textrm{(\alph*)}]
\item Express $A\implies B$ as an English sentence involving the disjunction ``or.''
\item Express $\neg(A\implies B)$ as an English sentence involving the conjunction ``and.''
\end{enumerate}
\end{exercise}

\begin{exercise}
The proposition ``If $.\overline{99}=\frac{9}{10}+\frac{9}{100}+\frac{9}{1000}+\cdots$, then $.\overline{99}\neq 1$'' is \emph{false}. Write its (true) negation, as a conjunction.
\end{exercise}

Recall that a proposition is exclusively either true or false---it never be both.

\begin{definition}
A compound proposition that is always false is called a \textbf{contradiction}.  A compound proposition that is always true is called a \textbf{tautology}.
\end{definition}

\begin{theorem}
For any proposition $A$, the proposition $\neg A\wedge A$ is a contradiction.
\end{theorem}

\begin{exercise}
Provide an example of a tautology using arbitrary propositions and any of the logical connectives $\neg$, $\wedge$, and $\vee$.  Prove that your example is in fact a tautology.
\end{exercise}

Suppose that we want to prove some proposition $P$ (which might be something like $A\implies B$ or even more complicated).  One approach, called \textbf{proof by contradiction}, is to assume $\neg P$ and then logically deduce a contradiction of the form $Q\wedge \neg Q$, where $Q$ is some proposition (possibly equal to $P$).  Since this is absurd, the assumption $\neg P$ must have been false, so $P$ is true.  The tricky part about a proof by contradiction is that it is not usually obvious what the statement $Q$ should be.

\begin{skeleton}[Proof of $P$ by contradiction]
Here is what the general structure for a proof by contradiction looks like if we are trying to prove the proposition $P$.

\begin{center}
\framebox{
\begin{minipage}{6in}
\vspace{.1in}
\begin{proof}
For sake of a contradiction, assume $\neg P$.
\begin{center}
$\ldots$ \ \emph{[Use definitions and known results to derive\\ some $Q$ and its negation $\neg Q$.]} \ $\ldots$\\
\end{center}
\noindent This is a contradiction. Therefore, $P$.
\end{proof}
\end{minipage}
}
\end{center}
\end{skeleton}

Proof by contradiction can be useful for proving statements of the form $A\implies B$, where $\neg B$ is easier to ``get your hands on,'' because $\neg(A \implies B)$ is equivalent to $A \wedge \neg B$ (see Corollary~\ref{cor:NegateImplication}).

\begin{skeleton}[Proof of $A\implies B$ by contradiction]\label{pf by contradiction for implication}
If you want to prove the implication $A\implies B$ via a proof by contradiction, then the structure of the proof is as follows.

\begin{center}
\framebox{
\begin{minipage}{6in}
\vspace{.1in}
\begin{proof}
For sake of a contradiction, assume $A$ and $\neg B$.
\begin{center}
$\ldots$ \ \emph{[Use definitions and known results to derive\\ some $Q$ and its negation $\neg Q$.]} \ $\ldots$\\
\end{center}
\noindent This is a contradiction.  Therefore, if $A$, then $B$.
\end{proof}
\end{minipage}
}
\end{center}
\end{skeleton}

Establish the following theorem in two ways: (i) prove the contrapositive, and (ii) prove via contradiction.

\begin{theorem}
Assume that $x\in\mathbb{Z}$.  If $x$ is odd, then 2 does not divide $x$. (Prove in two different ways.)
\end{theorem}

Prove the following theorem via contradiction. Afterward, consider the difficulties one might encounter when trying to prove the result more directly.

\begin{theorem}
Assume that $x,y\in\mathbb{N}$.\footnote{$\mathbb{N}=\{1,2,3,\ldots\}$ is the set of \textbf{natural numbers}. Some mathematicians (set theorists, in particular) include $0$ in $\mathbb{N}$, but this will not be our convention. The given statement is not true if we replace $\mathbb{N}$ with $\mathbb{Z}$. Do you see why?} If $x$ divides $y$, then $x\leq y$.
\end{theorem}

\end{section}