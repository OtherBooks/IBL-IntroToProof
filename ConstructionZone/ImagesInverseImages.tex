\begin{section}{Images and Inverse Images of Functions}\label{sec:Images and Inverse Images}

CONNECT DEFINITION TO THEOREM AT END OF PREVIOUS SECTION...

There are two important sets related to functions.

\begin{definition}
Let $f:X\to Y$ be a function.  \begin{enumerate}[label=\textrm{(\alph*)}]
\item If $S\subseteq X$, the \textbf{image} of $S$ under $f$ is defined via
\[
f(S):=\{f(x) \mid  x\in S\}.
\]
\item  If $T\subseteq Y$, the \textbf{inverse image} (or \textbf{preimage}) of $T$ under $f$ is defined via
\[
f^{-1}(T):=\{x\in X \mid  f(x)\in T\}.
\]
\end{enumerate}
\end{definition}

You've likely encountered inverse \emph{functions} before. But in this context, we are discussing inverse \emph{images}. It's important to point out that the use of the notation $f^{-1}$ does not make any assumptions about whether the inverse function exists. We will tackle inversion functions in the next section.

Note that the image of the domain is the same as the range of the function.  That is, $f(X)=\range(f)$. Similarly, the inverse image of the codomain is the domain.  That is, $f^{-1}(Y)=X$.

REWORD THIS AND PUT IN APPROPRIATE PLACE: There is real opportunity for confusion here: the letter f is being used in two different, though related, ways. We can apply f to elements of A to get elements of B, or we can apply it to subsets of A to get subsets of B. Similarly, we can talk about the images or preimages of either elements or subsets. Context should always make it clear what is meant, but you should be aware of the problem.

\begin{problem}
Define $f:\mathbb{Z}\to\mathbb{Z}$ via $f(x)=x^2$. Find $f(\{0,1,2\})$ and $f^{-1}(\{0,1,4\})$.
\end{problem}

\begin{problem}
Define $f:\mathbb{R}\to\mathbb{R}$ via $f(x)=3x^2-4$.
Find each of the following.
\begin{enumerate}[label=\textrm{(\alph*)}]
\item $f([-2,4])$
\item $f((-2,4))$
\item $f^{-1}([-10,1])$
\item $f^{-1}((-3,3))$
\item $f(\emptyset)$
\item $f(\mathbb{R})$
\item $f^{-1}(\emptyset)$
\item $f^{-1}(\mathbb{R})$
\item Find two nonempty subsets $A$ and $B$ of $\mathbb{R}$ such that $A\cap B=\emptyset$ but $f^{-1}(A)=f^{-1}(B)$.
\item Find two nonempty subsets $A$ and $B$ of $\mathbb{R}$ such that $A\cap B=\emptyset$ but $f(A)=f(B)$.
\end{enumerate}
\end{problem}

\begin{problem}
Let $f:X\to Y$ and consider the relation defined in Theorem~\ref{thm:equiv relation from inverse images}. For $x\in X$, what is the relationship between $f^{-1}(\{f(x)\})$ and the equivalence class containing $x$?  Justify your answer.
\end{problem}

\begin{problem}
Suppose $f:X\to Y$ is an injection and $A$ and $B$ are disjoint subsets of $X$. Are $f(A)$ and $f(B)$ necessarily disjoint subsets of $Y$?  If so, prove it. Otherwise, provide a counterexample.
\end{problem}

\begin{problem}
Find examples of functions $f$ and $g$ together with sets $S$ and $T$ such that $f(f^{-1}(T))\neq T$ and $g^{-1}(g(S))\neq S$.
\end{problem}

\begin{problem}
Let $f:X\to Y$ be a function and suppose $A, B\subseteq X$ and $C, D\subseteq Y$. Determine whether each of the following statements is true or false. If a statement is true, prove it.  Otherwise, provide a counterexample.
\begin{enumerate}[label=\textrm{(\alph*)}]
\item If $A\subseteq B$, then $f(A)\subseteq f(B)$.
\item If $C\subseteq D$, then $f^{-1}(C)\subseteq f^{-1}(D)$.
\item $f(A\cup B)\subseteq f(A)\cup f(B)$.
\item $f(A\cup B)\supseteq f(A)\cup f(B)$.
\item $f(A\cap B)\subseteq f(A)\cap f(B)$.
\item $f(A\cap B)\supseteq f(A)\cap f(B)$.
\item $f^{-1}(C\cup D)\subseteq f^{-1}(C)\cup f^{-1}(D)$.
\item $f^{-1}(C\cup D)\supseteq f^{-1}(C)\cup f^{-1}(D)$.
\item $f^{-1}(C\cap D)\subseteq f^{-1}(C)\cap f^{-1}(D)$.
\item $f^{-1}(C\cap D)\supseteq f^{-1}(C)\cap f^{-1}(D)$.
\item $A\subseteq f^{-1}(f(A))$.
\item $A\supseteq f^{-1}(f(A))$.
\item $f(f^{-1}(C))\subseteq C$.
\item $f(f^{-1}(C))\supseteq C$.
\end{enumerate}
\end{problem}

\begin{problem}
For each of the statements in the previous problem that were false, determine conditions, if any, on the corresponding sets that would make the statement true.
\end{problem}

\end{section}