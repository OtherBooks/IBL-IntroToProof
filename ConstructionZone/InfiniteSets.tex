\begin{section}{Infinite Sets}

In the previous section, we explored finite sets.  Now, let's tinker with infinite sets.

\begin{definition}\label{def:infiniteSet}
A set $A$ is \textbf{infinite} if and only if $A$ is not finite.
\end{definition}

Let's see if we can utilize this definition to prove that the set of natural numbers is infinite.

\begin{theorem}\label{thm:naturalNumbersInfinite}
The set $\mathbb{N}$ of natural numbers is infinite.\footnote{\emph{Hint:} For sake of a contradiction, assume otherwise.  Then there exists $n\in\mathbb{N}$ such that $\card([n])=\card(\mathbb{N})$, which implies that there exists a bijection $f:[n]\to \mathbb{N}$. What can you say about the number $m:=\max(f(1),f(2),\ldots,f(n))+1$?}
\end{theorem}

The next theorem is analogous to Theorem~\ref{thm:finiteSetsSameCardinality}, but for infinite sets. As we shall see later, the converse of this theorem is not generally true.

\begin{theorem}\label{thm:infiniteSetsSameCardinality}
If $A$ is infinite and $\card(A)=\card(B)$, then $B$ is infinite.\footnote{\emph{Hint:} Try a proof by contradiction. You should end up composing two bijections, say $f:A\to B$ and $g:B\to [n]$ for some $n\in\mathbb{N}$.}
\end{theorem}

\begin{exercise}\label{exer:someInfiniteSets}
Quickly verify that the following sets are infinite by appealing to Theorem~\ref{thm:naturalNumbersInfinite}, Theorem~\ref{thm:infiniteSetsSameCardinality}, and Problem~\ref{prob:cardinalityPractice}.
\begin{enumerate}[label=\textrm{(\alph*)}]
\item The set of odd natural numbers.
\item The set of even natural numbers.
\item The integers.
\item The set $R=\{\frac{1}{2^n}\mid n\in \mathbb{N}\}$.
\item The set $\mathbb{N}\times \{x\}$.
\end{enumerate}
\end{exercise}

Notice that Definition~\ref{def:infiniteSet} tells what infinite sets are not, but it doesn't really tell us what they are. In light of Theorem~\ref{thm:naturalNumbersInfinite}, one way of thinking about infinite sets is as follows.  Suppose $A$ is some nonempty set. Let's select a random element from $A$ and set it aside. We will call this element the ``first" element.  Then we select one of the remaining elements and set is aside, as well.  This is the ``second" element.  Imagine we continue this way, choosing a ``third" element, and ``fourth" element, etc.  If the set is infinite, we should never run out of elements to select. Otherwise, we would create a bijection with $[n]$ for some $n\in\mathbb{N}$.

The next problem, sometimes refered to as the Hilbert Hotel\footnote{The Hilbert Hotel is named after mathematician \href{https://en.wikipedia.org/wiki/David_Hilbert}{David Hilbert} (1862--1942).}, illustrates another way to think about infinite sets.

\begin{problem}
The Infinite Hotel has rooms numbered $1,2,3,4,\ldots$. Every room in the Infinite Hotel is currently occupied.  Is it possible to make room for one more guest (assuming they want a room all to themselves)?  An infinite number of new guests, say $g_1, g_2,g_3,\ldots$, show up in the lobby and each demands a room.  Is it possible to make room for all the new guests even in the hotel is already full?
\end{problem}

The previous problem verifies that a proper subset of the natural numbers is in bijection with the natural numbers themselves. We also witnessed this in parts (a) and (b) of Exercise~\ref{exer:someInfiniteSets}. Notice that Theorem~\ref{thm:cardinalityProperSubsetsFinite} forbids this type of behavior for finite sets. It turns out that this phenomenon is true for all infinite sets. The next theorem verifies that that the two viewpoints of infinite sets discussed above are valid.

\begin{theorem}\label{thm:infiniteSetTFAE}
Let $A$ be a set. Then the following statements are equivalent.\footnote{\emph{Hint:} Prove (i) if and only if (ii) and (ii) if and only if (iii). For (i) implies (ii), construct $f$ recursively. For (ii) implies (i), try a proof by contradiction. For (ii) implies (iii), let $B=A\setminus \{f(1),f(2),\ldots\}$ and show that $A$ can be put in bijection with $B\cup\{f(2),f(3),\ldots\}$. Lastly, for (iii) implies (ii), suppose $g:A\to C$ is a bijection for some proper subset $C$ of $A$. Let $a\in A\setminus C$. Define $f:\mathbb{N}\to A$ via $f(n)=g^n(a)$, where $g^n$ means compose $g$ with itself $n$ times.}
\begin{enumerate}[label=\textrm{(\roman*)}]
\item $A$ is an infinite set.
\item There exists a one-to-one function $f:\mathbb{N}\to A$.
\item $A$ can be put in one-to-one correspondence with a proper subset of $A$ (i.e., there exists a proper subset $B$ of $A$ such that $\card(B)=\card(A)$).
\end{enumerate}
\end{theorem}

\begin{corollary}\label{cor:infiniteSetInfiniteSubset}
A set is infinite if and only if it has an infinite subset.
\end{corollary}

\begin{corollary}
If $A$ is an infinite set, then $\card(\mathbb{N})\leq \card(A)$.
\end{corollary}

It is worth mentioning that for the previous theorem, (iii) implies (i) follows immediately from the contrapositive of Theorem~\ref{thm:cardinalityProperSubsetsFinite}.

\begin{problem}
Find a new proof of Theorem~\ref{thm:naturalNumbersInfinite} that uses (iii) implies (i) from Theorem~\ref{thm:infiniteSetTFAE}.
\end{problem}

\begin{exercise}\label{exer:moreInfiniteSets}
Quickly verify that the following sets are infinite by appealing to either Theorem~\ref{thm:infiniteSetTFAE} (use (ii) implies (i)) or Corollary~\ref{cor:infiniteSetInfiniteSubset}.
\begin{enumerate}[label=\textrm{(\alph*)}]
\item The set of odd natural numbers.
\item The set of even natural numbers.
\item The integers.
\item The set $\mathbb{N}\times \mathbb{N}$.
\item The set of rational numbers $\mathbb{Q}$.
\item The set of real numbers $\mathbb{R}$.
\item The set of perfect squares.
\item The interval $(0,1)$.
\item The set of complex numbers $\mathbb{C}:=\{a+bi\mid a,b\in\mathbb{R}\}$.
\end{enumerate}
\end{exercise}


\end{section}