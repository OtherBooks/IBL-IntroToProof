\begin{section}{Axioms of the Real Numbers}\label{sec:AxiomsRealNumbers}

Our axioms for the real numbers fall into three categories:
\begin{enumerate}
\item \textbf{Field Axioms:} These axioms provide the essential properties of arithmetic involving addition and subtraction.
\item \textbf{Order Axioms:} These axioms provide the necessary properties of inequalities.
\item \textbf{Completeness Axiom:} This axiom guarantees that the familiar number line representing the real numbers does not have any ``gaps". We will introduce this axiom in the next section.
\end{enumerate}

\begin{fieldaxioms}\label{axiom:field axioms}
There exist functions $(a,b)\mapsto a+b$ and $(a,b)\mapsto ab$ from $\mathbb{R}^2$ to $\mathbb{R}$ satisfying:
\begin{enumerate}
\item[(F1)] (Associativity for Addition) For all $a, b, c\in \mathbb{R}$ we have $(a+b)+c = a+(b+c)$;
\item[(F2)] (Commutativity for Addition) For all $a,b\in \mathbb{R}$, we have $a+b=b+a$;
\item[(F3)] (Additive Identity) There exists $0\in\mathbb{R}$ such that for all $a\in\mathbb{R}$, $0+a=a$;
\item[(F4)] (Additive Inverses) For all $a\in\mathbb{R}$ there exists $-a\in\mathbb{R}$ such that $a+(-a)=0$;

\item[(F5)] (Associativity for Multiplication) For all $a, b, c\in \mathbb{R}$ we have $(ab)c = a(bc)$;
\item[(F6)] (Commutativity for Multiplication) For all $a,b\in \mathbb{R}$, we have $ab=ba$;
\item[(F7)] (Multiplicative Identity) There exists $1\in\mathbb{R}$ such that $1\neq 0$ and for all $a\in\mathbb{R}$, $1a=a$;
\item[(F8)] (Multiplicative Inverses) For all $a\in\mathbb{R}\setminus\{0\}$ there exists $a^{-1}\in\mathbb{R}$ such that $aa^{-1}=1$.
\item[(F9)] (Distributive Property) For all $a,b,c\in \mathbb{R}$, $a(b+c)=ab+ac$;
\end{enumerate}
\end{fieldaxioms}

In the language of abstract algebra, Axioms (F1)--(F4) and (F5)--(F8) make each of $\mathbb{R}$ and $\mathbb{R}\setminus\{0\}$ an abelian group under addition and multiplication, respectively. Axiom~(F9) provides a way for the operations of addition and multiplication to interact.  Collectively, Axioms~(F1)--(F9) make the real numbers a \textbf{field}.  It follows from the axioms that the elements $0$ and $1$ of $\mathbb{R}$ are the unique \textbf{additive} and \textbf{multiplicative identities} in $\mathbb{R}$. To prove the following theorem, suppose $0$ and $0'$ are both additive identities in $\mathbb{R}$ and then show that $0=0'$. This shows that there can only be one additive identity. It might be helpful to review Skeleton Proof~\ref{skeleton:uniqueness}. 

\begin{theorem}\label{thm:unique additive identity}
The additive identity of $\mathbb{R}$ is unique.
\end{theorem}

To prove the next theorem, mimic the approach you used to prove Theorem~\ref{thm:unique additive identity}.

\begin{theorem}\label{thm:unique multiplicative identity}
The multiplicative identity of $\mathbb{R}$ is unique.
\end{theorem}

For every $a\in\mathbb{R}$, the elements $-a$ and $a^{-1}$ (as long as $a\neq 0$) are also the unique \textbf{additive} and \textbf{multiplicative inverses}. 

\begin{theorem}
Every real number has a unique additive inverse.
\end{theorem}

\begin{theorem}
Every nonzero real number has a unique multiplicative inverse.
\end{theorem}

We now introduce some common notation that you are likely familiar with.  Take a moment to think about why the following is a definition as opposed to an axiom or theorem.

\begin{definition}
For every $a,b\in\mathbb{R}$ and $n\in\mathbb{Z}$, we define the following:
\begin{enumerate}[label=\textrm{(\alph*)}]
\item $\boxed{a-b\coloneqq a+(-b)}$
\item $\boxed{\displaystyle\frac{a}{b}\coloneqq ab^{-1}}$ (for $b\neq 0$)
\item $\boxed{\displaystyle a^n\coloneqq \begin{cases}
\overbrace{aa\cdots a}^n, &\text{if }n\in \mathbb{N}\\
1, & \text{if }n=0\text{ and }a\neq 0\\
\displaystyle\frac{1}{a^{-n}}, & \text{if }-n\in \mathbb{N}\text{ and }a\neq 0
\end{cases}}$
\end{enumerate}
\end{definition}

Using the Field Axioms, we can prove each of the statements in the following theorem.

\begin{theorem}\label{thm:consequences of axioms}
For all $a,b,c\in\mathbb{R}$, we have the following:
\begin{enumerate}[label=\textrm{(\alph*)}]
\item $a=b$ if and only if $a+c=b+c$;
\item $0a=0$;
\item $-a=(-1)a$;
\item $(-1)^2 = 1$;
\item $-(-a)=a$;
\item If $a\neq 0$, then $(a^{-1})^{-1}=a$;
\item If $a\neq 0$ and $ab = ac$, then $b = c$.
\item If $ab=0$, then either $a=0$ or $b=0$.
\end{enumerate}
\end{theorem}

Carefully prove the next theorem by explicitly citing where you are utilizing the Field Axioms and Theorem~\ref{thm:consequences of axioms}.

\begin{theorem}
For all $a,b\in\mathbb{R}$, we have $(a+b)(a-b)=a^2-b^2$.
\end{theorem}

We now introduce the Order Axioms of the real numbers.

\begin{orderaxioms}\label{axiom:order axioms}
For $a,b,c\in \mathbb{R}$, there is a relation $\boxed{<}$ on $\mathbb{R}$ satisfying:
\begin{enumerate}
\item[(O1)] (Trichotomy Law) If $a\neq b$, then either $a<b$ or $b<a$ but not both;
\item[(O2)] (Transitivity) If $a<b$ and $b<c$, then $a<c$;
\item[(O3)] If $a<b$, then $a+c<b+c$;
\item[(O4)] If $a<b$ and $0<c$, then $ac<bc$;  
\end{enumerate}
\end{orderaxioms}

Given Axioms~(O1) and (O2) above, we say that the real numbers are \textbf{linearly ordered} (or \textbf{totally ordered}). We call numbers greater than zero \textbf{positive} and those greater than or equal to zero \textbf{nonnegative}. There are similar definitions for \textbf{negative} and \textbf{nonpositive}. 

Notice that the Order Axioms are phrased in terms of ``$<$". We would also like to be able to utilize ``$>$", ``$\leq$", and ``$\geq$".

\begin{definition}
For $a,b\in\mathbb{R}$, we define:
\begin{enumerate}[label=\textrm{(\alph*)}]
\item $\boxed{a>b}$ if $b<a$;
\item $\boxed{a\leq b}$ if $a<b$ or $a=b$;
\item $\boxed{a\geq b}$ if $b\leq a$.
\end{enumerate}
\end{definition}

Using the Order Axioms, we can prove many familiar facts. 

\begin{theorem}
For all $a,b\in\mathbb{R}$, if $a,b>0$, then $a+b>0$; and if $a,b<0$, then $a+b<0$.
\end{theorem}

The next result extends Axiom~(O3).

\begin{theorem}
For all $a,b,c,d\in\mathbb{R}$, if $a<b$ and $c<d$, then $a+c<b+d$.
\end{theorem}

\begin{theorem}\label{thm:additive inverse of a positive is negative}
For all $a\in\mathbb{R}$, $a>0$ if and only if $-a<0$.
\end{theorem}

\begin{theorem}
If $a$, $b$, $c$, and $d$ are positive real numbers such that $a<b$ and $c<d$, then $ac<bd$.
\end{theorem}

\begin{theorem}
For all $a,b\in\mathbb{R}$, we have the following:
\begin{enumerate}[label=\textrm{(\alph*)}]
\item $ab>0$ if and only if either $a,b>0$ or $a,b<0$;
\item $ab<0$ if and only if $a<0<b$ or $b<0<a$.
\end{enumerate}
\end{theorem}

\begin{theorem}
For all positive real numbers $a$ and $b$, $a< b$ if an only if $a^2< b^2$.
\end{theorem}

Consider using three cases when approaching the proof of the following theorem.

\begin{theorem}
For all $a\in\mathbb{R}$, we have $a^2\geq 0$.
\end{theorem}

It might come as a surprise that the following result requires proof.

\begin{theorem}
We have $0<1$.
\end{theorem}

The previous theorem together with Theorem~\ref{thm:additive inverse of a positive is negative} implies that $-1<0$ as you expect. It also follows from Axiom~(O3) that for all $n\in\mathbb{Z}$, we have $n<n+1$. We assume that there are no integers between $n$ and $n+1$.

\begin{theorem}
For all $a\in\mathbb{R}$, if $a>0$, then $a^{-1}>0$, and if $a<0$, then $a^{-1}<0$.
\end{theorem}

\begin{theorem}\label{thm:switch inequality using negative}
For all $a,b\in \mathbb{R}$, if $a<b$, then $-b<-a$. 
\end{theorem}

The last few results allow us to take for granted our usual understanding of which real numbers are positive and which are negative. The next theorem yields a result that extends Theorem~\ref{thm:switch inequality using negative}.

\begin{theorem}
For all $a,b,c\in \mathbb{R}$, if $a<b$ and $c<0$, then $bc<ac$. 
\end{theorem}

We could spend weeks building up from the axioms all of the necessary machinery for the real numbers. Instead we include a few additional axioms to save ourselves a little time.

\begin{additionalaxioms}\label{axiom:additional axioms}
The real numbers satisfy each of the following:
\begin{enumerate}
\item[(O5)] For every $x\in\mathbb{R}$, there exists $a,b\in\mathbb{R}$ such that $a<x<b$;
\item[(O6)] For every $a,b\in\mathbb{R}$, if $a<b$, there exists $x\in\mathbb{R}$ such that $a<x<b$ (in particular, $\frac{a+b}{2}$ is between $a$ and $b$);
\item[(O7)]\label{axiom:archimedean} For every $a\in\mathbb{R}$, there exists $n\in\mathbb{Z}$ such that $n \leq a < n + 1$.
\end{enumerate}	
\end{additionalaxioms}
  
Axiom~(O7) is sometimes referred to as the \textbf{Archimedean Property}. It turns out that we could derive this axiom from the \textbf{Completeness Axiom}, which we will introduce in the next section.

\begin{theorem}
For any positive real number $a$, there exists $N\in \mathbb{N}$ such that $0<\frac{1}{N}<a$.
\end{theorem}

Notice that Axiom~(O5) says that every real number is contained in a finite open interval. In particular, Axiom~(O7) says that every non-integer is contained in an open interval with consecutive integer endpoints. Axiom~(O6) tells us that every open interval is nonempty. In fact, repeated applications of Axiom~(O6) implies that every open interval contains infinitely many points.

Recall that the real numbers consist of rational and irrational numbers.  Two examples of an irrational number that you are likely familiar with are $\pi$ and  $\sqrt{2}$. In Section~\ref{sec:irrationality of root 2}, we will prove that $\sqrt{2}$ is irrational, but for now we will take this fact for granted. It turns out that $\sqrt{2}\approx 1.41421356237\in (1,2)$. This provides an example of an irrational number occurring between a pair of distinct rational numbers. The following two theorems are a good challenge to generalize this.

\begin{theorem}\label{thm:rationals dense}
Between any two distinct real numbers there is a rational number.
\end{theorem}

\begin{theorem}\label{thm:irrationals dense}
Between any two distinct real numbers there is an irrational number.
\end{theorem}

Repeated applications of the previous two theorems implies that every open interval contains infinitely many rational numbers and infinitely many irrational numbers. In light of these two theorems, we say that both the rationals and irrationals are \textbf{dense} in the real numbers. 

There is a special function that we can now introduce. 

\begin{definition}
Given $a\in\mathbb{R}$, we define the \textbf{absolute value of $a$}, denoted $|a|$, via
\[
\boxed{|a|\coloneqq \begin{cases}
a, & \text{if }a\geq 0\\
-a, & \text{if }a<0.
\end{cases}}
\]
\end{definition}

\begin{theorem}
For all $a\in\mathbb{R}$, $|a|\geq 0$ with equality only if $a=0$.
\end{theorem}

In the next theorem, writing $\pm a\leq b$ is an abbreviation for $a\leq b$ and $-a\leq b$.

\begin{theorem}
For all $a,b\in\mathbb{R}$, if $\pm a\leq b$, then $|a|\leq b$. 
\end{theorem}

\begin{theorem}
For all $a\in\mathbb{R}$, $|a|^2=a^2$.
\end{theorem}

\begin{theorem}\label{thm:plus minus less than or equal to abs value}
For all $a\in\mathbb{R}$, $\pm a\leq |a|$.
\end{theorem}

\begin{theorem}\label{thm:abs value less than or equal to iff squeezed by plus minus}
For all $a,r\in\mathbb{R}$, $|a|\leq r$ if and only if $-r\leq a\leq r$.
\end{theorem}

In the previous theorem, it must be the case that $r$ is nonnegative.  The letter $r$ was used because it is the first letter of the word ``radius". If $r$ is positive, we can think of the interval $(-r,r)$ as the interior of a 1-dimensional circle with radius $r$ centered at 0.

\begin{theorem}
For all $a,b\in\mathbb{R}$, $|ab|=|a||b|$.
\end{theorem}

Consider using Theorems~\ref{thm:plus minus less than or equal to abs value} and \ref{thm:abs value less than or equal to iff squeezed by plus minus} when attacking the next problem.  This result can be extremely useful in some contexts. 

\begin{theorem}[Triangle Inequality]
For all $a,b\in\mathbb{R}$, $|a+b|\leq |a|+|b|$.
\end{theorem}

The next theorem is related to the Triangle Inequality.

\begin{theorem}[Reverse Triangle Inequality]
For all $a,b\in\mathbb{R}$, $|a-b|\geq \left||a|-|b| \right|$.
\end{theorem}

\epigraph{If people do not believe that mathematics is simple, it is only because they do not realize how complicated life is.}{John von Neumann, mathematician}
\end{section}