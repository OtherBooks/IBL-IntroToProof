\begin{section}{Standard Topology of the Real Line}\label{sec:Topology}

In this section, we will introduce the notions of open, closed, and connected as they pertain to subsets of the real numbers.  These properties form the underpinnings of a branch of mathematics called \textbf{topology} (derived from the Greek words \emph{t\'opos}, meaning `place, location', and \emph{ology}, meaning `study of'). Topology, sometimes called ``rubber sheet geometry," is concerned with properties of spaces that are invariant under any continuous deformation (e.g., bending, twisting, and stretching like rubber while not allowing tearing apart or gluing together). The fundamental concepts in topology are continuity, compactness, and connectedness, which rely on ideas such as ``arbitrary close" and ``far apart". These ideas can be made precise using open sets.  

Once considered an abstract branch of pure mathematics, topology now has applications in biology, computer science, physics, and robotics. The goal of this section is to introduce you to the basics of the set-theoretic definitions used in topology and to provide you with an opportunity to tinker with open and closed subsets of the real numbers. In Section~\ref{sec:Continuity}, we will revisit these concepts and explore continuous functions.

For this entire section, our universe of discourse is the set of real numbers.  You may assume all the usual basic algebraic properties of the real numbers (addition, subtraction, multiplication, division, commutative property, distribution, etc.). We will often refer to an element in a subset of real numbers as a \textbf{point}.

\begin{definition}\label{def:open}
A set $U$ is called an \textbf{open set} if for every $x \in U$, there exists a finite-length open interval $(a,b)$ containing $x$ such that $(a,b)\subseteq U$.
\end{definition}

It follows immediately from the definition that every open set is a union of finite-length open intervals.

\begin{problem}\label{prob:open or not}
Determine whether each of the following sets is open. Justify your assertions.
\begin{multicols}{2}
\begin{enumerate}[label=\textrm{(\alph*)}]
\item $(1,2)$
\item $(1,\infty)$
\item $(1,2)\cup (\pi,5)$
\item $[1,2]$
\item $(-\infty,\sqrt{2}]$
\item $\{4,17,42\}$
\item $\{\frac{1}{n}\mid n\in \mathbb{N}\}$
\item $\{\frac{1}{n}\mid n\in \mathbb{N}\}\cup \{0\}$
\item $\mathbb{R}$
\item $\mathbb{Q}$
\item $\mathbb{Z}$
\item $\emptyset$
\end{enumerate}
\end{multicols}
\end{problem}

As expected, every open interval is an open set. Consider handling a few cases when approaching the proof of the next theorem.

\begin{theorem}
If $I$ is an open interval, then $I$ is an open set. 
\end{theorem}

It is important to point out that open sets can be more complicated than a single open interval.

\begin{problem}
Provide an example of an open set that is not a single open interval.
\end{problem}

\begin{theorem}\label{thm:finite union and intersection of open sets}
If $U$ and $V$ are open sets, then 
\begin{enumerate}[label=\textrm{(\alph*)}]
\item $U\cup V$ is an open set, and
\item $U\cap V$ is an open set.
\end{enumerate}
\end{theorem}

\begin{theorem}\label{thm:union of open sets}
If $\{U_{\alpha}\}_{\alpha\in\Delta}$ is a collection of open sets, then $\bigcup_{\alpha\in\Delta} U_{\alpha}$ is an open set.
\end{theorem}

\begin{problem}\label{prob:intersection of open sets}
Give an example of each of the following.
\begin{enumerate}[label=\textrm{(\alph*)}]
\item A collection of open sets $\{U_{\alpha}\}_{\alpha\in\Delta}$ such that $\bigcap_{\alpha\in\Delta} U_{\alpha}$ is an open set.
\item A collection of open sets $\{U_{\alpha}\}_{\alpha\in\Delta}$ such that $\bigcap_{\alpha\in\Delta} U_{\alpha}$ is not an open set.
\end{enumerate}
\end{problem}

\begin{remark}\label{rem:union vs intersection of open sets}
Taken together, Theorem~\ref{thm:finite union and intersection of open sets}, Theorem~\ref{thm:union of open sets}, and Problem~\ref{prob:intersection of open sets} tell us that the union of any arbitrary collection of open sets is open, but that the intersection of an arbitrary collection of open sets may or may not be open.  
\end{remark}

\begin{definition}
Suppose $A\subseteq \mathbb{R}$. A point $p\in \mathbb{R}$ is an \textbf{accumulation point} of $A$ if for every finite-length open interval $(a,b)$ containing $p$, there exists a point $q \in (a,b)\cap A$ such that $q\neq p$.
\end{definition}

Notice that if $p$ is an accumulation point of $A$, then $p$ may or may not be in $A$. Loosely speaking, $p$ is an accumulation point of a set $A$ if there are points in $A$ arbitrarily close to $p$. That is, if we zoom in on $p$, we should always see points in $A$ nearby.

\begin{problem}
Consider the open interval $I=(1,2)$. Prove each of the following.
\begin{enumerate}[label=\textrm{(\alph*)}]
\item The points $1$ and $2$ are accumulation points of $I$.
\item If $p\in I$, then $p$ is an accumulation point of $I$.
\item If $p<1$ or $p>2$, then $p$ is not an accumulation point of $I$.
\end{enumerate}
\end{problem}

\begin{theorem}
A point $p$ is an accumulation point of the intervals $(a,b)$, $(a,b]$, $[a,b)$, and $[a,b]$ if and only if $p\in [a,b]$.
\end{theorem}

\begin{problem}
Prove that the point $p=0$ is an accumulation point of $A=\{\frac{1}{n}\mid n \in \mathbb{N}\}$.  Are there any other accumulation points of $A$? 
\end{problem}

\begin{problem}
Provide an example of a set $A$ with exactly two accumulation points.
\end{problem}

Consider using Theorems~\ref{thm:rationals dense} and~\ref{thm:irrationals dense} when proving the next result.

\begin{theorem}
If $p\in\mathbb{R}$, then $p$ is an accumulation point of $\mathbb{Q}$.
\end{theorem}

\begin{definition}
A set is called \textbf{closed} if it contains all of its accumulation points.
\end{definition}

\begin{problem}\label{prob:closed or not}
Determine whether each of the sets in Problem~\ref{prob:open or not} is closed. Justify your assertions.
\end{problem}

The upshot of Part~(i) of Problems~\ref{prob:open or not} and \ref{prob:closed or not} is that $\mathbb{R}$ and $\emptyset$ are both open and closed.  It turns out that these are the only two subsets of the real numbers with this property.  Note that the fact that $\mathbb{R}$ is open and closed justifies referring $(-\infty, \infty)$ as both an open and a closed interval.

\begin{problem}\label{prob:open vs closed}
Provide an example of each of the following.  You do not need to prove that your answers are correct.
\begin{enumerate}[label=\textrm{(\alph*)}]
\item A set that is open but not closed.
\item A set that is closed but not open.
\item\label{prob:open vs closed last} A set that neither open nor closed.
\end{enumerate}
\end{problem}

One annoying feature of the terminology illustrated by Problem~\ref{prob:open vs closed} is that if a set is not open, it may or may not be closed.  Similarly, if a set is not closed, it may or may not be open.  That is, open and closed are not opposites of each other.

The next result justifies referring to $[a,b]$, $(-\infty,b]$, $[a,\infty)$, and $(-\infty,\infty)$ as closed intervals.

\begin{theorem}
If $I$ is a closed interval, then $I$ is a closed set.
\end{theorem}

\begin{theorem}
Every finite set is closed.
\end{theorem}

Despite the fact that open and closed are not opposites of each other, there is a nice relationship between open and closed sets in terms of complements.

\begin{theorem}
A set $U$ is open if and only if $U^C$ is closed.
\end{theorem}

\begin{theorem}\label{thm:finite union and intersection of closed sets}
If $A$ and $B$ are closed sets, then 
\begin{enumerate}[label=\textrm{(\alph*)}]
\item $A\cup B$ is a closed set, and
\item $A\cap B$ is a closed set.
\end{enumerate}
\end{theorem}

\begin{theorem}\label{thm:intersection of closed sets}
If $\{A_{\alpha}\}_{\alpha\in\Delta}$ is a collection of closed sets, then $\bigcap_{\alpha\in \Delta} A_{\alpha}$ is a closed set.
\end{theorem}

\begin{problem}\label{prob:union of closed sets}
Provide an example of a collection of closed sets $\{A_{\alpha}\}_{\alpha\in\Delta}$ such that $\bigcup_{\alpha\in \Delta} A_{\alpha}$ is not a closed set.
\end{problem}

\begin{problem}
In light of Theorem~\ref{thm:finite union and intersection of closed sets}, Theorem~\ref{thm:intersection of closed sets}, and Problem~\ref{prob:union of closed sets}, make an analogous statement to Remark~\ref{rem:union vs intersection of open sets}.
\end{problem}

In the next chapter, we will use mathematical induction to prove that the intersection of finitely many open sets is open and that the union of finitely many closed sets is closed.

\begin{problem}
Determine whether each of the following sets is open, closed, both, or neither.
\begin{enumerate}[label=\textrm{(\alph*)}]
\item $\displaystyle V=\bigcup_{n=2}^{\infty} \left(n - \frac{1}{2},n\right)$
\item $\displaystyle W=\bigcap_{n=2}^{\infty} \left(n - \frac{1}{2},n\right)$
\item $\displaystyle X=\bigcap_{n=1}^{\infty} \left(-\frac{1}{n}, \frac{1}{n}\right)$
\item $\displaystyle Y=\bigcap_{n=1}^{\infty} \left(-n, n\right)$
\item $Z=(0,1)\cap \mathbb{Q}$
\end{enumerate}
\end{problem}

\begin{problem}
Prove or provide a counterexample: Every non-closed set has at least one accumulation point.
\end{problem}

\begin{definition}
We say that a set $A\subseteq \mathbb{R}$ is \textbf{bounded above} if there is a point $b$ such that if $x\in A$ then $x\leq b$; such a point is an \textbf{upper bound} of $A$.
\end{definition}

\begin{problem}
The property of a set $M$ being \textbf{bounded below} and the notion of a \textbf{lower bound} are defined similarly; try defining them.
\end{problem}

\begin{definition}
We say that a set $A\subseteq \mathbb{R}$ is \textbf{bounded} if $A$ bounded above and below.
\end{definition}

Notice that a set $A$ is \textbf{bounded} if it is a subset of some finite-length closed interval.

\begin{problem}\label{prob:find upper bounds}
Find all upper bounds and all lower bounds for each of the following sets.
\begin{enumerate}[label=\textrm{(\alph*)}]
\item $(0,1]$
\item $(0,1]\cap \mathbb{Q}$
\item $(0,\infty)$
\item $\{42\}$
\end{enumerate}
\end{problem}

\begin{definition}
We say $p$ is a \textbf{supremum} (or \textbf{least upper bound}) of a set $A\subseteq \mathbb{R}$ if $p$ is an upper bound of $A$ and $p\leq b$ for any other upper bound $b$ of $A$. If the supremum of $A$ exists, it is denoted $\boxed{\sup(A)}$. Analogously, we say $p$ is a \textbf{infimum} (or \textbf{greatest lower bound}) of $A\subseteq \mathbb{R}$ if $p$ is a lower bound of $A$ and $p\geq b$ for any other lower bound $b$ of $A$. If the infimum of $A$ exists, it is denoted $\boxed{\inf(A)}$.
\end{definition}

\begin{problem}
Find the supremum and the infimum of each of the sets in Problem~\ref{prob:find upper bounds}.
\end{problem}

Implicit in the definition is that the supremum/infimum of a set is unique if it exists. 

\begin{theorem}
If $A\subseteq \mathbb{R}$ such that the supremum (respectively, infimum) exists, then the supremum (respectively, infimum) is unique.
\end{theorem}

The reason the supremum is so important is because of the following fundamental axiom.

\begin{axiom}[Completeness Axiom]\label{axiom:completeness}
If $A$ is a nonempty subset of $\mathbb{R}$ that is bounded above, then $\sup(A)$ exists.
\end{axiom}

Given the Completeness Axiom, we say that the real numbers satisfy the \textbf{least upper bound property}. Certainly, the real numbers satisfy the analogous result involving the infimum.

\begin{theorem}
If $A$ is a nonempty subset of $\mathbb{R}$ that is bounded below, then $\inf(A)$ exists.
\end{theorem}

At this point, it is not necessary that we complete the following problem, but you might find doing so to be a fun challenge.

\begin{problem}
Prove that the Archimedean Property follows from the Completeness Axiom.
\end{problem}

\begin{definition}
A set $K$ is called \textbf{compact} if $K$ is both closed and bounded.
\end{definition}

It is important to point out that there is a more general definition of compact in an arbitrary topological space.  However, using our notions of open and closed, it is a theorem that a subset of the real line is compact if and only if it is closed and bounded.

\begin{problem}
Provide several examples of sets that are compact and some that are not compact. Are finite sets compact? In particular, is the empty set compact?
\end{problem}

The next theorem says that every nonempty compact set contains its greatest lower bound and its least upper bound. In other words, every nonempty compact set attains a minimum and a maximum value.

\begin{theorem}
If $K$ is a nonempty compact set, then $\sup(K),\inf(K)\in K$.
\end{theorem}

We now introduce two additional classes of sets: connected and disconnected. 

\begin{definition}
We say that a set $A$ is \textbf{disconnected} if there exists two disjoint open sets $U_1$ and $U_2$ such that $A\cap U_1$ and $A\cap U_2$ are nonempty but $A\subseteq U_1\cup U_2$ (equivalently, $A=(A\cap U_1)\cup(A\cap U_2)$). If a set is not disconnected, then we say that it is \textbf{connected}.
\end{definition}

In other words, a set is disconnected if it can be partitioned into two nonempty subsets such that each subset does not contain points of the other and does not contain any accumulation points of the other. Showing that a set is disconnected is generally easier than showing a set is connected. To prove that a set is disconnected, you simply need to exhibit two open sets with the necessary properties. However, to prove that a set is connected, you need to prove that no such pair of open sets exists.

\begin{problem}
Determine whether each of the following sets is connected or disconnected.  Briefly justify your assertions.
\begin{multicols}{2}
\begin{enumerate}[label=\textrm{(\alph*)}]
\item $[0,1)\cup [2,3]$
\item $[0,1)\cup (1,2]$
\item $[0,1)\cup [1,2]$
\item $\mathbb{R}$
\item $\mathbb{Q}$
\item $\mathbb{R}\setminus\mathbb{Q}$
\item $\mathbb{Z}$
\item $\{\frac{1}{n}\mid n\in\mathbb{N}\}$
\item $[0,1]\cup\{1+\frac{1}{n}\mid n\in\mathbb{N}\}$
\item $\{17,42\}$
\item $\{17\}$
\item $\emptyset$
\end{enumerate}
\end{multicols}
\end{problem}

\begin{theorem}
If $a\in\mathbb{R}$, then $\{a\}$ is connected.
\end{theorem}

The proof of the next theorem is harder than you might expect. Consider a proof by contradiction and try to make use of the Completeness Axiom.

\begin{theorem}\label{thm:closed interval connected}
Every closed interval $[a,b]$ is connected.
\end{theorem}

It turns out that every connected set in $\mathbb{R}$ is either a singleton or an interval. We have not officially proved this claim, but we do have the tools to do so. Feel free to try your hand at proving this fact.

In this section, we have encountered open, closed, bounded, compact, disconnected, and connected subsets of $\mathbb{R}$. Of these types of sets, open and connected are relatively easy to understand and classify. That is, we have a sense of what all open sets and all connected sets ``look like."
\begin{itemize}
\item Open: A set is open if it is a union of (possibly infinitely many) finite-length open intervals.
%\item Bounded: A set is bounded if it is contained in a finite-length closed interval.
\item Connected: A set is connected if it is a singleton or an interval.
\end{itemize}
On the other hand, while we have precise definitions of closed, bounded, compact, and disconnected, these types of sets can take on various---and sometimes surprising---forms.



\end{section}