\chapter{Three Famous Theorems}\label{chap:three famous theorems}

\epigraphhead[70pt]{
\epigraph{A mathematician, like a painter or a poet, is a maker of patterns. If his patterns are more permanent than theirs, it is because they are made with ideas.}{G.H.~Hardy, mathematician}}

In the last few chapters, we have encountered all of the major proof techniques one needs in mathematics and enhanced our proof-writing skills.  In this chapter, we put these techniques and skills to work to prove three famous theorems, as well as numerous intermediate results along the way.  All of these theorems are ones you are likely familiar with from grade school, but perhaps these facts were never rigorously justified for you. 

In the first section, we develop all of the concepts necessary to state and then prove the \textbf{Fundamental Theorem of Arithmetic} (Theorem~\ref{thm:FTA}), which you may not recognize by name. The Fundamental Theorem of Arithmetic states that every natural number greater than 1 is the product of a unique combination of prime numbers. To prove the Fundamental Theorem of Arithmetic, we will need to make use of the \textbf{Division Algorithm} (Theorem~\ref{thm:DivisonAlgorthm}), which in turn utilizes the Well-Ordering Principle (Theorem~\ref{thm:WOP}). In the second section, we prove that $\sqrt{2}$ is irrational, which settles a claim made in Section~\ref{sec:AxiomsRealNumbers}. In the final section, we prove that there are infinitely many primes.