\chapter*{Preface}
\addcontentsline{toc}{chapter}{\protect\numberline{}Preface}

Mathematics is not about calculations, but ideas. My goal as a teacher is to provide students with the opportunity to grapple with these ideas and to be immersed in the process of mathematical discovery. Repeatedly engaging in this process hones the mind and develops mental maturity marked by clear and rigorous thinking. Like music and art, mathematics provides an opportunity for enrichment, experiencing beauty, elegance, and aesthetic value. The medium of a painter is color and shape, whereas the medium of a mathematician is abstract thought. The creative aspect of mathematics is what captivates me and fuels my motivation to keep learning and exploring.

While the content we teach our students is important, it is not enough. An education must prepare individuals to ask and explore questions in contexts that do not yet exist and to be able to tackle problems they have never encountered. It is important that we put these issues front and center and place an explicit focus on students producing, rather than consuming, knowledge. If we truly want our students to be independent, inquisitive, and persistent, then we need to provide them with the means to acquire these skills. Their viability as a professional in the modern workforce depends on their ability to embrace this mindset.

When I started teaching, I mimicked the experiences I had as a student. Because it was all I knew, I lectured. By standard metrics, this seemed to work out just fine. Glowing student and peer evaluations, as well as reoccurring teaching awards, indicated that I was effectively doing my job. People consistently told me that I was an excellent teacher. However, two observations made me reconsider how well I was really doing. Namely, many of my students seemed to depend on me to be successful, and second, they retained only some of what I had taught them. In the words of Dylan Retsek:
\begin{quotation}
``Things my students claim that I taught them masterfully, they don't know."
\end{quotation}
Inspired by a desire to address these concerns, I began transitioning away from direct instruction towards a more student-centered approach. The goals and philosophy behind inquiry-based learning (IBL) resonate deeply with my ideals, which is why I have embraced this paradigm. According to the Academy of Inquiry-Based Learning, IBL is a method of teaching that engages students in sense-making activities. Students are given tasks requiring them to solve problems, conjecture, experiment, explore, create, and communicate---all those wonderful skills and habits of mind that mathematicians engage in regularly. This book has IBL baked into its core.

The primary objectives of this book are to:
\begin{itemize}
\item Expand the mathematical content knowledge of the reader,
\item Provide an opportunity for the reader to experience the profound beauty of mathematics,
\item Allow the reader to exercise creativity in producing and discovering mathematics,
\item Enhance the ability of the reader to be a robust and persistent problem solver.
\end{itemize}
Ultimately, this is really a book about productive struggle and learning how to learn. Mathematics is simply the vehicle.

\epigraph{Much more important than specific mathematical results are the habits of mind used by the people who create those results. ... Although it is necessary to infuse courses and curricula with modern content, what is even more important is to give students the tools they will need in order to use, understand, and even make mathematics that does not yet exist.}{Cuoco, Goldenberg, \& Mark in \emph{Habit of Mind: An Organizing Principle for Mathematics Curriculum}}