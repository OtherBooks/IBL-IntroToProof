\chapter*{Preface}
\addcontentsline{toc}{chapter}{\protect\numberline{}Preface}

You are the creators. This book is a guide.

This book will not show you how to solve all the problems that are presented, but it should \emph{enable} you to find solutions, on your own and working together. The material you are about to study did not come together fully formed at a single moment in history. It was composed gradually over the course of centuries, with various mathematicians building on the work of others, improving the subject while increasing its breadth and depth.

Mathematics is essentially a human endeavor. Whatever you may believe about the true nature of mathematics—does it exist eternally in a transcendent Platonic realm, or is it contingent upon our shared human consciousness? is math ``invented" or ``discovered"?---our \emph{experience} of mathematics is temporal, personal, and communal. Like music, mathematics that is encountered only as symbols on a page remains inert. Like music, mathematics must be created in the moment, and it takes time and practice to master each piece. The creation of mathematics takes place in writing, in conversations, in explanations, and most profoundly in the mental construction of its edifices on the basis of reason and observation.

To continue the musical analogy, you might think of these notes like a performer's score. Much is included to direct you towards particular ideas, but much is missing that can only be supplied by you: participation in the creative process that will make those ideas come alive. Moreover, your success will depend on the pursuit of both \emph{individual} excellence and \emph{collective} achievement. Like a musician in an orchestra, you should bring your best work and be prepared to blend it with others' contributions.

In any act of creation, there must be room for experimentation, and thus allowance for mistakes, even failure. A key goal of our community is that we support each other---sharpening each other's thinking but also bolstering each other's confidence---so that we can make failure a \emph{productive} experience. Mistakes are inevitable, and they should not be an obstacle to further progress. It's normal to struggle and be confused as you work through new material. Accepting that means you can keep working even while feeling stuck, until you overcome and reach even greater accomplishments.

This book is a guide. You are the creators.