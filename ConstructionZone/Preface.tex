\chapter*{Preface}
\addcontentsline{toc}{chapter}{\protect\numberline{}Preface}


When I started teaching, I mimicked the experiences I had as a student. Because it was all I knew, I lectured. By standard metrics, this seemed to work out just fine. Glowing student and peer evaluations, as well as reoccurring teaching awards, indicated that I was effectively doing my job. People consistently told me that I was an excellent teacher. However, two observations made me reconsider how well I was really doing. Namely, many of my students seemed to depend on me to be successful, and second, they retained only some of what I had taught them. In the words of Dylan Retsek (Cal Poly):
``Things my students claim that I taught them masterfully, they don't know."
Inspired by a desire to address these concerns, I began transitioning away from direct instruction towards a more student-centered approach. The goals and philosophy behind inquiry-based learning (IBL) resonate deeply with my ideals, which is why I have embraced this paradigm. IBL is a pedagogical framework that I will elaborate on in the next section.


My primary objective is to play a role in developing persistent problem solvers...learn how to learn...content secondary...educated citizen...

provide students with the opportunity 


Mathematics is not about calculations, but ideas. My goal as an instructor is to provide students with the opportunity to grapple with these ideas and to be immersed in the process of mathematical discovery. Repeatedly engaging in this process hones the mind and develops mental maturity marked by clear and rigorous thinking. Bertrand Russell wrote that
``Mathematics, rightly viewed, possesses not only truth, but supreme beauty."
REWRITE THIS PART: Like music and art, mathematics provides an opportunity for enrichment. The medium of a painter is color and shape, whereas the medium of a mathematician is abstract thought. As a teacher, I attempt to convey the elegance and aesthetic value of mathematics, and I regularly remind my students that the creative aspect of mathematics is what captivates me and fuels my motivation to keep learning and exploring.

While the content we teach our students is important, it is not enough. What is perhaps more true than ever, is that we need individuals capable of asking and exploring questions in contexts that do not yet exist and to be able to tackle problems they have never encountered. It is important that we put these issues front and center and place an explicit focus on students producing, rather than consuming, knowledge. If we truly want our students to be independent, inquisitive, and persistent, then we need to provide them with the means to acquire these skills.


%shorter
This book is intended to be used for a one-semester introduction to proof course. The book is designed around an inquiry-based learning (IBL) approach to teaching and learning mathematics, but makes no assumptions about the specifics of how the instructor chooses to implement this approach.  Loosely speaking, IBL is a student-centered method of teaching that engages students in sense-making activities.  Students are given tasks requiring them to solve problems, conjecture, experiment, explore, create, and communicate.  Rather than showing facts or a clear, smooth path to a solution, the instructor guides and mentors students via well-crafted problems through an adventure in mathematical discovery. The book includes more content than one can expect to cover in a single semester. This allows the instructor to pick and choose the sections that suit their needs and desires. Each chapter takes a focused approach to the included topics, but also includes many gentle exercises aimed at developing intuition.


%longer
This book is intended to be used for a one-semester/quarter introduction to proof course (sometimes referred to as a transition to proof course). The intended audience is mathematics majors and minors. The book is designed around an inquiry-based learning (IBL) approach to teaching and learning mathematics, but makes no assumptions about the specifics of how the instructor chooses to implement this approach.  Loosely speaking, IBL is a student-centered method of teaching that engages students in sense-making activities.  Students are given tasks requiring them to solve problems, conjecture, experiment, explore, create, and communicate.  Rather than showing facts or a clear, smooth path to a solution, the instructor guides and mentors students via well-crafted problems through an adventure in mathematical discovery.  According to \href{https://www.colorado.edu/eer/sites/default/files/attached-files/laursenrasmussencommentaryauthorversion0219.pdf}{Laursen and Rasmussen (2019)}, the Four Pillars of IBL are:
\begin{itemize}
\item Students engage deeply with coherent and meaningful mathematical tasks.
\item Students collaboratively process mathematical ideas.
\item Instructors inquire into student thinking.
\item Instructors foster equity in their design and facilitation choices.
\end{itemize}

The book includes more content than one can expect to cover in a single semester. This allows the instructor to pick and choose the sections that suit their needs and desires. Each chapter takes a focused approach to the included topics, but also includes many gentle exercises aimed at developing intuition.




%%%%%%%%%

You are the creators. This book is a guide.

This book will not show you how to solve all the problems that are presented, but it should \emph{enable} you to find solutions, on your own and working together. The material you are about to study did not come together fully formed at a single moment in history. It was composed gradually over the course of centuries, with various mathematicians building on the work of others, improving the subject while increasing its breadth and depth.

Mathematics is essentially a human endeavor. Whatever you may believe about the true nature of mathematics---does it exist eternally in a transcendent Platonic realm, or is it contingent upon our shared human consciousness?---our \emph{experience} of mathematics is temporal, personal, and communal. Like music, mathematics that is encountered only as symbols on a page remains inert. Like music, mathematics must be created in the moment, and it takes time and practice to master each piece. The creation of mathematics takes place in writing, in conversations, in explanations, and most profoundly in the mental construction of its edifices on the basis of reason and observation.

To continue the musical analogy, you might think of this textbook like a performer's score. Much is included to direct you towards particular ideas, but much is missing that can only be supplied by you: participation in the creative process that will make those ideas come alive. Moreover, your success will depend on the pursuit of both \emph{individual} excellence and \emph{collective} achievement. Like a musician in an orchestra, you should bring your best work and be prepared to blend it with others' contributions.

In any act of creation, there must be room for experimentation, and thus allowance for mistakes, even failure. A key goal of our community is that we support each other---sharpening each other's thinking but also bolstering each other's confidence---so that we can make failure a \emph{productive} experience. Mistakes are inevitable, and they should not be an obstacle to further progress. It is normal to struggle and be confused as you work through new material. Accepting that means you can keep working even while feeling stuck, until you overcome and reach even greater accomplishments.

This book is a guide. You are the creators.