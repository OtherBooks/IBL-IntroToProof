\begin{section}{Cartesian Products of Sets}\label{sec:Cartesian Products}

Given a collection of sets, we can form new sets by taking unions, intersections, complements, and set differences.  In this section, we introduce a type of ``product" of sets.  You have already encountered this concept when you learned to plot points in the plane.  You also crossed paths with this notion if you have taken a course in linear algebra.

\begin{definition}
For each $n\in \mathbb{N}$, we define an \textbf{$n$-tuple} to be an ordered list of $n$ elements of the form $\boxed{(a_1, a_2,\ldots,a_n)}$. We refer to $a_i$ as the $i$th \textbf{component} (or \textbf{coordinate}) of $(a_1, a_2,\ldots,a_n)$. Two $n$-tuples $(a_1, a_2,\ldots,a_n)$ and $(b_1, b_2,\ldots,b_n)$ are equal if $a_i=b_i$ for all $1\leq i\leq n$. A $2$-tuple $(a,b)$ is more commonly referred to as an \textbf{ordered pair} while a $3$-tuple $(a,b,c)$ is often called an \textbf{ordered triple}.
\end{definition}

Occasionally, other symbols are used to surround the components of an $n$-tuple, such as square brackets ``$[\ ]$" or angle brackets ``$\langle\ \rangle$". In some programming languages, curly braces ``$\{\ \}$" are used to specify arrays. However, we avoid this convention in mathematics since curly braces are the standard notation for sets. The term ``tuple" can also occur when discussing other mathematical objects, such as vectors.

We can use the notion of $n$-tuples to construct new sets from existing sets.

\begin{definition}
If $A$ and $B$ are sets, the \textbf{Cartesian product} (or \textbf{direct product}) of $A$ and $B$, denoted $A\times B$ (read as ``$A$ times $B$" or ``$A$ cross $B$"), is the set of all ordered pairs where the first component is from $A$ and the second component is from $B$. In set-builder notation, we have
\[
\boxed{A\times B\coloneqq \{(a,b)\mid a\in A, b\in B\}}.
\]
We similarly define the Cartesian product of $n$ sets, say $A_1, \ldots, A_n$, by
\[
\boxed{\prod_{i=1}^{n} A_i\coloneqq A_1\times \cdots \times A_n\coloneqq \{(a_1,\ldots,a_n)\mid  a_j\in A_j \mbox{ for all }1\leq j\leq n\}}\ ,
\]
where $A_i$ is referred to as the $i$th \textbf{factor} of the Cartesian product. As a special case, the set 
\[
\underbrace{A\times \cdots \times A}_{n\text{ factors}}
\]
is often abbreviated as $A^n$.
\end{definition}

Cartesian products are named after French philosopher and mathematician \href{https://en.wikipedia.org/wiki/Rene_Descartes}{Ren\'e Descartes} (1596--1650).

\begin{example}\label{ex:CartesianProduct}
If $A=\{a,b,c\}$ and $B=\{\smiley,\frownie\}$, then 
\[
A\times B=\{(a,\smiley), (a,\frownie),(b,\smiley),(b,\frownie), (c,\smiley),(c,\frownie)\}.
\]
\end{example}

\begin{example}
The standard two-dimensional plane $\mathbb{R}^2$ and standard three space $\mathbb{R}^{3}$ are familiar examples of Cartesian products.  In particular, we have
\[
\mathbb{R}^2=\mathbb{R}\times \mathbb{R}=\{(x,y)\mid x,y\in \mathbb{R}\}
\]
and
\[
\mathbb{R}^3=\mathbb{R}\times \mathbb{R}\times \mathbb{R}=\{(x,y,z)\mid x,y,z\in \mathbb{R}\}.
\]
\end{example}

\begin{problem}
Consider the sets $A$ and $B$ from Example~\ref{ex:CartesianProduct}.
\begin{enumerate}[label=\textrm{(\alph*)}]
\item Find $B\times A$. 
\item Find $B\times B$.
\end{enumerate}
\end{problem}

\begin{problem}
If $A$ and $B$ are sets, why do think that $A\times B$ is referred to as a type of ``product"? Think about the area model for multiplication of natural numbers.
\end{problem}

\begin{problem}
If $A$ and $B$ are both finite sets, then how many elements will $A\times B$ have?
\end{problem}

\begin{problem} 
Let $A=\{1, 2, 3\}$, $B=\{1,2\}$, and $C=\{1,3\}$. Find $A \times B\times C$. 
\end{problem}

\begin{problem}
Let $X=[0,1]$ and $Y=\{1\}$.  Write each of the following using set-builder notation and then describe the set geometrically (e.g., draw a picture). 
\begin{enumerate}[label=\textrm{(\alph*)}]
\item $X\times Y$
\item $Y\times X$
\item $X\times X$
\item $Y\times Y$
\end{enumerate}
\end{problem}

\begin{problem}
If $A$ is a set, then what is $A\times \emptyset$ equal to?
\end{problem}

\begin{problem}
Given sets $A$ and $B$, when will $A\times B$ be equal to $B\times A$?
\end{problem}

We now turn our attention to subsets of Cartesian products.

\begin{problem}\label{prob:some lines}
Write $\mathbb{N}\times \mathbb{R}$ using set-builder notation and then describe this set geometrically by interpreting it as a subset of $\mathbb{R}^2$.
\end{problem}

\begin{theorem}
Let $A$, $B$, $C$, and $D$ be sets. If $A\subseteq C$ and $B\subseteq D$, then $A\times B\subseteq C\times D$.
\end{theorem}

\begin{problem}
Is it true that if $A\times B\subseteq C\times D$, then $A\subseteq C$ and $B\subseteq D$?  Don't forget to think about cases involving the empty set.
\end{problem}

\begin{problem}
Is every subset of $C\times D$ of the form $A\times B$, where $A\subseteq C$ and $B\subseteq D$?  If so, prove it.  If not, find a counterexample.
\end{problem}

\begin{problem}
If $A$, $B$, and $C$ are nonempty sets, is $A\times B$ a subset of $A\times B\times C$?
\end{problem}

\begin{problem}
Let $A=[2,5]$, $B=[3,7]$, $C=[1,3]$, and $D=[2,4]$.  Compute each of the following.
\begin{enumerate}[label=\textrm{(\alph*)}]
\item $(A\cap B)\times (C\cap D)$
\item $(A\times C)\cap (B\times D)$
\item $(A\cup B)\times (C\cup D)$
\item $(A\times C)\cup (B\times D)$
\item $A\times (B\cap C)$
\item $(A\times B)\cap (A\times C)$
\item $A\times (B\cup C)$
\item $(A\times B)\cup (A\times C)$
\end{enumerate}
\end{problem}

\begin{problem}
Let $A$, $B$, $C$, and $D$ be sets.  Determine whether each of the following statements is true or false.  If a statement is true, prove it.  Otherwise, provide a counterexample.
\begin{enumerate}[label=\textrm{(\alph*)}]
\item $(A\cap B)\times (C\cap D)=(A\times C)\cap (B\times D)$
\item $(A\cup B)\times (C\cup D)=(A\times C)\cup (B\times D)$
\item $A\times (B\cap C)=(A\times B)\cap (A\times C)$
\item $A\times (B\cup C)=(A\times B)\cup (A\times C)$
\item $A\times (B\setminus C) = (A\times B)\setminus (A\times C)$
\end{enumerate}
\end{problem}

\begin{problem}
If $A$ and $B$ are sets, conjecture a way to rewrite $(A\times B)^C$ in a way that involves $A^C$ and $B^C$ and then prove your conjecture.
\end{problem}

\end{section}