\begin{section}{Cartesian Products of Sets}

ADD INTRO BLURB HERE...

\begin{definition}
For each $n\in \mathbb{N}$, we define an \textbf{$n$-tuple} to be an ordered list of $n$ elements of the form $(a_1, a_2,\ldots,a_n)$. We refer to $a_i$ as the $i$th \textbf{component} (or \textbf{coordinate}) of $(a_1, a_2,\ldots,a_n)$. Two $n$-tuples $(a_1, a_2,\ldots,a_n)$ and $(b_1, b_2,\ldots,b_n)$ are equal if and only if $a_i=b_i$ for all $1\leq i\leq n$. A $2$-tuple $(a,b)$ is more commonly referred to as an \textbf{ordered pair} while a $3$-tuple $(a,b,c)$ is often called an \textbf{ordered triple}.
\end{definition}

We can use the notion of $n$-tuples to construct new sets from existing sets.

\begin{definition}
If $A$ and $B$ are sets, the \textbf{Cartesian product} of $A$ and $B$ is defined by
\[
A\times B:=\{(a,b)\mid a\in A, b\in B\}.
\]
That is, $A\times B$ is the set of all ordered pairs where the first component is from $A$ and the second component is from $B$.  We similarly define the Cartesian product of $n$ sets, say $A_1, \ldots, A_n$, by
\[
\prod_{i=1}^{n} A_i:=A_1\times \cdots \times A_n:=\{(a_1,\ldots,a_n)\mid  a_j\in A_j \mbox{ for all }1\leq j\leq n\},
\]
where $A_i$ is referred to as the $i$th \textbf{factor} of the Cartesian product. As a special case, the set 
\[
\underbrace{A\times \cdots \times A}_{n\text{ factors}}
\]
is often abbreviated as $A^n$.
\end{definition}

\begin{example}\label{ex:CartesianProduct}
If $A=\{a,b,c\}$ and $B=\{\smiley,\frownie\}$, then 
\[
A\times B=\{(a,\smiley), (a,\frownie),(b,\smiley),(b,\frownie), (c,\smiley),(c,\frownie)\}.
\]
\end{example}

\begin{example}\label{ex:CartesianProduct}
The standard two-dimensional plane $\mathbb{R}^2$ and standard three space $\mathbb{R}^{3}$ are familiar examples of Cartesian products.  In particular, we have
\[
\mathbb{R}^2=\mathbb{R}\times \mathbb{R}=\{(x,y)\mid x,y\in \mathbb{R}\}
\]
and
\[
\mathbb{R}^3=\mathbb{R}\times \mathbb{R}\times \mathbb{R}=\{(x,y,z)\mid x,y,z\in \mathbb{R}\}.
\]
\end{example}

\begin{problem}
Consider the sets $A$ and $B$ from Example~\ref{ex:CartesianProduct}
\begin{enumerate}[label=\textrm{(\alph*)}]
\item Find $B\times A$. 
\item Find $B\times B$.
\end{enumerate}
\end{problem}

\begin{problem}
If $A$ and $B$ are both finite sets, then how many elements will $A\times B$ have?
\end{problem}

\begin{problem} 
Let $A=\{1, 2, 3\}$, $B=\{1,2\}$, and $C=\{1,3\}$. Find $A \times B\times C$. 
\end{problem}

\begin{problem}
Let $X=[0,1]$ and $Y=\{1\}$.  Write each of the following using set builder notation and then describe the set geometrically (e.g., draw a picture). 
\begin{enumerate}[label=\textrm{(\alph*)}]
\item $X\times Y$
\item $Y\times X$
\item $X\times X$
\item $Y\times Y$
\end{enumerate}
\end{problem}

\begin{problem}\label{prob:some lines}
Write $\mathbb{N}\times \mathbb{R}$ using set builder notation and then describe this set geometrically by interpreting it as a subset of $\mathbb{R}^2$.
\end{problem}

\begin{problem}
Given sets $A$ and $B$, when will $A\times B$ be equal to $B\times A$?
\end{problem}

\end{section}