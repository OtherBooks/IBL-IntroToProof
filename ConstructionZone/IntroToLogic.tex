\begin{section}{Introduction to Logic}

After diving in head first in the last section, we'll take a step back and do a more careful examination of what it is we are actually doing.

\begin{definition}
A \textbf{proposition} (or \textbf{statement}) is a sentence that is either true or false.
\end{definition}

For example, the sentence ``All liberals are hippies'' is a false proposition.  However, the perfectly good sentence ``$x=1$'' is \emph{not} a proposition all by itself since we don't actually know what $x$ is.

\begin{exercise} Determine whether the following are propositions or not. Explain.
\begin{enumerate}[label=\textrm{(\alph*)}]
\item All cars are red.
\item Led Zeppelin is the best band of all time. 
%\item Van Gogh was the best artist ever. 
%\item If my name is Joe, then my name starts with the letter J.
\item If my name starts with the letter J, then my name is Joe.
%\item $f$ is continuous.
%\item All functions are continuous.
%\item If $f$ is a differentiable function, then $f$ is continuous function.
%\item The president had eggs for breakfast the morning of his tenth birthday.
%\item What time is it?
\item $x^2=4$.
\item There exists an $x$ such that $x^2=4$.
\item For all real numbers $x$, $x^2=4$.
\item $\sqrt{2}$ is an irrational number.
%\item For all real numbers $x$, $x^3=x$.
%\item There exists a real number $x$ such that $x^3=x$.
\item $p$ is prime.
\end{enumerate}
\end{exercise}

Given two propositions, we can form more complicated propositions using logical connectives.

\begin{definition}
Let $A$ and $B$ be propositions.  
\begin{enumerate}[label=\textrm{(\alph*)}]
\item The proposition ``\textbf{not $A$}" is true iff\footnote{Throughout mathematics, the phrase ``if and only if'' is common enough that it is often abbreviated ``iff.'' Roughly speaking, this phrase/word means ``exactly when.''} $A$ is false; expressed symbolically as $\boxed{\neg A}$ and called the \textbf{negation} of $A$.
\item The proposition ``\textbf{$A$ and $B$}'' is true iff both $A$ and $B$ are true; expressed symbolically as $\boxed{A \wedge B}$ and called the \textbf{conjunction} of $A$ and $B$.
\item The proposition ``\textbf{$A$ or $B$}'' is true iff at least one of $A$ or $B$ is true; expressed symbolically as $\boxed{A \vee B}$ and called the \textbf{disjunction} of $A$ and $B$.
\item The proposition ``\textbf{If $A$, then $B$}'' is true iff both $A$ and $B$ are true, or $A$ is false; expressed symbolically as $\boxed{A \implies B}$ and called an \textbf{implication} or \textbf{conditional statement}. Note that $A \implies B$ may also be read as ``$A$ implies $B$" or ``$A$ only if $B$". 
\end{enumerate}
\end{definition}

\begin{exercise}
Describe the meaning of $\neg (A \wedge B)$ and $\neg (A \vee B)$.
\end{exercise}

\begin{exercise}\label{exer:translations}
Let $A$ represent ``6 is an even number'' and $B$ represent ``6 is a multiple of 4.''  Express each of the following in ordinary English sentences and state whether the statement is true or false.
\begin{enumerate}[label=\textrm{(\alph*)}]
  \item $A \wedge B$
  \item $A \vee B$
  \item $\neg A$
  \item $\neg B$
  \item $\neg (A \wedge B)$
  \item $\neg (A \vee B)$
  \item $A \implies B$
\end{enumerate}
\end{exercise}

\begin{definition}
A \textbf{truth table} is a table that illustrates all possible truth values for a proposition.  
\end{definition}

\begin{example}
Let $A$ and $B$ be propositions.  Then the truth table for the conjunction $A\wedge B$ is given by the following.
\begin{center}
\begin{tabular}{@{}ccc@{}}
\toprule
$A$  &  $B$ & $A \wedge B$  \\
\midrule
T & T & T  \\ 
T & F & F  \\ 
F & T & F  \\ 
F & F & F  \\
\bottomrule
\end{tabular}
\end{center}
Notice that we have columns for each of $A$ and $B$.  The rows for these two columns correspond to all possible combinations for $A$ and $B$.  The third column gives us the truth value of $A\wedge B$ given the possible truth values for $A$ and $B$.
\end{example}

Note that each proposition has two possible truth values: true or false. Thus, if a compound proposition $P$ is built from $n$ propositions, then the truth table for $P$ will require $2^n$ rows.

\begin{exercise}
Create a truth table for each of $A \vee B$, $\neg A$, $\neg (A \wedge B)$, and $\neg A \wedge \neg B$.  Feel free to add additional columns to your tables to assist you with intermediate steps.
\end{exercise}

\begin{problem}
A coach promises, ``If we win tonight, then I will buy you pizza tomorrow.''  Determine the case(s) in which the players can rightly claim to have been lied to. Use this to help create a truth table for $A \implies B$.
\end{problem}

\begin{definition}
Two statements $P$ and $Q$ are (\textbf{logically}) \textbf{equivalent}, expressed symbolically as $\boxed{P\iff Q}$ and read ``$P$ iff $Q$", iff they have the same truth table.
\end{definition}

Each of the next three facts can be justified using truth tables.

\begin{theorem}
If $A$ is a proposition, then $\neg(\neg A)$ is equivalent to $A$.
\end{theorem}

\begin{theorem}[DeMorgan's Law]\label{thm:demorgan}
If $A$ and $B$ are propositions, then $\neg(A \wedge B) \iff \neg A \vee \neg B$.
\end{theorem}

\begin{problem}
Let $A$ and $B$ be propositions.  Conjecture a statement similar to Theorem~\ref{thm:demorgan} for the proposition $\neg(A\vee B)$ and then prove it. This is also called DeMorgan's Law.
\end{problem}

\begin{definition}\label{def:converse}
The \textbf{converse} of $A \implies B$ is $B \implies A$.
\end{definition}

\begin{exercise}
Provide an example of a true conditional proposition whose converse is false.
\end{exercise}

\begin{definition}
The \textbf{contrapositive} of $A \implies B$ is $\neg B \implies \neg A$.
\end{definition}

\begin{exercise}
Let $A$ and $B$ represent the statements from Exercise~\ref{exer:translations}.  Express the following in ordinary English sentences.
\begin{enumerate}[label=\textrm{(\alph*)}]
\item The converse of $A \implies B$.
\item The contrapositive of $A \implies B$.
\end{enumerate}
\end{exercise}

\begin{exercise}
Find the converse and the contrapositive of the following statement: ``If a person lives in Flagstaff, then that person lives in Arizona."
\end{exercise}

%\begin{exercise}
%Find the contrapositive of the following statements: 
%\begin{enumerate}
%\item If $n$ is an even natural number, then $n+1$ is an odd natural number.\footnote{Despite the fact that each of ``$n$ is an even natural number" and ``$n+1$ is an odd natural number" are not propositions (since we cannot determine their truth values without knowing what $n$ is), the implication is a proposition.  We discuss this further when we introduce predicates.}
%\item If it rains today, then I will bring my umbrella.
%\item If it does not rain today, then I will not bring my umbrella.
%\end{enumerate}
%\end{exercise}

Use truth tables to prove the following theorem.

\begin{theorem}\label{thm:contrapos}
The implication ${A\implies B}$ is equivalent to its contrapositive.
\end{theorem}

The upshot of Theorem~\ref{thm:contrapos} is that if you want to prove a conditional proposition, you can prove its contrapositive instead.  Try proving each of the next three theorems by proving the contrapositive of the given statement instead.

\begin{theorem}
Assume $x,y\in\mathbb{Z}$.  If $xy$ is odd, then both $x$ and $y$ are odd.
\end{theorem}

\begin{theorem}
Assume $x,y\in\mathbb{Z}$.  If $xy$ is even, then $x$ or $y$ is even.
\end{theorem}

\begin{theorem}
Assume $x\in\mathbb{Z}$. If $x^2$ is even, then $x$ is even.
\end{theorem}

\end{section}