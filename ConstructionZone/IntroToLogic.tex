\begin{section}{Introduction to Logic}

After diving in head first in the last section, we'll take a step back and do a more careful examination of what it is we are actually doing.

\begin{definition}
A \textbf{proposition} (or \textbf{statement}) is a sentence that is either true or false.
\end{definition}

For example, the sentence ``All dogs have four legs.''~is a false proposition.  However, the perfectly good sentence ``$x=1$'' is \emph{not} a proposition all by itself since we don't actually know what $x$ is.

\begin{problem} Determine whether the following are propositions or not. Explain.
\begin{enumerate}[label=\textrm{(\alph*)}]
\item All cars are red.
\item Every person whose name begins with J has the name Joe.
\item $x^2=4$.
\item There exists an $x$ such that $x^2=4$.
\item For all real numbers $x$, $x^2=4$.
\item $\sqrt{2}$ is an irrational number.
\item $p$ is prime.
\item Led Zeppelin is the best band of all time.
\end{enumerate}
\end{problem}

Given two propositions, we can form more complicated propositions using logical connectives.

\begin{definition}
Let $A$ and $B$ be propositions.  
\begin{enumerate}[label=\textrm{(\alph*)}]
\item The proposition ``\textbf{not $A$}" is true if and only if\footnote{Throughout mathematics, the phrase ``if and only if'' is common enough that it is sometimes abbreviated ``iff.'' Roughly speaking, this phrase/word means ``exactly when.''} $A$ is false; expressed symbolically as $\boxed{\neg A}$ and called the \textbf{negation} of $A$.
\item The proposition ``\textbf{$A$ and $B$}'' is true if and only if both $A$ and $B$ are true; expressed symbolically as $\boxed{A \wedge B}$ and called the \textbf{conjunction} of $A$ and $B$.
\item The proposition ``\textbf{$A$ or $B$}'' is true if and only if at least one of $A$ or $B$ is true; expressed symbolically as $\boxed{A \vee B}$ and called the \textbf{disjunction} of $A$ and $B$.
\item The proposition ``\textbf{If $A$, then $B$}'' is true if and only if both $A$ and $B$ are true, or $A$ is false; expressed symbolically as $\boxed{A \implies B}$ and called an \textbf{implication} or \textbf{conditional statement}. Note that $A \implies B$ may also be read as ``$A$ implies $B$" or ``$A$ only if $B$". 
\end{enumerate}
\end{definition}

Each of the theorems that we proved in Section~\ref{sec:baby number theory} are examples of conditional statements. However, some of the statements were disguised as such. For example, Theorem~\ref{thm:two consecutive ints} states, ``The sum of two consecutive integers is odd." We can reword this theorem as, ``If $x\in\mathbb{Z}$, then $x+(x+1)$ is odd."

\begin{problem}
Reword Theorem~\ref{thm:sum of three consecutive integers} so that it explicitly reads as a conditional statement.
\end{problem}

The proofs of each of the theorems in Section~\ref{sec:baby number theory} had the same format, which we refer to as a \textbf{direct proof}. 

\begin{skeleton}[Proof of $A\implies B$ by direct proof]

If you want to prove the implication $A\implies B$ via a direct proof, then the structure of the proof is as follows.

\begin{center}
\framebox{
\begin{minipage}{6in}
\vspace{.1in}
\begin{proof}
Assume $A$.
\begin{center}
$\ldots$ \ \emph{[Use definitions and known results to derive $B$]} \ $\ldots$\\
\end{center}
\noindent Therefore, $B$.
\end{proof}
\end{minipage}
}
\end{center}
\end{skeleton}

\begin{problem}
Describe the meaning of $\neg (A \wedge B)$ and $\neg (A \vee B)$.
\end{problem}

\begin{problem}\label{prob:translations}
Let $A$ represent ``6 is an even number'' and $B$ represent ``6 is a multiple of 4.''  Express each of the following in ordinary English sentences and state whether the statement is true or false.
\begin{enumerate}[label=\textrm{(\alph*)}]
  \item $A \wedge B$
  \item $A \vee B$
  \item $\neg A$
  \item $\neg B$
  \item $\neg (A \wedge B)$
  \item $\neg (A \vee B)$
  \item $A \implies B$
\end{enumerate}
\end{problem}

\begin{definition}
A \textbf{truth table} is a table that illustrates all possible truth values for a proposition.  
\end{definition}

\begin{example}
Let $A$ and $B$ be propositions.  Then the truth table for the conjunction $A\wedge B$ is given by the following.
\begin{center}
\begin{tabular}{@{}ccc@{}}
\toprule
$A$  &  $B$ & $A \wedge B$  \\
\midrule
T & T & T  \\ 
T & F & F  \\ 
F & T & F  \\ 
F & F & F  \\
\bottomrule
\end{tabular}
\end{center}
Notice that we have columns for each of $A$ and $B$.  The rows for these two columns correspond to all possible combinations for $A$ and $B$.  The third column gives us the truth value of $A\wedge B$ given the possible truth values for $A$ and $B$.
\end{example}

Note that each proposition has two possible truth values: true or false. Thus, if a compound proposition $P$ is built from $n$ propositions, then the truth table for $P$ will require $2^n$ rows.

\begin{problem}
Create a truth table for each of $A \vee B$, $\neg A$, $\neg (A \wedge B)$, and $\neg A \wedge \neg B$.  Feel free to add additional columns to your tables to assist you with intermediate steps.
\end{problem}

\begin{problem}
A coach promises, ``If we win tonight, then I will buy you pizza tomorrow.''  Determine the case(s) in which the players can rightly claim to have been lied to. Use this to help create a truth table for $A \implies B$.
\end{problem}

\begin{definition}
Two statements $P$ and $Q$ are (\textbf{logically}) \textbf{equivalent}, expressed symbolically as $\boxed{P\iff Q}$ and read ``$P$ if and only if $Q$", if and only if they have the same truth table.
\end{definition}

Each of the next three facts can be justified using truth tables.

\begin{theorem}
If $A$ is a proposition, then $\neg(\neg A)$ is equivalent to $A$.
\end{theorem}

\begin{theorem}[DeMorgan's Law]\label{thm:demorgan}
If $A$ and $B$ are propositions, then $\neg(A \wedge B) \iff \neg A \vee \neg B$.
\end{theorem}

\begin{problem}
Let $A$ and $B$ be propositions.  Conjecture a statement similar to Theorem~\ref{thm:demorgan} for the proposition $\neg(A\vee B)$ and then prove it. This is also called DeMorgan's Law.
\end{problem}

\begin{definition}\label{def:converse}
The \textbf{converse} of $A \implies B$ is $B \implies A$.
\end{definition}

\begin{problem}
Provide an example of a true conditional proposition whose converse is false.
\end{problem}

\begin{definition}\label{def:inverse}
The \textbf{inverse} of $A \implies B$ is $\neg A \implies \neg B$.
\end{definition}

\begin{problem}
Provide an example of a true conditional proposition whose inverse is false.
\end{problem}

\begin{definition}
The \textbf{contrapositive} of $A \implies B$ is $\neg B \implies \neg A$.
\end{definition}

\begin{problem}
Let $A$ and $B$ represent the statements from Problem~\ref{prob:translations}.  Express the following in ordinary English sentences.
\begin{enumerate}[label=\textrm{(\alph*)}]
\item The converse of $A \implies B$.
\item The contrapositive of $A \implies B$.
\end{enumerate}
\end{problem}

\begin{problem}
Find the converse and the contrapositive of the following statement: ``If a person lives in Flagstaff, then that person lives in Arizona."
\end{problem}

%\begin{problem}
%Find the contrapositive of the following statements: 
%\begin{enumerate}
%\item If $n$ is an even natural number, then $n+1$ is an odd natural number.\footnote{Despite the fact that each of ``$n$ is an even natural number" and ``$n+1$ is an odd natural number" are not propositions (since we cannot determine their truth values without knowing what $n$ is), the implication is a proposition.  We discuss this further when we introduce predicates.}
%\item If it rains today, then I will bring my umbrella.
%\item If it does not rain today, then I will not bring my umbrella.
%\end{enumerate}
%\end{problem}

Use a truth table to prove the following theorem.

\begin{theorem}\label{thm:contrapos}
The implication ${A\implies B}$ is equivalent to its contrapositive.
\end{theorem}

The upshot of Theorem~\ref{thm:contrapos} is that if you want to prove a conditional proposition, you can prove its contrapositive instead, called \textbf{proof by contraposition}.

\begin{skeleton}[Proof of $A\implies B$ by contraposition]\label{pf by contraposition}
If you want to prove the implication $A\implies B$ by proving its contrapositive $\neg B\implies \neg A$ instead, then the structure of the proof is as follows.

\begin{center}
\framebox{
\begin{minipage}{6in}
\vspace{.1in}
\begin{proof}
We will prove that if $A$, then $B$ by proving its contrapositive.  Assume $\neg B$.
\begin{center}
$\ldots$ \ \emph{[Use definitions and known results to derive $\neg A$]} \ $\ldots$\\
\end{center}
\noindent This proves that if $\neg B$, then $\neg A$.  Therefore, if $A$, then $B$.
\end{proof}
\end{minipage}
}
\end{center}
\end{skeleton}

\begin{problem}
Consider the following statement:
\begin{quote}
Assume $x\in\mathbb{Z}$. If $x^2$ is odd, then $x$ is an odd integer.
\end{quote}
The items below can be assembled to form a proof of this statement, but they are currently out of order.  Put them in the proper order.
\begin{enumerate}
\item Thus, we assume that $x$ is an even integer.

\item We will prove this by contraposition.
\item Thus, $x^2$ is twice an integer.
\item Since $x=2k$, we have that $x^2 =(2k)^2 =4k^2$.
\item Since $k$ is an integer, $2k^2$ is also an integer.
\item By the definition of even, there is an integer $k$ such that $x=2k$.
\item Since the contrapositive is equivalent to the original statement and we have proved the contrapositive, the original statement is true.
\item By the definition of even integer, $x^2$ is an even integer.
\item The contrapositive is ``If $x$ is an even integer, then $x^2$ is an even integer."
\item Notice that $x^2 = 2(2k^2)$.
\end{enumerate}
\end{problem}

Try proving each of the next three theorems by proving the contrapositive of the given statement.

\begin{theorem}
Assume $x\in\mathbb{Z}$. If $x^2$ is even, then $x$ is even.
\end{theorem}

\begin{theorem}
Assume $x,y\in\mathbb{Z}$.  If $xy$ is odd, then both $x$ and $y$ are odd.
\end{theorem}

\begin{theorem}
Assume $x,y\in\mathbb{Z}$.  If $xy$ is even, then $x$ or $y$ is even.
\end{theorem}

\end{section}