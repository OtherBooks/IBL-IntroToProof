\begin{section}{Equivalence Relations}

As we have seen in the previous section, the notions of reflexive, symmetric, and transitive are independent of each other. That is, a relation may have some combination of these properties, possibly none of them and possibly all of them.  However, we have a special name for when a relation satisfies all three.

\begin{definition}
Let $\sim$ be a relation on a set $A$.  Then $\sim$ is called an \textbf{equivalence relation} on $A$ if and only if $\sim$ is reflexive, symmetric, and transitive.
\end{definition}

\begin{problem}\label{prob:digraph}
Let $S=\{1,2,3,4,5,6\}$ and define
\[
{\sim}=\{(1,1),(1,6),(2,2),(2,3),(2,4),(3,3),(3,2),(3,4),(4,4),(4,2),(4,3),(5,5),(6,6),(6,1)\}.
\]
Using $\sim$, complete each of the following.
\begin{enumerate}[label=\textrm{(\alph*)}]
\item Draw the digraph for $\sim$.
\item Determine whether $\sim$ is an equivalence relation on $A$.
\item Find $\Omega_{\sim}$ by determining $[x]$ for each $x\in S$.
\end{enumerate}
\end{problem}

\begin{problem}\label{prob:made up}
Let $A=\{a,b,c,d,e\}$.  
\begin{enumerate}[label=\textrm{(\alph*)}]
\item Make up an equivalence relation $\sim$ on $A$ by drawing a digraph such that $a$ is not related to $b$ and $c$ is not related to $b$.  
\item Using your digraph, find $\Omega_{\sim}$ by determining $[x]$ for each $x\in A$.
\end{enumerate}
\end{problem}

\begin{problem}
Given a finite set $A$ and a relation $\sim$ on $A$, describe what the corresponding digraph would have to look like in order for $\sim$ to be an equivalence relation.
\end{problem}

\begin{problem}\label{prob:lots of them}
Determine which relations given in Problem~\ref{prob:lots of relations} are equivalence relations.
\end{problem}

\begin{problem}
Let $\mathcal{T}$ be the set of all triangles and define $\sim$ on $\mathcal{T}$ via $T_1\sim T_2$ if and only if $T_1$ is similar to $T_2$.  Determine whether $\sim$ is an equivalence relation on $\mathcal{T}$.
\end{problem}

\begin{problem}
If possible, construct an equivalence relation on the empty set.  If this is not possible, explain why.
\end{problem}

\begin{theorem}\label{thm:related if and only if same class}
Suppose $\sim$ is an equivalence relation on a set $A$ and let $a,b\in A$.  Then $[a]=[b]$ if and only if $a\sim b$.
\end{theorem}

\begin{theorem}\label{thm:equiv yields partition}
Suppose $\sim$ is an equivalence relation on a set $A$.  Then
\begin{enumerate}[label=\textrm{(\alph*)}]
\item $\displaystyle \bigcup_{a\in A}\ [a]=A$, and
\item For all $a,b\in A$, either $[a]=[b]$ or $[a]\cap [b]=\emptyset$.
\end{enumerate}
\end{theorem}

In light of Theorem~\ref{thm:equiv yields partition}, we have the following definition.

\begin{definition}\label{def:equivalence class}
If $\sim$ is an equivalence relation on a set $A$, then for each $a\in A$, we refer to $[a]$ as the \textbf{equivalence class} of $a$.
\end{definition}

Another common notation for the equivalence class of $x$ is $\overline{x}$.  When $\sim$ is an equivalence relation on a set $A$, the collection of equivalence classes is denoted by $A/\mathord\sim$, which is read as ``$A$ modulo $\sim$" or ``$A$ mod $\sim$".  The collection $A/\mathord\sim$ is sometimes referred to as the \textbf{quotient set of $A$ by $\sim$}. Note that $\Omega_{\sim}$ equals $A/\mathord\sim$ whenever $\sim$ is an equivalence relation.

The upshot of Theorem~\ref{thm:equiv yields partition} is that given an equivalence relation, every element lives in exactly one equivalence class.  In the next section, we will see that we can run this in reverse.  That is, if we separate out the elements of a set so that every element is an element of exactly one subset (like the bins of my kids' toys), then this determines an equivalence relation.

\begin{example}
The collection of sets of relatives that you found in part~\ref{prob:mod 5} of Problem~\ref{prob:lots of them} is the set of equivalence classes modulo 5.
\end{example}

\begin{problem}
If $\sim$ is an equivalence relation on a finite set $A$, describe $A/\mathord\sim$ in terms of the digraph corresponding to $\sim$.
\end{problem}

\begin{problem}
For each of the equivalence relations in Problem~\ref{prob:lots of them}, succinctly describe the corresponding equivalence classes.
\end{problem}

\end{section}