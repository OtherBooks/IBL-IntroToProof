\begin{section}{Introduction to Relations}

\begin{definition}
An \textbf{ordered pair} is an object of the form $(x,y)$. Two ordered pairs $(x,y)$ and $(a,b)$ are \textbf{equal} if and only if $x=a$ and $y=b$. 
\end{definition}

\begin{definition}
An \textbf{$n$-tuple} is an object of the form $(x_1, x_2,\ldots,x_n)$.  Each $x_i$ is referred to as the $i$th \textbf{component}.
\end{definition}

Note that an ordered pair is just a 2-tuple.

\begin{definition}
If $X$ and $Y$ are sets, the \textbf{Cartesian product} of $X$ and $Y$ is defined by
\[
X\times Y:=\{(x,y)\mid x\in X, y\in Y\}.
\]
That is, $X\times Y$ is the set of all ordered pairs where the first element is from $X$ and the second element is from $Y$.  We similarly define the Cartesian product of $n$ sets, say $X_1, \ldots, X_n$, by
\[
\prod_{i=1}^{n} X_i:=X_1\times \cdots \times X_n:=\{(x_1,\ldots,x_n)\mid  x_j\in X_j \mbox{ for all }1\leq j\leq n\},
\]
where $X_i$ is referred to as the $i$th \textbf{factor} of the Cartesian product and $x_j$ is referred to as the $j$th \textbf{coordinate} of $(x_1,\ldots,x_n)$. The set 
\[
\underbrace{X\times \cdots \times X}_{n\text{ factors}}
\]
is often abbreviated as $X^n$.
\end{definition}

\begin{example}\label{ex:CartesianProduct}
If $A=\{a,b,c\}$ and $B=\{\smiley,\frownie\}$, then 
\[
A\times B=\{(a,\smiley), (a,\frownie),(b,\smiley),(b,\frownie), (c,\smiley),(c,\frownie)\}.
\]
\end{example}

\begin{example}\label{ex:CartesianProduct}
The standard two-dimensional plane $\mathbb{R}^2$ and standard three space $\mathbb{R}^{3}$ are familiar examples of Cartesian products.  In particular, we have
\[
\mathbb{R}^2=\mathbb{R}\times \mathbb{R}=\{(x,y)\mid x,y\in \mathbb{R}\}
\]
and
\[
\mathbb{R}^3=\mathbb{R}\times \mathbb{R}\times \mathbb{R}=\{(x,y,z)\mid x,y,z\in \mathbb{R}\}.
\]
\end{example}

\begin{problem}
Consider the sets $A$ and $B$ from Example~\ref{ex:CartesianProduct}
\begin{enumerate}[label=\textrm{(\alph*)}]
\item Find $B\times A$. 
\item Find $B\times B$.
\end{enumerate}
\end{problem}

\begin{problem}
If $X$ and $Y$ are both finite sets, then how many elements will $X\times Y$ have?
\end{problem}

\begin{problem} 
Let $A=\{1, 2, 3\}$, $B=\{1,2\}$, and $C=\{1,3\}$. Find $A \times B\times C$. 
\end{problem}

\begin{problem}
Let $X=[0,1]$ and $Y=\{1\}$.  Write each of the following using set builder notation and then describe the set geometrically (e.g., draw a picture). 
\begin{enumerate}[label=\textrm{(\alph*)}]
\item $X\times Y$
\item $Y\times X$
\item $X\times X$
\item $Y\times Y$
\end{enumerate}
\end{problem}

\begin{problem}\label{prob:some lines}
Write $\mathbb{N}\times \mathbb{R}$ using set builder notation and then describe this set geometrically by interpreting it as a subset of $\mathbb{R}^2$.
\end{problem}

\begin{problem}
Given sets $X$ and $Y$, when will $X\times Y$ be equal to $Y\times X$?
\end{problem}

\begin{definition}
Let $X$ and $Y$ be sets. A \textbf{relation} from a set $X$ to a set $Y$ is a subset of $X \times Y$. A relation on $X$ is a subset of $X \times X$.  
\end{definition}

\begin{example}
You may not realize it, but you are familiar with many relations.  For example, on the real numbers, we have the relation $\leq$.  We could say that $(3,\pi)$ is in the relation $\leq$ since $3\leq \pi$.  However, $(1,-1)$ is not in the relation since $1\nleq -1$.  Order matters!
\end{example}

\begin{example}
The set $\mathbb{N}\times \mathbb{R}$ from Problem~\ref{prob:some lines} is an example of a relation on $\mathbb{R}$.
\end{example}

Different notations for relations are used in different contexts.  When talking about relations in the abstract, we indicate that a pair $(a,b)$ is in the relation by some notation like $a\sim b$, which is read ``$a$ is related to $b$."

\begin{example}
REPLACE THIS ONE. Let $P$ denote the set of all people with accounts on Facebook and define $F$ on $P$ via $xFy$ if and only if $x$ is friends with $y$.  Then $F$ is a relation on $P$.
\end{example}

\begin{example}\label{ex:digraph}
Let $A$ be any set.  Since $\emptyset \subseteq A\times A$, the empty set forms a relation on $A$. This relation is called the \textbf{empty relation} on $A$.
\end{example}

We can often represent relations using digraphs.  Given a finite set $X$ and a relation $\sim$ on $X$, a \textbf{digraph} (short for \emph{directed graph}) is a discrete graph having the members of $X$ as vertices and a directed edge from $x$ to $y$ if and only if $x\sim y$.

\begin{example}
Figure~\ref{fig:digraph} depicts a digraph that represents a relation $R$ given by
\[
R=\{(a,a),(a,b),(a,c),(b,b),(b,a),(b,c),(c,d),(c,e),(d,d),(d,a),(d,c),(e,a)\}.
\]

\begin{figure}[h!]
\begin{center}
\begin{tikzpicture}[->,>=stealth',shorten >=1pt,auto,node distance=2cm,semithick]
  \tikzstyle{every state}=[]

  \node[state] (A)                    {$a$};
  \node[state]         (B) [above right of=A] {$b$};
  \node[state]         (D) [below right of=A] {$d$};
  \node[state]         (C) [below right of=B] {$c$};
  \node[state]         (E) [below of=D]       {$e$};

  \path (B) edge [bend left]  node {} (A)
        (A) edge [bend left]  node {} (B)
            edge              node {} (C)
        (B) edge [loop above] node {} (B)
            edge [bend left]  node {} (C)
        (D) edge [bend left]  node {} (C)
        (C) edge [bend left]  node {} (D)
            edge [bend left]  node {} (E)
        (A) edge [loop left] node {} (A)
        (D) edge [loop below] node {} (D)
            edge              node {} (A)
        (E) edge [bend left]  node {} (A);
\end{tikzpicture}
\caption{An example of a digraph for a relation.}\label{fig:digraph}
\end{center}
\end{figure}

\end{example}

\begin{problem}
Let $A=\{a,b,c\}$ and define ${\sim}=\{(a,a),(a,b),(b,c),(c,b),(c,a)\}$.  
\begin{enumerate}[label=\textrm{(\alph*)}]
\item Draw the digraph for $\sim$.
\item Draw the digraph for the empty relation on $A$.
\end{enumerate}
\end{problem}

\begin{problem}
Let $A=\{1,2,3,4,5,6\}$.  Define $|$ on $A$ via $x|y$ if and only if $x$ divides $y$.  Draw the digraph for $|$ on $A$.
\end{problem}

When $X$ or $Y$ is infinite, it is not practical to draw a digraph.  However, you are familiar with the graphs of some relations involving infinite sets.

\begin{example}
When we write $x^2+y^2=1$, we are implicitly defining a relation.  In particular, the relation is the set of ordered pairs $(x,y)$ satisfying $x^2+y^2=1$, namely $\{(x,y)\mid x^2+y^2=1\}$. The graph of this relation in $\mathbb{R}^2$ is the unit circle centered at the origin.
\end{example}

\begin{problem}
Define $\sim$ on $\mathbb{R}$ via $x\sim y$ if and only if $x\leq y$.  Draw a picture of this relation in $\mathbb{R}^2$. In other words, draw all points $(x,y)$ where $x\sim y$.
\end{problem}

\begin{definition}
Let $\sim$ be a relation on a set $A$.
\begin{enumerate}[label=\textrm{(\alph*)}]
\item The relation $\sim$ is \textbf{reflexive} if for all $x\in A$, $x\sim x$ (every element is related to itself).
\item The relation $\sim$ is \textbf{symmetric} if for all $x,y\in A$, if $x\sim y$, then $y\sim x$.
\item The relation $\sim$ is \textbf{transitive} if for all $x,y,z\in A$, if $x\sim y$ and $y\sim z$, then $x\sim z$.
\end{enumerate}
\end{definition}

\begin{example}
Here are a few examples that illustrate the concepts in the previous definition.
\begin{enumerate}[label=\textrm{(\alph*)}]
\item The relation $=$ on $\mathbb{R}$ is reflexive, symmetric, and transitive.
\item The relation $\leq$ is reflexive and transitive on $\mathbb{R}$, but not symmetric. However, notice that $<$ is transitive on $\mathbb{R}$, but neither symmetric nor reflexive.
\item If $S$ is a set, then $\subseteq$ on $\mathcal{P}(S)$ is reflexive and transitive, but not symmetric.
\end{enumerate}
\end{example}

\begin{problem}
Determine whether the relation given in Example~\ref{ex:digraph} is reflexive, symmetric, or transitive.
\end{problem}

\begin{problem}
Given a relation $\sim$ on a finite set $A$, describe what each of reflexive, symmetric, and transitive look like in terms of a digraph. That is, draw a picture that represents reflexive, symmetric, and transitive. One thing to keep in mind is that the elements used in the definitions of symmetric and transitive do not have to be distinct.  So, you might need to consider multiple cases.
\end{problem}

\begin{problem}
Let $P$ be the set of people at a party and define $N$ via $(x,y)\in N$ if and only if $x$ knows the name of $y$.  Describe what it would mean for $N$ to be reflexive, symmetric, and transitive.
\end{problem}

\begin{problem}\label{prob:lots of relations}
Determine whether each of the following relations is reflexive, symmetric, or transitive.
\begin{enumerate}[label=\textrm{(\alph*)}]
\item Let $P$ denote the set of all people with accounts on Facebook and define $F$ on $P$ via $xFy$ if and only if $x$ is friends with $y$. 
\item\label{prob:twitter} Let $P$ denote the set of all people with accounts on Twitter and define $T$ on $P$ via $xTy$ if and only if $x$ follows $y$ on Twitter. 
\item Let $P$ be the set of all people and define $H$ via $xHy$ if and only if $x$ and $y$ have the same height.
\item Let $P$ be the set of all people and define $T$ via $xTy$ if and only if $x$ is taller than $y$.
\item Consider the relation ``divides" on $\mathbb{N}$.
\item Let $L$ be the set of lines and define $||$ via $l_1||l_2$ if and only if $l_1$ is parallel to $l_2$.
\item Let $C[0,1]$ be the set of continuous functions on $[0,1]$.  Define $f\sim g$ if and only if
\[
\int_0^1|f(x)|\ dx=\int_0^1|g(x)|\ dx.
\]
\item Define $\sim$ on $\mathbb{N}$ via $n\sim m$ if and only if $n+m$ is even.
\item Define $D$ on $\mathbb{R}$ via $(x,y)\in D$ if and only if $x=2y$.
\item\label{prob:mod 5} Define $\sim$ on $\mathbb{Z}$ via $a\sim b$ if and only if $a-b$ is a multiple of 5.
\item Define $\sim$ on $\mathbb{R}^2$ via $(x_1,y_1)\sim (x_2,y_2)$ if and only if $x_1^2+y_1^2=x_2^2+y_2^2$.
\item Define $\sim$ on $\mathbb{R}$ via $x\sim y$ if and only if $\lfloor x\rfloor =\lfloor y\rfloor$, where $\lfloor x\rfloor$ is the greatest integer less than or equal to $x$ (e.g., $\lfloor \pi\rfloor=3$, $\lfloor -1.5\rfloor=-2$, and $\lfloor 4\rfloor=4$).
\item Define $\sim$ on $\mathbb{R}$ via $x \sim y$ if and only if $|x-y|<1$.
\item Consider the empty relation on the set $A=\{a,b,c\}$.
\end{enumerate}
\end{problem}

\begin{definition}\label{def:relatives}
Let $\sim$ be a relation on a set $A$. For each $x\in A$, we define the \textbf{set of relatives of $x$ with respect to $\sim$} via
\[
[x]_{\sim}=\{y\in A\mid x\sim y\}.
\]
We also define
\[
\Omega_{\sim}=\{[x]\mid x\in A\}.
\]
\end{definition}

If $\sim$ is clear from the context, we will often write $[x]$ in place of $[x]_{\sim}$.  In terms of digraphs, $[x]$ is the collection of vertices that have arrows pointing towards them from the vertex labeled by $x$. Notice that $\Omega_{\sim}$ is a set of sets.  In particular, an element in $\Omega_{\sim}$ is a subset of $A$---equivalently, an element of $\mathcal{P}(A)$.

\begin{problem}
Consider the relation given in Example~\ref{ex:digraph}. Find $\Omega_R$ by determining $[x]$ for each $x\in A$.
\end{problem}

\begin{problem}
Let $P$ and $F$ be as in part~\ref{prob:facebook} of Problem~\ref{prob:lots of relations}.  Describe $[\text{Bob}]$, where Bob is the name of a specific Facebook user.  What is $\Omega_F$?
\end{problem}

\begin{problem}
Let $P$ and $T$ be as in part~\ref{prob:twitter} of Problem~\ref{prob:lots of relations}.  Describe $[\text{Maria}]$, where Maria is the name of a specific Twitter user.  What is $\Omega_T$?
\end{problem}

\begin{problem}
Consider the relation $\leq$ on $\mathbb{R}$.  If $x\in \mathbb{R}$, what is $[x]$?
\end{problem}

\begin{problem}\label{prob:mod5classes}
Find $[1]$ and $[2]$ for the relation given in part~\ref{prob:mod 5} of Problem~\ref{prob:lots of relations}.  How many different sets of relatives are there?  What are they?
\end{problem}

\end{section}