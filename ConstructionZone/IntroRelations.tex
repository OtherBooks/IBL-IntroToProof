\begin{section}{Introduction to Relations}

ADD INTRO BLURB HERE...

\begin{definition}
Let $A$ and $B$ be sets. A \textbf{relation $\sim$ from a set $A$ to a set $B$} is a subset of $A \times B$. If $\sim$ is a relation from $A$ to $A$, then we say that $\sim$ is a \textbf{relation on $A$}.  
\end{definition}

\begin{example}
The set $\mathbb{N}\times \mathbb{R}$ from Problem~\ref{prob:some lines} is an example of a relation on $\mathbb{R}$ since $\mathbb{N}\times \mathbb{R}$ is a subset of $\mathbb{R}\times \mathbb{R}$.
\end{example}

Different notations for relations are used in different contexts.  If $\sim$ is a relation from $A$ to $B$ and $(a,b)\in {\sim}$, then we say that \textbf{$a$ is related $b$}.  In this case, we may also write $a\sim b$.  It's important to notice that order matters here.  That is, if $a\sim b$, it may or may not be the case that $b\sim a$.

\begin{example}
You are already familiar with many relations.  For example, $=$, $\leq$, and $<$ are each examples of relations on the real numbers. We could say that $(3,\pi)$ is in the relation $\leq$ and the relation $<$ since $3\leq \pi$ and $3<\pi$.  However, $(3,\pi)$ is not in the relation $=$ since $3\neq \pi$.  Also, notice that order matters for the relation $\leq$ yet does not for $=$. For example, $(-\sqrt{2}, 4)$ is in the relation $\leq$ while $(4,-\sqrt{2})$ is not.
\end{example}

\begin{example}
Let $P$ denote the set of all people and let $F$ denote the set of all possible first names.  Define the relation $N$ from $P$ to $F$ via
\[
N:=\{(p,f)\mid p\text{ has first name }f\}.
\]
That is, $pNf$ if and only if $p$ has first name $f$.  Then $N$ is a relation from $P$ to $F$.
\end{example}

\begin{example}
Let $P$ denote the set of all people. Define $\mapsto$ on $P$ via $x\mapsto y$ if and only if $x$ is friends with $y$.  Then $\mapsto$ is a relation on $P$.
\end{example}

\begin{example}
Define $S$ from $\{-1,1\}$ to $\mathbb{Z}$ via $1Sx$ if and only if $x$ is even and $-1Sx$ if and only if $x$ is odd.  That is, $1$ is related to all even integers and $-1$ is related to all odd integers.  Then $S$ is a relation from $\{-1,1\}$ to $\mathbb{Z}$.
\end{example}

\begin{example}
Let $A$ be any set.  Since $\emptyset \subseteq A\times A$, the empty set forms a relation on $A$. This relation is called the \textbf{empty relation} on $A$.
\end{example}

We can often represent relations using digraphs.  Given a finite set $A$ and a relation $\sim$ on $A$, a \textbf{digraph} (short for \emph{directed graph}) is a discrete graph having the members of $A$ as vertices and a directed edge from $a$ to $b$ if and only if $a\sim b$.

\begin{example}\label{ex:digraph}
Figure~\ref{fig:digraph} depicts a digraph that represents a relation $\sim$ on $A=\{a,b,c,d,e\}$ given by
\[
{\sim}=\{(a,a),(a,b),(a,c),(b,b),(b,a),(b,c),(c,d),(c,e),(d,d),(d,a),(d,c),(e,a)\}.
\]

\begin{figure}[h!]
\begin{center}
\begin{tikzpicture}[->,>=stealth',shorten >=1pt,auto,node distance=2cm,semithick]
  \tikzstyle{every state}=[]

  \node[state] (A)                    {$a$};
  \node[state]         (B) [above right of=A] {$b$};
  \node[state]         (D) [below right of=A] {$d$};
  \node[state]         (C) [below right of=B] {$c$};
  \node[state]         (E) [below of=D]       {$e$};

  \path (B) edge [bend left]  node {} (A)
        (A) edge [bend left]  node {} (B)
            edge              node {} (C)
        (B) edge [loop above] node {} (B)
            edge [bend left]  node {} (C)
        (D) edge [bend left]  node {} (C)
        (C) edge [bend left]  node {} (D)
            edge [bend left]  node {} (E)
        (A) edge [loop left]  node {} (A)
        (D) edge [loop below] node {} (D)
            edge [bend left]  node {} (A)
        (E) edge [bend left]  node {} (A);
\end{tikzpicture}
\caption{An example of a digraph for a relation.}\label{fig:digraph}
\end{center}
\end{figure}

\end{example}

\begin{problem}
Let $A=\{a,b,c\}$ and define ${\sim}=\{(a,a),(a,b),(b,c),(c,b),(c,a)\}$.  
\begin{enumerate}[label=\textrm{(\alph*)}]
\item Draw the digraph for $\sim$.
\item Draw the digraph for the empty relation on $A$.
\end{enumerate}
\end{problem}

\begin{problem}
Let $A=\{1,2,3,4,5,6\}$ and define $|$ on $A$ via $x|y$ if and only if $x$ divides $y$.  Draw the digraph for $|$ on $A$.
\end{problem}

When $A$ or $B$ is infinite, it is not practical to draw a digraph.  However, you are familiar with the graphs of some relations involving infinite sets.

\begin{example}
When we write $x^2+y^2=1$, we are implicitly defining a relation.  In particular, the relation is the set of ordered pairs $(x,y)$ satisfying $x^2+y^2=1$, namely $\{(x,y)\mid x^2+y^2=1\}$. The graph of this relation in $\mathbb{R}^2$ is the unit circle centered at the origin.
\end{example}

\begin{problem}
Consider the relation $\leq$ on $\mathbb{R}$. Draw a picture of this relation in $\mathbb{R}^2$. In other words, identify all points $(x,y)$ where $x\leq y$.
\end{problem}

\begin{definition}
Let $\sim$ be a relation on a set $A$.
\begin{enumerate}[label=\textrm{(\alph*)}]
\item The relation $\sim$ is \textbf{reflexive} if for all $a\in A$, $a\sim a$ (every element is related to itself).
\item The relation $\sim$ is \textbf{symmetric} if for all $a,b\in A$, if $a\sim b$, then $b\sim a$.
\item The relation $\sim$ is \textbf{transitive} if for all $a,b,c\in A$, if $a\sim b$ and $b\sim c$, then $a\sim c$.
\end{enumerate}
\end{definition}

\begin{example}
Here are a few examples that illustrate the concepts in the previous definition.
\begin{enumerate}[label=\textrm{(\alph*)}]
\item The relation $=$ on $\mathbb{R}$ is reflexive, symmetric, and transitive.
\item The relation $\leq$ is reflexive and transitive on $\mathbb{R}$, but not symmetric. However, notice that $<$ is transitive on $\mathbb{R}$, but neither symmetric nor reflexive.
\item If $S$ is a set, then $\subseteq$ on $\mathcal{P}(S)$ is reflexive and transitive, but not symmetric.
\end{enumerate}
\end{example}

\begin{problem}
Determine whether the relation given in Example~\ref{ex:digraph} is reflexive, symmetric, or transitive.
\end{problem}

ADD OTHER EXAMPLES ABOVE TO PREVIOUS PROBLEM???

\begin{problem}
Given a relation $\sim$ on a finite set $A$, describe what each of reflexive, symmetric, and transitive look like in terms of a digraph. That is, draw a picture that represents reflexive, symmetric, and transitive. One thing to keep in mind is that the elements used in the definitions of symmetric and transitive do not have to be distinct.  So, you might need to consider multiple cases.
\end{problem}

\begin{problem}
Let $P$ be the set of people at a party and define $N$ via $(x,y)\in N$ if and only if $x$ knows the name of $y$.  Describe what it would mean for $N$ to be reflexive, symmetric, and transitive.
\end{problem}

Below, we provide skeleton proofs for proving that a relation is reflexive, symmetric, or transitive.  Notice that the skeleton proof for proving that a relation is reflexive is a special case of Skeleton Proof~\ref{skeleton:for all}. Similarly, the skeleton proofs involving symmetric and transitive are both special cases of Skeleton Proof~\ref{skeleton:for all direct proof}. It is important to point out that every relation on the empty set is vacuously reflexive, symmetric, and transitive.  In the skeleton proofs below, we are implicitly assuming that the set in question is nonempty.  In some circumstances, it may be necessary to mention the possibility of the empty set.

\begin{skeleton}[Proof that a relation is reflexive]
Here is the general structure for proving that a relation is reflexive. 
\begin{center}
\framebox{
\begin{minipage}{6in}
\vspace{.1in}
\begin{proof}
Assume $\sim$ is a relation on $A$.  Let $a\in A$.
\begin{center}
$\ldots$ \ \emph{[Use the definition of $\sim$ to verify that $a\sim a$]} \ $\ldots$\\
\end{center}
\noindent Therefore, the relation $\sim$ is reflexive on $A$.
\end{proof}
\end{minipage}
}
\end{center}
\end{skeleton}

\begin{skeleton}[Proof that a relation is symmetric]
Here is the general structure for proving that a relation is symmetric.
\begin{center}
\framebox{
\begin{minipage}{6in}
\vspace{.1in}
\begin{proof}
Assume $\sim$ is a relation on $A$.  Let $a, b\in A$ and suppose $a\sim b$.
\begin{center}
$\ldots$ \ \emph{[Use assumption that $a\sim b$ with definition of $\sim$ to verify that $b\sim a$]} \ $\ldots$\\
\end{center}
\noindent Therefore, the relation $\sim$ is symmetric on $A$.
\end{proof}
\end{minipage}
}
\end{center}
\end{skeleton}

\begin{skeleton}[Proof that a relation is transitive]
Here is the general structure for proving that a relation is transitiv.
\begin{center}
\framebox{
\begin{minipage}{6in}
\vspace{.1in}
\begin{proof}
Assume $\sim$ is a relation on $A$.  Let $a, b, c\in A$ and suppose $a\sim b$ and $b\sim c$.
\begin{center}
$\ldots$ \ \emph{[Use assumption that $a\sim b$ and $b\sim c$ with definition of $\sim$ to verify that $a\sim c$]} \ $\ldots$\\
\end{center}
\noindent Therefore, the relation $\sim$ is transitive on $A$.
\end{proof}
\end{minipage}
}
\end{center}
\end{skeleton}



\begin{problem}\label{prob:lots of relations}
Determine whether each of the following relations is reflexive, symmetric, or transitive. In each case, you should either provide a specific counterexample or a proof.
\begin{enumerate}[label=\textrm{(\alph*)}]
\item\label{prob:facebook} Let $P$ denote the set of all people with accounts on Facebook and define $F$ on $P$ via $xFy$ if and only if $x$ is friends with $y$. 
\item\label{prob:twitter} Let $P$ denote the set of all people with accounts on Twitter and define $T$ on $P$ via $xTy$ if and only if $x$ follows $y$ on Twitter. 
\item Let $P$ be the set of all people and define $H$ via $xHy$ if and only if $x$ and $y$ have the same height.
\item Let $P$ be the set of all people and define $T$ via $xTy$ if and only if $x$ is taller than $y$.
\item Consider the relation ``divides" on $\mathbb{N}$.
\item Let $L$ be the set of lines and define $||$ via $l_1||l_2$ if and only if $l_1$ is parallel to $l_2$.
\item Let $C[0,1]$ be the set of continuous functions on $[0,1]$.  Define $f\sim g$ if and only if
\[
\int_0^1|f(x)|\ dx=\int_0^1|g(x)|\ dx.
\]
\item Define $\sim$ on $\mathbb{N}$ via $n\sim m$ if and only if $n+m$ is even.
\item Define $D$ on $\mathbb{R}$ via $(x,y)\in D$ if and only if $x=2y$.
\item Define $F$ on $\mathbb{Z}\times \left(\mathbb{Z}\setminus \{0\}\right)$ via $(a,b)F(c,d)$ if and only $\frac{a}{b}=\frac{c}{d}$.
\item\label{prob:mod 5} Define $\sim$ on $\mathbb{Z}$ via $a\sim b$ if and only if $a-b$ is a multiple of 5.
\item Define $\sim$ on $\mathbb{R}^2$ via $(x_1,y_1)\sim (x_2,y_2)$ if and only if $x_1^2+y_1^2=x_2^2+y_2^2$.
\item Define $\sim$ on $\mathbb{R}$ via $x\sim y$ if and only if $\lfloor x\rfloor =\lfloor y\rfloor$, where $\lfloor x\rfloor$ is the greatest integer less than or equal to $x$ (e.g., $\lfloor \pi\rfloor=3$, $\lfloor -1.5\rfloor=-2$, and $\lfloor 4\rfloor=4$).
\item Define $\sim$ on $\mathbb{R}$ via $x \sim y$ if and only if $|x-y|<1$.
\item Consider the empty relation on the set $A=\{a,b,c\}$.
\end{enumerate}
\end{problem}

CHANGE NOTATION FOR SET OF RELATIVES???

\begin{definition}\label{def:relatives}
Let $\sim$ be a relation on a set $X$. For each $x\in X$, we define the \textbf{set of relatives of $x$ with respect to $\sim$} via
\[
[x]_{\sim}=\{y\in X\mid x\sim y\}.
\]
We also define
\[
\Omega_{\sim}=\{[x]\mid x\in X\}.
\]
\end{definition}

If $\sim$ is clear from the context, we will often write $[x]$ in place of $[x]_{\sim}$.  In terms of digraphs, $[x]$ is the collection of vertices that have arrows pointing towards them from the vertex labeled by $x$. Notice that $\Omega_{\sim}$ is a set of sets.  In particular, an element in $\Omega_{\sim}$ is a subset of $X$---equivalently, an element of $\mathcal{P}(X)$.

ADD MORE PROBLEMS THAT CORRESPOND TO THE EXAMPLES I ADDED???

\begin{problem}
Consider the relation given in Example~\ref{ex:digraph}. Find $\Omega_{\sim}$ by determining $[x]$ for each $x\in A$.
\end{problem}

\begin{problem}
Let $P$ and $F$ be as in part~\ref{prob:facebook} of Problem~\ref{prob:lots of relations}.  Describe $[\text{Bob}]$, where Bob is the name of a specific Facebook user.  What is $\Omega_F$?
\end{problem}

\begin{problem}
Let $P$ and $T$ be as in part~\ref{prob:twitter} of Problem~\ref{prob:lots of relations}.  Describe $[\text{Maria}]$, where Maria is the name of a specific Twitter user.  What is $\Omega_T$?
\end{problem}

\begin{problem}
Consider the relation $\leq$ on $\mathbb{R}$.  If $x\in \mathbb{R}$, what is $[x]$?
\end{problem}

\begin{problem}\label{prob:mod5classes}
Find $[1]$ and $[2]$ for the relation given in part~\ref{prob:mod 5} of Problem~\ref{prob:lots of relations}.  List all of the distinct sets of relatives for this relation.
\end{problem}

\begin{problem}\label{prob:find sim from Omega}
Suppose $\sim$ is a relation on $A=\{1,2,3,4,5\}$ such that $[1]=\{1,3,4\}$, $[2]=\{4\}$, $[3]=\{3,4,5\}$, $[4]=\{1,2\}$, and $[5]=\emptyset$. List the ordered pairs in $\sim$ and draw the corresponding digraph.
\end{problem}

\end{section}