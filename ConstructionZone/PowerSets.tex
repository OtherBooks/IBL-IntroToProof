\begin{section}{Power Sets}\label{sec:PowerSets}

We've already seen that using union, intersection, set difference, and complement we can create new sets (in the same universe) from existing sets.  In this section, we will describe another way to generate new sets; however, the new sets will not ``live" in the same universe this time.

\begin{definition}
If $S$ is a set, then the \textbf{power set} of $S$ is the set of subsets of $S$.  The power set of $S$ is denoted $\boxed{\mathcal{P}(S)}$.
\end{definition}

It follows immediately from the definition that $A\subseteq S$ if and only if $A\in\mathcal{P}(S)$.  For example, if $S=\{a,b\}$, then $\mathcal{P}=\{\emptyset, \{a\}, \{b\}, S\}$.

\begin{problem}\label{prob:PowerSets}
For each of the following sets, find the power set.
\begin{multicols}{2}
\begin{enumerate}[label=\textrm{(\alph*)}]
\item $A=\{\circ, \triangle, \square\}$
\item $B=\{a,\{a\}\}$
\item $C=\emptyset$
\item $D=\{\emptyset\}$
\end{enumerate}
\end{multicols}
\end{problem}

\begin{problem}\label{conjecture:PowerSets}
How many subsets do you think that a set with $n$ elements has?  What if $n=0$?  You do not need to prove your conjecture at this time.  We will prove this later using mathematical induction.
\end{problem}

It is important to realize that the concepts of \emph{element} and \emph{subset} need to be carefully delineated.  For example, consider the set $A=\{x,y\}$.  The object $x$ is an element of $A$, but the object $\{x\}$ is both a subset of $A$ and an element of $\mathcal{P}(A)$.  This can get confusing rather quickly.  Consider the set $B$ from Problem~\ref{prob:PowerSets}.  The set $\{a\}$ happens to be an element of $B$, a subset of $B$, and an element of  $\mathcal{P}(B)$. The upshot is that it is important to pay close attention to whether ``$\subseteq$" or ``$\in$" is the proper symbol to use.

\begin{theorem}
Let $S$ and $T$ be sets.  Then $S\subseteq T$ if and only if $\mathcal{P}(S)\subseteq \mathcal{P}(T)$.\footnote{To prove this theorem, you have to write two distinct subproofs: $A\implies B$ and $B\implies A$.}
\end{theorem}

\begin{theorem}
Let $S$ and $T$ be sets.  Then $\mathcal{P}(S)\cap\mathcal{P}(T)=\mathcal{P}(S\cap T)$.
\end{theorem}

\begin{theorem}\label{thm:UnionPowerSets}
Let $S$ and $T$ be sets.  Then $\mathcal{P}(S)\cup\mathcal{P}(T)\subseteq \mathcal{P}(S\cup T)$.
\end{theorem}

\begin{problem}
Provide a counterexample to show that it is not necessarily true that $\mathcal{P}(S)\cup\mathcal{P}(T)= \mathcal{P}(S\cup T)$. This verifies that the converse of Theorem~\ref{thm:UnionPowerSets} is not true in general. Is it ever true that $\mathcal{P}(S)\cup\mathcal{P}(T)$ and  $\mathcal{P}(S\cup T)$ are equal?
\end{problem}

\end{section}