\chapter{Introduction}\label{chap:intro}

\epigraphhead[70pt]{
\epigraph{The mathematician does not study pure mathematics because it is useful; he studies it because he delights in it, and he delights in it because it is beautiful.}{Henri Poincar\'e, mathematician \& physicist}}

\begin{section}{What is This Book All About?}

This book is intended to be used for a one-semester/quarter introduction to proof course (sometimes referred to as a transition to proof course). The purpose of this book is to introduce the reader to the process of constructing and writing formal and rigorous mathematical proofs. The intended audience is mathematics majors and minors. However, this book is also appropriate for anyone curious about mathematics and writing proofs. Most users of this book will have taken at least one semester of calculus, although other than some familiarity with a few standard functions in Chapter~\ref{chap:Functions}, content knowledge of calculus is not required. The book includes more content than one can expect to cover in a single semester/quarter. This allows the instructor/reader to pick and choose the sections that suit their needs and desires. Each chapter takes a focused approach to the included topics, but also includes many gentle exercises aimed at developing intuition.

The following sections form the core of the book and are likely the sections that an instructor would focus on in a one-semester introduction to proof course.
\begin{itemize}
\item Chapter~\ref{chap:IntroToMath}: Mathematics and Logic. All sections.
\item Chapter~\ref{chap:SetTheory}: Set Theory. Sections 3.1, 3.3, 3.4, and 3.5.
\item Chapter~\ref{chap:Induction}: Induction. All sections.
\item Chapter~\ref{chap:Relations Partitions}: Relations and Partitions. Sections 7.1, 7.2, and 7.3.
\item Chapter~\ref{chap:Functions}: Functions. Sections 8.1, 8.2, 8.3, and 8.4.
\item Chapter~\ref{chap:Cardinality}: Cardinality. All sections.
\end{itemize}
Time permitting, instructors can pick and choose topics from the remaining sections.  I typically cover the core sections listed above together with Chapter~\ref{chap:ThreeFamousTheorems}: Three Famous Theorems during a single semester. The \emph{Instructor Guide} contains examples of a few possible paths through the material, as well as information about which sections and theorems depend on material earlier in the book.

\epigraph{Mathematics, rightly viewed, possesses not only truth, but supreme beauty---a beauty cold and austere, like that of sculpture, without appeal to any part of our weaker nature, without the gorgeous trappings of painting or music, yet sublimely pure, and capable of a stern perfection such as only the greatest art can show. The true spirit of delight, the exaltation, the sense of being more than Man, which is the touchstone of the highest excellence, is to be found in mathematics as surely as poetry.}{Bertrand Russell, philosopher \& mathematician}

\end{section}

\begin{section}{What Should You Expect?}\label{sec:what should you expect}

Up to this point, it is possible that your experience of mathematics has been about using formulas and algorithms. You are used to being asked to do things like: ``solve for x", ``take the derivative of this function", ``integrate this function", etc. Accomplishing tasks like these usually amounts to mimicking examples that you have seen in class or in your textbook. However, this is only one part of mathematics.  Mathematicians experiment, make conjectures, write definitions, and prove theorems.  While engaging with the material contained in this book, we will learn about doing all of these things, especially writing proofs. Mathematicians are in the business of proving theorems and this is exactly our endeavor. Ultimately, the focus of this book is on producing and discovering mathematics.

Your progress will be fueled by your ability to wrestle with mathematical ideas and to prove theorems.  As you work through the book, you will find that you have ideas for proofs, but you are unsure of them.  Do not be afraid to tinker and make mistakes.  You can always revisit your work as you become more proficient. Do not expect to do most things perfectly on your first---or even second or third---attempt. The material is too rich for a human being to completely understand immediately. Learning a new skill requires dedication and patience during periods of frustration. Moreover, solving genuine problems is difficult and takes time. But it is also rewarding! 

%As a reader of this textbook, you have the right to:
%\begin{enumerate}
%\item be confused,
%\item make a mistake and to revise your thinking,
%\item speak, listen, and be heard, and
%\item enjoy doing mathematics.
%\end{enumerate}

\epigraph{You may encounter many defeats, but you must not be defeated.}{Maya Angelou, poet \& activist}
	
\end{section}

\begin{section}{An Inquiry-Based Approach}

In many mathematics classrooms, ``doing mathematics" means following the rules dictated by the teacher, and ``knowing mathematics" means remembering and applying them. However, this is not a typical mathematics textbook and is likely a significant departure from your prior experience, where mimicking prefabricated examples led you to success. In order to promote a more active participation in your learning, this book adheres to an educational philosophy called inquiry-based learning (IBL). IBL is a student-centered method of teaching that engages students in sense-making activities and challenges them to create or discover mathematics.  In this book, you will be expected to actively engage with the topics at hand and to construct your own understanding.  You will be given tasks requiring you to solve problems, conjecture, experiment, explore, create, and communicate.  Rather than showing facts or a clear, smooth path to a solution, this book will guide and mentor you through an adventure in mathematical discovery. 

This book makes no assumptions about the specifics of how your instructor chooses to implement an IBL approach. Generally speaking, students are told which problems and theorems to grapple with for the next class sessions, and then the majority of class time is devoted to students working in groups on unresolved solutions/proofs or having students present their proposed solutions/proofs to the rest of the class. Students should---as much as possible---be responsible for guiding the acquisition of knowledge and validating the ideas presented. That is, you should not be looking to the instructor as the sole authority. In an IBL course, instructor and students have joint responsibility for the depth and progress of the course. While effective IBL courses come in a variety of forms, they all possess a few essential ingredients. According to \href{https://www.colorado.edu/eer/sites/default/files/attached-files/laursenrasmussencommentaryauthorversion0219.pdf}{Laursen and Rasmussen (2019)}, the Four Pillars of IBL are:
\begin{itemize}
\item Students engage deeply with coherent and meaningful mathematical tasks.
\item Students collaboratively process mathematical ideas.
\item Instructors inquire into student thinking.
\item Instructors foster equity in their design and facilitation choices.
\end{itemize}
This book can only address the first pillar while it is the responsibility of your instructor and class to develop a culture that provides an adequate environment for the remaining pillars to take root.  If you are studying this material independent of a classroom setting, I encourage you to find a community where you can collaborate and discuss your ideas.

Just like learning to play an instrument or sport, you will have to learn new skills and ideas. Along this journey, you should expect a cycle of victory and defeat, experiencing a full range of emotions.  Sometimes you will feel exhilarated, other times you might be seemingly paralyzed by extreme confusion. You will experience struggle and failure before you experience understanding. This is part of the normal learning process. If you are doing things well, you should be confused on a regular basis. Productive struggle and mistakes provide opportunities for growth.  As the author of this text, I am here to guide and challenge you, but I cannot do the learning for you, just as a music teacher cannot move your fingers and your heart for you. This is a very exciting time in your mathematical career.  You will experience mathematics in a new and profound way. Be patient with yourself and others as you adjust to a new paradigm.

You could view this book as mountaineering guidebook.  I have provided a list of mountains to summit, sometimes indicating which trailhead to start at or which trail to follow.  There will always be multiple routes to top, some more challenging than others. Some summits you will attain quickly and easily, others might require a multi-day expedition.  Oftentimes, your journey will be laced with false summits.  Some summits will be obscured by clouds.  Sometimes you will have to wait out a storm, perhaps turning around and attempting another route, or even attempting to summit on a different day after the weather has cleared. The strength, fitness, and endurance you gain along the way will allow you to take on more and more challenging, and often beautiful, terrain. Do not forget to take in the view from the top! The joy you feel from overcoming obstacles and reaching each summit under your own will and power has the potential to be life changing.  But make no mistake, the journey is vastly more important than the destinations. 

\epigraph{Don't fear failure.  Not failure, but low aim, is the crime. In great attempts it is glorious even to fail.}{Bruce Lee, martial artist \& actor}

\end{section}

\begin{section}{Structure of the Textbook}

As you read this book, you will be required to digest the material in a meaningful way.  It is your responsibility to read and understand new definitions and their related concepts.  In addition, you will be asked to complete problems aimed at solidifying your understanding of the material.  Most importantly, you will be asked to make conjectures, produce counterexamples, and prove theorems. All of these tasks will almost always be challenging.

The items labeled as \textbf{Definition} and \textbf{Example} are meant to be read and digested.  However, the items labeled as \textbf{Problem}, \textbf{Theorem}, and \textbf{Corollary} require action on your part.  Items labeled as \textbf{Problem} are sort of a mixed bag. Some Problems are computational in nature and aimed at improving your understanding of a particular concept while others ask you to provide a counterexample for a statement if it is false or to provide a proof if the statement is true. Items with the \textbf{Theorem} and \textbf{Corollary} designation are mathematical facts and the intention is for you to produce a valid proof of the given statement. The main difference between a theorem and a corollary is that corollaries are typically statements that follow quickly from a previous theorem.  In general, you should expect corollaries to have very short proofs.  However, that does not mean that you cannot produce a more lengthy yet valid proof of a corollary.

Oftentimes, the problems and theorems are guiding you towards a substantial, more  general result. Other times, they are designed to get you to apply ideas in a new way. One thing to always keep in mind is that every task in this book can be done by you, the student. But it may not be on your first try, or even your second.

Discussion of new topics is typically kept at a minimum and there are very few examples in this book. This is intentional.  One of the objectives of the items labeled as \textbf{Problem} is for you to produce the examples needed to internalize unfamiliar concepts.  The overarching goal of this book is to help you develop a deep and meaningful understanding of the processes of producing mathematics by putting you in direct contact with mathematical phenomena.

\epigraph{Don't just read it; fight it! Ask your own questions, look for your own examples, discover your own proofs. Is the hypothesis necessary? Is the converse true? What happens in the classical special case? What about the degenerate cases? Where does the proof use the hypothesis?}{Paul Halmos, mathematician}

\end{section}

\begin{section}{Some Minimal Guidance}\label{sec:guidance}
Especially in the opening sections, it will not be clear what facts from your prior experience in mathematics you are ``allowed" to use.  Unfortunately, addressing this issue is difficult and is something we will sort out along the way.  In addition, you are likely unfamiliar with how to structure a valid mathematical proof.  So that you do not feel completely abandoned, here are some guidelines to keep in mind as you get started with writing proofs.

\begin{itemize}
\item The statement you are proving should be on the same page as the beginning of your proof.   
\item You should indicate where the proof begins by writing ``\emph{Proof.}" at the beginning.  
\item Make it clear to yourself and the reader what your assumptions are at the very beginning of your proof. Typically, these statements will start off ``Assume\ldots", ``Suppose\ldots", or ``Let\ldots".  Sometimes there will be some implicit assumptions that we can omit, but at least in the beginning, you should get in the habit of clearly stating your assumptions up front. 
\item Carefully consider the order in which you write your proof. Each sentence should follow from an earlier sentence in your proof or possibly a result you have already proved.
\item Unlike the experience many of you had writing proofs in your high school geometry class, our proofs should be written in complete sentences.  You should break sections of a proof into paragraphs and use proper grammar.  There are some pedantic conventions for doing this that will be pointed out along the way.  Initially, this will be an issue that you may struggle with, but you will get the hang of it.
\item There will be many situations where you will want to refer to an earlier definition, problem, theorem, or corollary.  In this case, you should reference the statement by number, but it is also helpful to the reader to summarize the statement you are citing.  For example, you might write something like, ``According to Theorem~\ref{thm:two consecutive ints}, the sum of two consecutive integers is odd, and so\ldots" or ``By the definition of divides (Definition~\ref{def:divides}), it follows that\ldots". One thing worth pointing out is that if we are citing a definition, theorem, or problem by number, we should capitalize ``Definition", ``Theorem", or ``Problem", respectively (e.g., ``According to Theorem~\ref{thm:two consecutive ints}\ldots"). Otherwise, we do not capitalize these words (e.g., ``By the definition of divides\ldots").
\item There will be times when we will need to do some basic algebraic manipulations.  You should feel free to do this whenever the need arises.  But you should show sufficient work along the way.  In addition, you should organize your calculations so that each step follows from the previous.  The order in which we write things matters. You do not need to write down justifications for basic algebraic manipulations (e.g., adding 1 to both sides of an equation, adding and subtracting the same amount on the same side of an equation, adding like terms, factoring, basic simplification, etc.).  
\item On the other hand, you do need to make explicit justification of the logical steps in a proof.  As stated above, you should cite a previous definition, theorem, etc. when necessary.
\item Similar to making it clear where your proof begins, you should indicate where it ends.  It is common to conclude a proof with the standard ``proof box" ($\square$ or $\blacksquare$).  This little square at end of a proof is sometimes called a \textbf{tombstone} or \textbf{Halmos symbol} after Hungarian-born American mathematician \href{https://en.wikipedia.org/wiki/Paul_Halmos}{Paul Halmos} (1916--2006).
\end{itemize}

It is of utmost importance that you work to understand every proof.  Questions---asked to your instructor, your peers, and yourself---are often your best tool for determining whether you understand a proof.  Another way to help you process and understand a proof is to try and make observations and connections between different ideas, proof statements and methods, and to compare various approaches. 
%Observations might sound like some of the following:
%\begin{itemize}
%\item When I tried this proof, I thought I needed to \ldots But I didn't need that, because \ldots
%\item I think that \ldots is important to this proof, because \ldots
%\item When I read the statement of this theorem, it seemed similar to this earlier theorem. Now I see that it [is/isn't] because \ldots
%\end{itemize}

If you would like additional guidance before you dig in, look over the guidelines in Appendix~\ref{appendix:elements_of_style}: Elements of Style for Proofs. It is suggested that you review this appendix occasionally as you progress through the book as some guidelines may not initially make sense or seem relevant.  Be prepared to put in a lot of time and do all the work. Your effort will pay off in intellectual development. Now, go have fun and start exploring mathematics!

\epigraph{Our greatest glory is not in never falling, but in rising every time we fall.}{Confucius, philosopher}

\end{section}