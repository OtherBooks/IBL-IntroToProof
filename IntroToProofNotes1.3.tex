\documentclass[11pt]{article}

\usepackage{amsfonts}
\usepackage{amsmath}
\usepackage{amssymb}
\usepackage{stmaryrd}
\usepackage{amsthm}
\usepackage{fancyhdr}
\usepackage[margin=1in]{geometry}
\usepackage[hang,flushmargin,symbol*]{footmisc}
\usepackage{color}
\definecolor{darkblue}{rgb}{0, 0, .6}
\definecolor{grey}{rgb}{.7, .7, .7}
\usepackage[breaklinks]{hyperref}
\hypersetup{
	colorlinks=true,
	linkcolor=darkblue,
	anchorcolor=darkblue,
	citecolor=darkblue,
	pagecolor=darkblue,
	urlcolor=darkblue,
	pdftitle={},
	pdfauthor={}
}

\pagestyle{fancy}

\lhead{\scriptsize Notes for an Introduction to Proof Course (Version Spring 2013)}
\rhead{\scriptsize Instructor: \href{http://danaernst.com}{D.C. Ernst}} 
\lfoot{\scriptsize This work is an adaptation of notes written by Stan Yoshinobu of Cal Poly and Matthew Jones of California State University, Dominguez Hills.} 
\cfoot{} 
\renewcommand{\headrulewidth}{0.4pt} 
\renewcommand{\footrulewidth}{0.4pt} 

\theoremstyle{definition}
\newtheorem{theorem}{Theorem}[section]
\newtheorem{acknowledgement}[theorem]{Acknowledgement}
\newtheorem{algorithm}[theorem]{Algorithm}
\newtheorem{question}[theorem]{Question}
\newtheorem{axiom}[theorem]{Axiom}
\newtheorem{case}[theorem]{Case}
\newtheorem{claim}[theorem]{Claim}
\newtheorem{conclusion}[theorem]{Conclusion}
\newtheorem{condition}[theorem]{Condition}
\newtheorem{conjecture}[theorem]{Conjecture}
\newtheorem{corollary}[theorem]{Corollary}
\newtheorem{criterion}[theorem]{Criterion}
\newtheorem{definition}[theorem]{Definition}
\newtheorem{example}[theorem]{Example}
\newtheorem{exercise}[theorem]{Exercise}
\newtheorem{journal}[theorem]{Journal}
\newtheorem{lemma}[theorem]{Lemma}
\newtheorem{notation}[theorem]{Notation}
\newtheorem{problem}[theorem]{Problem}
\newtheorem{proposition}[theorem]{Proposition}
\newtheorem{remark}[theorem]{Remark}
\newtheorem{solution}[theorem]{Solution}
\newtheorem{summary}[theorem]{Summary}
\newtheorem{skeleton}[theorem]{Skeleton Proof}

\newsavebox{\savepar}
\newenvironment{textbox}{\noindent\begin{lrbox}{\savepar}\begin{minipage}[c]{.98\textwidth}}{\end{minipage}\end{lrbox}\fcolorbox{black}{white}{\usebox{\savepar}}}

\begin{document}

\addtocounter{section}{0}

\begin{section}{Introduction to Mathematics (Continued)}

\addtocounter{subsection}{2}
\addtocounter{theorem}{45}

\begin{subsection}{Negating Implications and Proof by Contradiction}

So far we have discussed how to negate propositions of the form $A$, $A\wedge B$, and $A\vee B$ for propositions $A$ and $B$.  However, we have yet to discuss how to negate propositions of the form $A\implies B$.

\begin{problem}\label{prob:implication as disjunction}
Let $A$ and $B$ be propositions.  Conjecture an equivalent way of expressing the conditional proposition $A\implies B$ as a proposition involving the disjunction symbol $\vee$ and possibly the negation symbol $\neg$, but not the implication symbol $\implies$.  Prove your conjecture using a truth table.
\end{problem}

\begin{exercise}\label{exer:Darth Vader}
Let $A$ and $B$ be the propositions ``Darth Vader is a hippie" and ``Sarah Palin is a liberal", respectively.  Using Problem \ref{prob:implication as disjunction}, express $A\implies B$ as an English sentence involving the disjunction ``or."
\end{exercise}

\begin{problem}[*]\label{prob:negation of implication}
Let $A$ and $B$ be two propositions.  Conjecture an equivalent way of expressing the proposition $\neg(A\implies B)$ as a proposition involving the conjunction symbol $\wedge$ and possibly the negation symbol $\neg$, but not the implication symbol $\implies$.  Prove your conjecture using previous results.
\end{problem}

\begin{exercise}
Let $A$ and $B$ be the propositions in Exercise \ref{exer:Darth Vader}.  Using Problem \ref{prob:negation of implication}, express $\neg(A\implies B)$ as an English sentence involving the conjunction ``and."
\end{exercise}

\begin{exercise}
The following proposition is \emph{false}.  Negate this proposition to obtain a true statement.  Write your statement as a conjunction.
\begin{quote}
If $.\overline{99}=\frac{9}{10}+\frac{9}{100}+\frac{9}{1000}+\cdots$, then $.\overline{99}\neq 1$.
\end{quote}
You do \emph{not} need to prove your new statement.
\end{exercise}

Recall that a proposition is exclusively either true or false.  That is, a proposition can never be both true and false.  This idea leads us to the next definition.

\begin{definition}
A compound proposition that is always false is called a \textbf{contradiction}.  A compound statement that is always true is called a \textbf{tautology}.
\end{definition}

\begin{theorem}
Let $A$ be a proposition. Then $\neg A\wedge A$ is a contradiction.
\end{theorem}

\begin{exercise}
Provide an example of a tautology using arbitrary positions and any of the logical connectives $\neg$, $\wedge$, and $\vee$.  Then prove that your example is in fact a tautology.
\end{exercise}

Suppose that we want to prove some proposition $P$ (which might be something like $A\implies B$ or possibly more complicated).  One approach, called \textbf{proof by contradiction}, involves assuming $\neg P$ and then logically deducing a contradiction of the form $Q\wedge \neg Q$, where $Q$ is some proposition (possibly equal to $P$).  Since this is absurd, it cannot be the case that $\neg P$ is true, which implies that $P$ is true.  The tricky part about a proof by contradiction is that it is not usually obvious what the statement $Q$ is going to be.  Here is what the general structure for a proof by contradiciton looks like.

\bigskip

\begin{skeleton}[Proof of $P$ by contradiction]
Here is what the general structure for a proof by contradiciton looks like if we are trying to prove the proposition $P$.

\bigskip

\begin{textbox}
\begin{proof}
For sake of a contradiction, assume $\neg P$.
\begin{center}
$\vdots$\\
(Use defintions and previous theorems to derive some $Q$ and its negation $\neg Q$.)\\
$\vdots$
\end{center}
\noindent This is a contradiction.  Therefore, $P$.
\end{proof}
\end{textbox}

\end{skeleton}

Among other situations, proof by contradiction can be useful for proving statements of the form $A\implies B$, where $B$ is worded negatively or $\neg B$ is easier to ``get your hands on."  

\begin{skeleton}[Proof of $A\implies B$ by contradiction]\label{pf by contradiction for implication}
If you want to prove the proposition $A\implies B$ via a proof by contradiction, then the structure of the proof is as follows.

\bigskip

\begin{textbox}
\begin{proof}
For sake of a contradiction, assume $A$ and $\neg B$.
\begin{center}
$\vdots$\\
(Use defintions and previous theorems to derive some $Q$ and its negation $\neg Q$.)\\
$\vdots$
\end{center}
\noindent This is a contradiction.  Therefore, if $A$, then $B$.
\end{proof}
\end{textbox}
\end{skeleton}

\begin{question}
In Skeleton Proof \ref{pf by contradiction for implication}, why did we start by assuming $A$ and $\neg B$?
\end{question}

Prove the following theorem in two ways: (i) prove the contrapositive, and (ii) prove using a proof by contradiction.

\begin{theorem}[*]
Assume that $x\in\mathbb{Z}$.  If $x$ is odd, then 2 does not divide $x$. (Prove in two different ways.)
\end{theorem}

Prove the following theorem by contradiction.

\begin{theorem}[*]
Assume that $x,y\in\mathbb{N}$.  If $x$ divides $y$, then $x\leq y$. (Prove using a proof by contradiction.)
\end{theorem}

\begin{question}
What obstacles (if any) are there to proving the previous theorem directly without using proof by contradiction?
\end{question}

\end{subsection}

\end{section}

\end{document}