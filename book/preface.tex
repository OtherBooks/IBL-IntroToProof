\begin{section*}{What is Meant by Foundations?}

The foundations of mathematics refers to logic and set theory; the axioms of number and space.  Also, it refers to an introduction to the techniques of proof, and at a larger level the process of \emph{doing Mathematics}.  Proof is central to doing mathematics.

Up to this point, it is likely that your experience of mathematics has been about using formulas and algorithms. That is only one part of mathematics. Mathematicians do much more than just use formulas.  Mathematicians experiment, make conjectures, write definitions, and prove theorems.  In this class, then, we will learn about doing all of these things.

What will this class require?  Daily practice.  Just like learning to play an instrument or sport, you will have to learn new skills and ideas.  Sometimes you'll feel good, sometimes frustrated.  You'll probably go through a range of feelings from being exhilarated, to being stuck.  Figuring it out, victories, defeats, and all that is part of real life is what you can expect.  Most importantly it will be rewarding.  Learning mathematics requires dedication.  It will require that you be patient despite periods of confusion.  It will require that you persevere in order to understand.  As the instructor, I am here to guide you, but I cannot do the learning for you, just as music teacher cannot move your fingers and your heart for you.  Only you can do that.  I can give suggestions, structure the course to assist you, and try to help you figure out how to think through things.  Do your best, be prepared to put in a lot of time, and do all the work.  Ask questions in class, ask questions in office hours, and ask your classmates questions.  When you work hard and you come to understand, you feel good about yourself.  In the meantime, you have to believe that your work will pay off in intellectual development.

How will this class be organized?  You have probably heard that mathematics is not a spectator sport.  Our focus in this class is on learning to DO mathematics, not learning to sit patiently while others do it.  Therefore, class time will be devoted to working on problems, and especially on students presenting conjectures and proofs to the class, asking questions of presenters in order to understand their work and their thinking, and sharing and clarifying our thinking and understanding of each other's ideas.  

The class is fueled by your ability to prove theorems and share your ideas.  As we progress, you will find that you have ideas for proofs, but you are unsure of them.  In that case, you can either bring your idea to the class, or you can bring it to office hours.  By coming to office hours, you have a chance to refine your ideas and get individual feedback before bringing them to the class.  The more you use office hours, the more you will learn.  If the whole class is stuck, we can work on some ego-booster problems to get your ideas flowing.

Finally, this is a very exciting time in your mathematical career.  It's where you learn what mathematics is really about!

\end{section*}

\begin{section*}{Your Toolbox, Questions, and Observations}

Throughout the semester, we will develop a list of \emph{tools} that will help you understand and do mathematics. Your job is to keep a list of these tools, and it is suggested that you keep a running list someplace.

Next, it is of utmost importance that you work to understand every proof. (Every!)  Questions are often your best tool for determining whether you understand a proof.  Therefore, here are some sample questions that apply to any proof that you should be prepared to ask of yourself or the presenter:
\begin{itemize}
\item What method(s) of proof are you using?
\item What form will the conclusion take?
\item How did you know to set up that [equation, set, whatever]?
\item How did you figure out what the problem was asking?
\item Was this the first thing you tried?
\item Can you explain how you went from this line to the next one?
\item What were you thinking when you introduced this?
\item Could we have \ldots instead?
\item Would it be possible to \ldots?
\item What if \ldots?
\end{itemize}

Another way to help you process and understand proofs is to try and make observations and connections between different ideas, proof statements and methods, and to compare approaches used by different people. Observations might sound like some of the following:
\begin{itemize}
\item When I tried this proof, I thought I needed to \ldots But I didn't need that, because \ldots
\item I think that \ldots is important to this proof, because \ldots
\item When I read the statement of this theorem, it seemed similar to this earlier theorem. Now I see that it [is/isn't] because \ldots
\end{itemize}

Lastly, it is highly important to respect learning and to respect other people's ideas.  Whether you disagree or agree, please praise and encourage your fellow classmates.  Use ideas from others as a starting point rather than something to be judgmental about.  Judgement is not the same as being judgmental.  Helpfulness, encouragement, and compassion are highly valued.

\end{section*}

\begin{section*}{An Inquiry-Based Approach}

This course will likely be different than any other math class that you have taken before for two main reasons. First, you are used to being asked to do things like: ``solve for x", ``take the derivative of this function", ``integrate this function", etc. Accomplishing tasks like these usually amounts to mimicking examples that you have seen in class or in your textbook. The steps you take to ``solve" problems like these are always justified by mathematical facts (theorems), but rarely are you paying explicit attention to when you are actually using these facts. Furthermore, justifying (i.e., proving) the mathematical facts you use may have been omitted by the instructor. And, even if the instructor did prove a given theorem, you may not have taken the time or have been able to digest the content of the proof.

Unlike previous courses, this course is all about ``proof". Mathematicians are in the business of proving theorems and this is exactly our endeavor. For the first time, you will be exposed to what ``doing" mathematics is really all about. This will most likely be a shock to your system. Considering the number of math courses that you have taken before you arrived here, one would think that you have some idea what mathematics is all about. You must be prepared to modify your paradigm. The second reason why this course will be different for you is that the method by which the class will run and the expectations I have of you will be different. In a typical course, math or otherwise, you sit and listen to a lecture. (Hopefully) These lectures are polished and well-delivered. You may have often been lured into believing that the instructor has opened up your head and is pouring knowledge into it. I absolutely love lecturing and I do believe there is value in it, but I also believe that in reality most students do not learn by simply listening. You must be active in the learning you are doing. I'm sure each of you have said to yourselves, ``Hmmm, I understood this concept when the professor was going over it, but now that I am alone, I am lost." In order to promote a more active participation in your learning, we will incorporate ideas from an educational philosophy called the Moore method (after R.L. Moore). Modifications of the Moore method are also referred to as inquiry-based learning (IBL) or discovery-based learning.

Loosely speaking, IBL is a student-centered method of teaching mathematics. At the college-level, one form of IBL is the \href{http://legacyrlmoore.org/method.html}{Modified Moore Method}, named after R.L. Moore.  In 1966, Moore wrote

\begin{quote}
\emph{That student is taught the best who is told the least.}
\end{quote}

According to the \href{http://www.inquirybasedlearning.org}{Academy of Inquiry-Based Learning}, IBL engages students in sense-making activities.  Students are given tasks requiring them to:
\begin{itemize}
\item solve problems,
\item conjecture,
\item experiment,
\item explore,
\item create,
\item communicate.
\end{itemize}

\noindent Rather than showing facts or a clear, smooth path to a solution, the instructor guides and mentors students via well-crafted problems through an adventure in mathematical discovery.  Effective IBL courses encourage deep engagement in rich mathematical activities and provide opportunities to collaborate with peers (either through class presentations or group-oriented work).

Perhaps this is sufficiently vague, but I believe that there are two essential elements to IBL:

\begin{enumerate}
\item Students should as much as possible be responsible for guiding the acquisition of knowledge.
\item Students should as much as possible be responsible for validating the ideas presented.  That is, students should not be looking to the instructor as the sole authority.
\end{enumerate}

\noindent For me, the guiding principle of IBL is the following question:

\begin{quote}
\emph{Where do I draw the line between content I must impart to my students versus the content they can produce independently?}
\end{quote}

\noindent E. Lee May (Salisbury State University) may have said it best:

\begin{quote}
\emph{Inquiry-based learning (IBL) is a method of instruction that places the student, the subject, and their interaction at the center of the learning experience.  At the same time, it transforms the role of the teacher from that of dispensing knowledge to one of facilitating learning.  It repositions him or her, physically, from the front and center of the classroom to someplace in the middle or back of it, as it subtly yet significantly increases his or her involvement in the thought-processes of the students.}
\end{quote}

\noindent For additional information, see AIBL's \href{http://www.inquirybasedlearning.org/?page=What_is_IBL}{What is IBL?} and \href{http://www.inquirybasedlearning.org/?page=Why_Use_IBL}{Why use IBL?} pages.

Much of the course will be devoted to students proving theorems on the board and a significant portion of your grade will be determined by how much mathematics you produce. I use the word ``produce" because I believe that the best way to learn mathematics is by doing mathematics. Someone cannot master a musical instrument or a martial art by simply watching, and in a similar fashion, you cannot master mathematics by simply watching; you must do mathematics!

Furthermore, it is important to understand that proving theorems is difficult and takes time. You should not expect to complete a single proof in 10 minutes. Sometimes, you might have to stare at the statement for an hour before even understanding how to get started. 

In this course, everyone will be required to

\begin{itemize}
\item read and interact with course notes on your own;
\item write up quality proofs to assigned problems;
\item present proofs on the board to the rest of the class;
\item participate in discussions centered around a student's presented proof;
\item call upon your own prodigious mental faculties to respond in flexible, thoughtful, and creative ways to problems that may seem unfamiliar on first glance.
\end{itemize}

\noindent As the semester progresses, it should become clear to you what the expectations are. This will be new to many of you and there may be some growing pains associated with it.

\end{section*}

\begin{section*}{Structure of the Notes}

As you read the notes, you will be required to digest the material in a meaningful way.  It is your responsibility to read and understand new definitions and their related concepts.  However, you will be supported in this sometimes difficult endeavor. In addition, you will be asked to complete exercises aimed at solidifying your understanding of the material.  Most importantly, you will be asked to make conjectures, produce counterexamples, and prove theorems.

Most items in the notes are labelled with a number.  The items labelled as \textbf{Definition} and \textbf{Example} are meant to be read and digested.  However, the items labelled as \textbf{Exercise}, \textbf{Question}, \textbf{Theorem}, \textbf{Corollary}, and \textbf{Problem} require action on your part.  In particular, items labelled as \textbf{Exercise} are typically computational in nature and are aimed at improving your understanding of a particular concept.  There are very few items in the notes labelled as \textbf{Question}, but in each case it should be obvious what is required of you.  Items with the \textbf{Theorem} and \textbf{Corollary} designation are mathematical facts and the intention is for you to produce a valid proof of the given statement.  The main difference between a \textbf{Theorem} and \textbf{Corollary} is that corollaries are typically statements that follow quickly from a previous theorem.  In general, you should expect corollaries to have very short proofs.  However, that doesn't mean that you can't produce a more lengthy yet valid proof of a corollary.  The items labelled as \textbf{Problem} are sort of a mixed bag.  In many circumstances, I ask you to provide a counterexample for a statement if it is false or to provide a proof if the statement is true.  Usually, I have left it to you to determine the truth value.  If the statement for a problem is true, one could relabel it as a theorem.

It is important to point out that there are very few examples in the notes.  This is intentional.  One of the goals of the items labelled as \textbf{Exercise} is for you to produce the examples.

Lastly, there are many situations where you will want to refer to an earlier definition or theorem/corollary/problem.  In this case, you should reference the statement by number.  For example, you might write something like, ``By Theorem 1.13, we see that\ldots."

\end{section*}

\begin{section*}{Some Minimal Guidance}
Especially in the opening sections, it won't be clear what facts from your prior experience in mathematics we are ``allowed" to use.  Unfortunately, addressing this issue is difficult and is something we will sort out along the way.  However, in general, here are some minimal and vague guidelines to keep in mind.  

First, there are times when we will need to do some basic algebraic manipulations.  You should feel free to do this whenever the need arises.  But you should show sufficient work along the way.  You do not need to write down justifications for basic algebraic manipulations (e.g., adding 1 to both sides of an equation, adding and subtracting the same amount on the same side of an equation, adding like terms, factoring, basic simplification, etc.).  

On the other hand, you do need to make explicit justification of the logical steps in a proof.  When necessary, you should cite a previous definition, theorem, etc. by number.

Unlike the experience many of you had writing proofs in geometry, our proofs will be written in complete sentences.  You should break sections of a proof into paragraphs and use proper grammar.  There are some pedantic conventions for doing this that I will point out along the way.  Initially, this will be an issue that most students will struggle with, but after a few weeks everyone will get the hang of it.

Ideally, you should rewrite the statements of theorems before you start the proof.  Moreover, for your sake and mine, you should label the statement with the appropriate number.  I will expect you to indicate where the proof begins by writing ``\emph{Proof.}" at the beginning.  Also, we will conclude our proofs with the standard ``proof box" (i.e., $\square$ or $\blacksquare$), which is typically right-justified.

Lastly, every time you write a proof, you need to make sure that you are making your assumptions crystal clear.  Sometimes there will be some implicit assumptions that we can omit, but at least in the beginning, you should get in the habit of stating your assumptions up front.  Typically, these statements will start off ``Assume\ldots" or ``Let\ldots".  

This should get you started.  We will discuss more as the semester progresses.  Now, go have fun and kick some butt!

\end{section*}
