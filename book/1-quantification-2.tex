\begin{section}{More on Quantification}

In the last section, we introduced the universal quantifier ``for all'' and the existential quantifier ``there exists\ldots such that.''  Here are a couple of important points to remember about quantification:
\begin{enumerate}
\item In order to have a proposition, all variables must be bound.  That is, all variables must be quantified.  This can happen in at least two ways:
\begin{enumerate}
\item The variables are explicitly bound by quantifiers in the same sentence, or
\item The variables are implicitly bound by preceding sentences and/or by context.  \emph{Note:}  Statements of the form ``Let $x=\ldots$" and ``Let $x\in\ldots$" bind the variable $x$ and remove ambiguity.
\end{enumerate}
\item The order of the quantification is important.  Reversing the order of the quantifiers can substantially change the meaning of a proposition.
\end{enumerate}

Using our logical connectives (``and", ``or", ``If\ldots, then\ldots", and ``not") together with quantification, we can build very complex mathematical statements.

\begin{example}\label{ex:def limit}
Let $f$ be a function and consider the formal definition of the
calculus statement: $\lim_{x\to c}f(x)=L$:

\begin{quote}
For all $\epsilon >0$, there exists $\delta >0$ such that for all $x$, if $0<|x-c|<\delta$, then $|f(x)-L|<\epsilon$.
\end{quote}
\end{example}

\begin{exercise}
Identify all the quantifiers from Example \ref{ex:def limit} and any logical connectives.  Are there any implicit bound variables?
\end{exercise}

In order to study the abstract nature of complicated mathematical statements, it is useful to adopt some notation.

\begin{definition}
We use the symbol $\forall$ to denote the universal quantifier ``for all" and the symbol $\exists$ to denote the existential quantifier ``there exists\ldots such that".\footnote{The \TeX\ symbol commands are \texttt{\textbackslash forall} and \texttt{\textbackslash exists}, respectively.}
\end{definition}

Using our abbreviations for the logical connectives and quantifiers, we can symbolically represent mathematical propositions.

\begin{example}
For each of the following, suppose our universe of discourse is the set of real numbers.
\begin{enumerate}

\item Consider the following (true) proposition:

\begin{quote}
There exists $x$ such that $x^2-1=0$.
\end{quote}

This proposition can be denoted symbolically as $(\exists x)(x^2-1=0)$.

\item Consider the following (false) proposition:

\begin{quote}
For all $x\in \mathbb{N}$, there exists $y\in\mathbb{N}$ such that $y<x$.
\end{quote}

This one can be represented symbolically as $(\forall x)(x\in\mathbb{N}\implies (\exists y)(y\in\mathbb{N}\implies y<x))$ or more simply as $(\forall x\in\mathbb{N})(\exists y\in\mathbb{N})(y<x)$.

\item Consider the following (true) proposition:

\begin{quote}
Every positive real number has a multiplicative inverse.
\end{quote}

There are several ways of representing this statement symbolically.  However, if you unpack what a multiplicative inverse is, you'll get something like $(\forall x)(x>0 \implies (\exists y)(xy=1))$.  Alternatively, you can shorten the statement to $(\forall x>0)(\exists y)(xy=1)$.

\end{enumerate}
\end{example}

\begin{exercise} Convert the following statements into statements using only logical symbols.  Assume that the universe of discourse is the set of real numbers.
\begin{enumerate}
\item There exists a number $x$ such that $x^2+1$ is greater than zero.
\item There exists a natural number $n$ such that $n^2=36$. 
\item For every real number $x$, $x^2$ is greater than or equal to zero.
\end{enumerate}
\end{exercise}

\begin{exercise}
Express the definition of the limit in Example \ref{ex:def limit} using only logic symbols.
\end{exercise}

\begin{remark}\label{rem:implicit universal}
If $A(x)$ and $B(x)$ are predicates, then it is standard practice for the statement $A(x)\implies B(x)$ to mean $(\forall x)(A(x)\implies B(x))$ (where the universe of discourse for $x$ needs to be made clear).  In this case, we say that the universal quantifier is implicit.
\end{remark}

\begin{exercise}
Find at least two examples earlier in the notes that exhibit the claim made in Remark \ref{rem:implicit universal}.  Attempt to write the statements symbolically using explicit quantifiers.
\end{exercise}

\begin{exercise}
Convert the following proposition into a statement using only logical symbols.  The universe of discourse is the set of real numbers.  (Watch out for implicit quantifiers.)
\begin{quote}
If $\epsilon >0$, then there exists $N\in\mathbb{N}$ such that $1/N<\epsilon$.
\end{quote}
Is this statement true?
\end{exercise}

\begin{exercise}
In the last exercise, you should end up with more than one quantifier.  Reverse the order of the quantifiers to get a new statement.  Does the meaning of the statement change?  If so, how does it change?  Is the new statement true?
\end{exercise}

\begin{remark}
The symbolic expression $(\forall x)(\forall y)$ can be replaced with the simpler expression $(\forall x,y)$ as long as $x$ and $y$ are coming from the same set.
\end{remark}

\begin{exercise}
For each of the following statements, (i) unpack the statement into words, and (ii) determine whether the statement is true or false.

\begin{enumerate}
\item $(\forall n \in \mathbb{N})(n^2 \geq 5)$
\item $(\exists n \in \mathbb{N})(n^2-1=0)$
\item $(\exists N \in \mathbb{N})(\forall  n > N)(\frac{1}{n} < 0.01)$
\item $(\forall m, n \in \mathbb{Z})(2|m \wedge 2|n \implies 2|(m+n))$
\item $(\forall x \in \mathbb{N})(\exists y \in \mathbb{N})(x-2y=0)$
\item $(\exists x \in \mathbb{N})(\forall y \in \mathbb{N})(y \leq x)$
\end{enumerate}
\end{exercise}

To whet your appetite for the next section, tackle the following questions.

\begin{question}
If a statement is false, then its negation is true.  How would you go about negating a statement involving quantifiers?  In particular, if $P(x)$ is a predicate, what are the negations of $(\forall x)(P(x))$ and $(\exists x)(P(x))$, respectively?
\end{question}

\end{section}


%%% Local Variables: 
%%% mode: latex
%%% TeX-master: "IntroToProof"
%%% End: 
