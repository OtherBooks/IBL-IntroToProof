\chapter{Set Theory and Topology}

\begin{section}{Introduction to Set Theory and Topology}

At its essence, all of mathematics is built on set theory.  In this chapter, we will introduce some of the basics of sets and their properties.

\begin{subsection}{Sets} 

\begin{definition}
A \textbf{set} is a collection of objects called \textbf{elements}. If $A$ is a set and $x$ is an element of $A$, we write
$x\in A$. Otherwise, we write $x\notin A$.
\end{definition}

\begin{definition}
The set containing no elements is called the \textbf{empty set}, and is denoted by the symbol $\emptyset$.
\end{definition}

If we think of a set as a box containing some stuff, then the empty set is a box with nothing in it.

\begin{definition} 
The language associated to sets is specific.  We will often define sets using the following notation, called \textbf{set builder notation}.
\[
S=\{x \in A: x \text{ satisfies some condition}\}
\]
The first part ``$x \in A$" denotes what type of $x$ is being considered.  The statements to the right of the colon are the conditions that $x$ must satisfy in order to be members of the set.  This notation is read as ``The set of all $x$ in $A$ such that $x$ satisfies some condition,'' where ``some condition" is something specific about the restrictions on $x$ relative to $A$.\footnote{Often, a vertical bar is used instead of a colon, like $S=\{x \in A \mid x \text{ satisfies some condition}\}$.}
\end{definition}

\begin{exercise}
Unpack each of the following sets into a description using a sentence and see if you can determine exactly what elements each set contains.
\begin{enumerate}
\item $M=\{x \in \mathbb{R} :  x \geq 2 \}$
\item $A=\{x \in \mathbb{N} : x = 3k \mbox{ for some } k\in \mathbb{N} \}$
\item $T=\{t \in \mathbb{R} : t^2 \leq 2 \}$
\item $H=\{t \in \mathbb{R} : t = 1 - \frac{1}{n} \mbox{, where } n \in \mathbb{N} \}$
\end{enumerate}
\end{exercise}

\begin{exercise}
Write each of the following sentences using set builder notation.
\begin{enumerate}
\item Suppose $R$ is the set of all real numbers $x$ such that $x$ is less than $-\sqrt{2}$. 
\item Suppose $A$ is the set of all real numbers $y$, such that $y$ is greater than $-12$ and less than 42.4.
\item Suppose $D$ is the set of all even natural numbers.
\end{enumerate}
\end{exercise}

\begin{definition}
If $A$ and $B$ are sets, then we say that $A$ is a \textbf{subset} of $B$, written $A\subseteq B$, provided that every element of $A$ is also an element of $B$.
\end{definition}

\begin{remark}
Observe that $A\subseteq B$ is equivalent to ``For all $x$ (in the universe of discourse), if $x\in A$, then $x\in B$."  Since we know how to deal with ``for all" statements and conditional propositions, we know how to go about proving $A\subseteq B$.
\end{remark}

\begin{question}
Suppose that $A$ and $B$ are sets.  Describe a general strategy for proving that $A\subseteq B$.
\end{question}

\begin{theorem}
Let $S$ be a set.  Then
\begin{enumerate}
\item $S\subseteq S$,
\item $\emptyset \subseteq S$.
\end{enumerate}
\end{theorem}

\begin{exercise}
List all of the subsets of $A=\{1,2,3\}$.  Any conjectures about how many there might be for a set with $n$ elements?
\end{exercise}

\begin{theorem}[*, Transitivity of subsets]
Suppose that $A$, $B$, and $C$ are sets.  If $A\subseteq B$ and $B\subseteq C$, then $A\subseteq C$.
\end{theorem}

\begin{definition}
If $A\subseteq B$, then $A$ is called a \textbf{proper subset} provided that $A\neq B$.  In this case, we may write $A\subset B$ or $A\subsetneq B$.\footnote{\emph{Warning:} Some books use $\subset$ to mean $\subseteq$.}
\end{definition}

\begin{definition}[Interval Notation]
For $a,b\in\mathbb{R}$ with $a<b$, we define the following.
\begin{enumerate}
\item $(a,b)=\{x\in\mathbb{R}:a<x<b\}$
\item $(a,\infty)=\{x\in\mathbb{R}:a<x\}$
\item $(-\infty,b)=\{x\in\mathbb{R}:x<b\}$
\item $[a,b]=\{x\in\mathbb{R}:a\leq x\leq b\}$
\end{enumerate}
We analogously define $[a,b)$, $(a,b]$, $[a,\infty)$, and $(-\infty,b]$.
\end{definition}

\begin{exercise}
Provide two examples of proper subsets of the interval $[0,1]$.
\end{exercise}

Here are some more definitions.  In each case, take $U$ to be the universe of discourse.

\begin{definition}
The \textbf{union} of the sets $A$ and $B$ is $A \cup B =\{x\in U : x\in A \mbox{ or } x\in B \}$.
\end{definition}

\begin{definition}
The \textbf{intersection} of the sets $A$ and $B$ is $A \cap B =\{x\in U : x\in A \mbox{ and } x\in B \}$.
\end{definition}

\begin{definition}
The \textbf{set difference} of the sets $A$ and $B$ is $A \setminus B =\{x\in U : x\in A \mbox{ and } x\notin B \}$.
\end{definition}

\begin{definition}
The \textbf{complement of $A$} (relative to $U$) is the set $A^c=U \setminus A =\{x \in U : x \notin A\}$.
\end{definition}

\begin{definition}
If two sets $A$ and $B$ have the property that $A \cap B = \emptyset$, then we say that $A$ and $B$ are \textbf{disjoint} sets.
\end{definition}

\begin{exercise}
Suppose that the universe of discourse is $U=\{1,2,3,4,5,6,7,8,9,10\}$.  Let $A=\{1, 2, 3, 4, 5\}$, $B=\{1, 3, 5\}$, and $C=\{2, 4, 6, 8\}$.  Find each of the following.
\begin{enumerate}
  \item $A \cap C$
  \item $A \cap B$
  \item $A \cup C$
  \item $A \cup B$
  \item $A\setminus B$
  \item $B \setminus A$
  \item $C \setminus B$
  \item $B \cap C$
  \item $B^c$
  \item $A^c$
  \item $(A\cup B)^c$
  \item $A^c\cap B^c$
\end{enumerate}
\end{exercise}

\begin{exercise}
Suppose that the universe of discourse is $U=\mathbb{R}$.  Let $A=[-3,-1)$, $B=(-2.5,2)$, and $C=(-2,0]$.  Find each of the following.
\begin{enumerate}
\item $A^c$
  \item $A \cap C$
  \item $A \cap B$
  \item $A \cup C$
  \item $A \cup B$
  \item $(A\cap B)^c$
  \item $(A\cup B)^c$
  \item $A \setminus B$
  \item $A\setminus (B \cup C)$
  \item $B \setminus A$
  \item $B \cap C$
\end{enumerate}
\end{exercise}

\begin{theorem}[*]
Let $A$ and $B$ be sets.  If $A \subseteq B$, then $B^c \subseteq A^c$.
\end{theorem}

\begin{definition}
Two sets $A$ and $B$ are \textbf{equal} if and only if $A \subseteq B$ and $B \subseteq A$.  In this case we write $A = B$.
\end{definition}

\begin{remark}
Given two sets $A$ and $B$, if we want to prove $A=B$, then we have to do two separate ``mini" proofs: one for $A\subseteq B$ and one for $B\subseteq A$.
\end{remark}

\begin{theorem}[*]
Let $A$ and $B$ be sets.  Then $A\setminus B = A \cap B^c$.
\end{theorem}

\begin{theorem}[*, DeMorgan's Law]
Let $A$ and $B$ be sets. Then 
\begin{enumerate}
\item $(A \cup B)^c = A^c \cap B^c$,
\item $(A \cap B)^c = A^c \cup B^c$.
\end{enumerate}
(You only need to prove one of these; the other is similar.)
\end{theorem}

\begin{theorem}[*, Distribution of Union and Intersection]
Let $A$, $B$, and $C$ be sets. Then
\begin{enumerate}
\item $A \cup(B\cap C) = (A\cup B)\cap (A\cup C)$,
\item $A\cap (B\cup C)= (A\cap B)\cup (A\cap C)$.
\end{enumerate}
(You only need to prove one of these; the other is similar.)
\end{theorem}

\end{subsection}

\end{section}

%%% Local Variables: 
%%% mode: latex
%%% TeX-master: "IntroToProof"
%%% End: 
